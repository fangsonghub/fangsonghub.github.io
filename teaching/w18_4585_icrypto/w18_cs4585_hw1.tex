% Created by Fang Song on April 4 2017. Instructions: you don't need
% to change anything in the macros, but feel free to define new
% commands as you wish. Starting from the main body, change the specs
% (e.g., your name). use \begin{solution} \end{solution} environment
% to write your solutions. Don't forget to list your collaborators.

\documentclass[12pt,answers]{exam}
%============Macros==================%
\usepackage{amsmath,amsfonts,amssymb,amsthm}
\usepackage{qcircuit}
\usepackage[margin=1in]{geometry}
%--------------Cosmetic----------------%
\usepackage{mathtools}
\usepackage{hyperref}
\usepackage{fullpage}
\usepackage{microtype}
\usepackage{xspace}
\usepackage[svgnames]{xcolor}
\usepackage[sc]{mathpazo}
\usepackage{enumitem}
\setlist[enumerate]{itemsep=1pt,topsep=2pt}
\setlist[itemize]{itemsep=1pt,topsep=2pt}
%----------Header--------------------%
\def\course{Winter 2018 CS 485/585 Introduction to Cryptography}
\def\term{Portland State University}
\def\prof{Lecturer: Fang Song}
\newcommand{\handout}[5]{
   \renewcommand{\thepage}{\arabic{page}}
   \begin{center}
   \framebox{
      \vbox{
    \hbox to 5.78in { \hfill \large{\course} \hfill }
       \vspace{2mm}
       \hbox to 5.78in { {\Large \hfill #5  \hfill} }
       \vspace{2mm}
       \hbox to 5.78in { \term \hfill \emph{#2}}
       \hbox to 5.78in { {#3 \hfill \emph{#4}}}
      }
   }
   \end{center}
   \vspace*{4mm}
}
\newcommand{\hw}[4]{\handout{#1}{#2}{#3}{#4}{Homework #1}}

%-----defs and commands-----%
\def\mG{[\textbf{G}]\xspace}
\def\veps{\varepsilon}
\newcommand{\klbk}[1]{[\texttt{KL}: #1]}
\newcommand{\N}{\ensuremath{\mathbb{N}}}
\newcommand{\Q}{\ensuremath{\mathbb{Q}}}
\newcommand{\R}{\ensuremath{\mathbb{R}}}
\newcommand{\T}{\ensuremath{\mathbb{T}}}
\newcommand{\Z}{\ensuremath{\mathbb{Z}}}
\newcommand{\algo}[1]{\ensuremath{\mathsf{#1}}}
% asymptotics
\DeclareMathOperator{\poly}{poly}
\DeclareMathOperator{\polylog}{polylog}
\DeclareMathOperator{\negl}{negl}
% pseudorandom stuff
\newcommand{\prg}{\algo{PRG}}
\newcommand{\prf}{\algo{PRF}}
\newcommand{\prp}{\algo{PRP}}
\newcommand{\owf}{\algo{OWF}}
% ADVERSARIES
\newcommand{\chg}{\ensuremath{CH}}
\newcommand{\attacker}[1]{\ensuremath{\mathcal{#1}}}
\newcommand{\Adv}{\attacker{A}}
\newcommand{\AdvA}{\attacker{A}}
\newcommand{\AdvB}{\attacker{B}}
% games
\newcommand{\privk}{\ensuremath{\textsf{PrivK}_{\attacker{A},\Pi}^{\textsf{eav}}}} %
\newcommand{\privkn}{\ensuremath{\textsf{PrivK}_{\attacker{A},\Pi}^{\textsf{eav}}(n)}} %
\newcommand{\privcpa}[1]{\ensuremath{\textsf{PrivK}_{#1}^{\textsf{cpa}}(n)}} %

%=======Main document==============%
\begin{document}

%----Specs: change accordingly-----%
\newif\ifstudent % comment out false
\studenttrue 
%\studentfalse

\def\issuedate{Jan. 09, 2018}
\def\duedate{Jan. 18, 2018} % 
\def\yourname{your name} % put your name here
%------------------------------%
\ifstudent
\hw{1}{\issuedate}{Student: \yourname}{Due: \duedate}%
\else
\hw{1}{\issuedate}{\prof}{Due: \duedate}%
\fi
\noindent \textbf{Instructions.}  Your solutions will be graded on
\emph{correctness} and \emph{clarity}. You should only submit work
that you believe to be correct; if you cannot solve a problem
completely, you will get significantly more partial credit if you
clearly identify the gap(s) in your solution. It is good practice to
start any long solution with an informal (but accurate) ``proof
summary'' that describes the main idea. %For this problem set, a
% random subset of problems will be graded.
Problems marked with ``\mG'' are required for graduate
students. Undergraduate students will get bonus points for solving
them. The .tex source is provided on course webpage as a template if
you want to typeset your solutions in Latex.

\medskip
\noindent You may collaborate with others on this problem set% and consult external sources
.  However, you must \textbf{\emph{write up your own solutions}} and
\textbf{\emph{list your collaborators}} for each problem. 

\begin{questions}
  \question (Probability)
  
  \begin{parts}
    \part (Exercise. Do not turn in.) Let $E[\cdot]$ be the
    expectation. Describe two random variables $X_1$ and $Y_1$ such
    that $E[X_1\cdot Y_1] = E[X_1]\cdot E[Y_1]$. Describe another two
    random variables $X_2$ and $Y_2$ such that
    $E[X_2\cdot Y_2] \neq E[X_2]\cdot E[Y_2]$.

    \part[5] Suppose there is a new test for a medical condition, and
    it has the following error rates:
    \begin{itemize}
    \item False positive: if you do not have the condition, there is a
      $5\%$ chance that the test comes positive.
    \item False negative: if you have the condition, there is $10\%$
      chance that the test comes negative. 
    \end{itemize}
    Assuming incidence of this condition occurs in $1\%$ of the
    population. How likely that a person tested positive indeed has
    this condition?
    % write your solutions in solution environment

    % \begin{solution}
      
    % \end{solution}
    
    \part[5] Consider a biased coin that appears HEADS with
    probability $p$. We flip it multiple times independently. What is
    the expected number of trials until you see HEADS for the first
    time?

    \part (Bonus 5pts) Suppose that a certain brand of cereal includes
    a free coupon in each box. There are $n$ different types of
    coupons. How many boxes do you expect to buy before finally
    getting a coupon of each type? (Hint: the harmonic number
    $ H(n) :=\sum_{i=1}^n\frac 1 i = \Theta(\log n)$.)

  \end{parts}

  \question (Asymptotic notations) Recall the following definitions
  \begin{itemize}
  \item A non-negative function $f \colon \N \to \R$ is
    \emph{polynomially bounded}, written $f(n) = \poly(n)$, if
    $f(n) = O(n^{c})$ for some constant $c \geq 0$.
  \item A non-negative function $\varepsilon : \N \to \R$ is
    \emph{negligible}, written $\varepsilon(n) = \negl(n)$, if it
    decreases faster than the inverse of any polynomial.  Formally:
    $\lim_{n \to \infty} \varepsilon(n) \cdot n^{c} = 0$ for any
    constant $c \geq 0$.  (Otherwise, we say that $\varepsilon(n)$ is
    \emph{non-negligible}.)
  \end{itemize}

  \begin{parts}
    \part[3] Is $\varepsilon(n) = 2^{- 10^6 \log^3 n}$ negligible or
    not?  Prove your answer.  (Why doesn't the base of the logarithm
    matter?)
    
    \part[3] Suppose that $\varepsilon(n) = \negl(n)$ and
    $f(n) = \poly(n)$.  Is it always the case that
    $f(n) \cdot \varepsilon(n) = \negl(n)$?  If so, prove it;
    otherwise, give concrete functions $\varepsilon(n), f(n)$ that
    serve as a counterexample.

  \end{parts}
  
  \question (Perfect Secrecy)

  \begin{parts}
    \part[8] \klbk{Exercise 2.5.} Prove Lemma 2.6. (Hint: you need to
    prove both directions.)

    \part[4] \klbk{Exercise 2.7} When using one-time pad with key
    $k=0^{\ell}$, we have $E_k(m)=k\oplus m =m$, and the message is
    sent in the clear! It is therefore suggested to modify
    one-time-pad by only using a random non-zero key $k$. Is this
    modified scheme still perfectly secret? Justify your answer.
    
  \end{parts}

  \question (Computational secrecy)
  \begin{parts}
    \part[5] \klbk{Exercise 3.2} Prove that Definition 3.8 CANNOT be
    satisfied if $\Pi$ can encrypt arbitrary-length messages and the
    adversary is NOT restricted to output equal-length messages in
    experiment $\privk$.
    \part[5] \mG \klbk{Exercise 3.3}   
  \end{parts}

  \question (Pseudorandom generators)

  \begin{parts}
    
  \part[8] \klbk{Exercise 3.5} 

  \part[4] Let $G$ be a pseudorandom generator with expansion factor
  $\ell(n) > 2n$. Define $G'(s):= G(s) \| G(s+1)$. Is $G'$ necessarily
  a pseudorandom generator? If so, give a proof; otherwise, show a
  counterexample.
    \end{parts}

\end{questions}


\end{document}
