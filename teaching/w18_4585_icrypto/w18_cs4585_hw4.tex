% Created by Fang Song on April 4 2017. Instructions: you don't need
% to change anything in the macros, but feel free to define new
% commands as you wish. Starting from the main body, change the specs
% (e.g., your name). use \begin{solution} \end{solution} environment
% to write your solutions. Don't forget to list your collaborators.

\documentclass[12pt,answers]{exam}
%============Macros==================%
\usepackage{amsmath,amsfonts,amssymb,amsthm}
\usepackage{qcircuit}
\usepackage[margin=1in]{geometry}
%--------------Cosmetic----------------%
\usepackage{mathtools}
\usepackage{hyperref}
% \usepackage{ulem}
\usepackage{cancel}
\usepackage{fullpage}
\usepackage{microtype}
\usepackage{xspace}
\usepackage[svgnames]{xcolor}
\usepackage[sc]{mathpazo}
\usepackage{enumitem}
\setlist[enumerate]{itemsep=1pt,topsep=2pt}
\setlist[itemize]{itemsep=1pt,topsep=2pt}
%----------Header--------------------%
\def\course{Winter 2018 CS 485/585 Introduction to Cryptography}
\def\term{Portland State University}
\def\prof{Lecturer: Fang Song}
\newcommand{\handout}[5]{
   \renewcommand{\thepage}{\arabic{page}}
   \begin{center}
   \framebox{
      \vbox{
    \hbox to 5.78in { \hfill \large{\course} \hfill }
       \vspace{2mm}
       \hbox to 5.78in { {\Large \hfill #5  \hfill} }
       \vspace{2mm}
       \hbox to 5.78in { \term \hfill \emph{#2}}
       \hbox to 5.78in { {#3 \hfill \emph{#4}}}
      }
   }
   \end{center}
   \vspace*{4mm}
}
\newcommand{\hw}[4]{\handout{#1}{#2}{#3}{#4}{Homework #1}}

%-----defs and commands-----%
\def\mG{[\textbf{G}]\xspace}
\def\veps{\varepsilon}
\newcommand{\bit}{\{0,1\}}
\newcommand{\klbk}[1]{[\texttt{KL}: #1]}
\newcommand{\N}{\ensuremath{\mathbb{N}}}
\newcommand{\Q}{\ensuremath{\mathbb{Q}}}
\newcommand{\R}{\ensuremath{\mathbb{R}}}
\newcommand{\T}{\ensuremath{\mathbb{T}}}
\newcommand{\Z}{\ensuremath{\mathbb{Z}}}
\newcommand{\algo}[1]{\ensuremath{\mathsf{#1}}}
% asymptotics
\DeclareMathOperator{\poly}{poly}
\DeclareMathOperator{\polylog}{polylog}
\DeclareMathOperator{\negl}{negl}
% pseudorandom stuff
\newcommand{\prg}{\algo{PRG}}
\newcommand{\prf}{\algo{PRF}}
\newcommand{\prp}{\algo{PRP}}
\newcommand{\owf}{\algo{OWF}}
% ADVERSARIES
\newcommand{\chg}{\ensuremath{CH}}
\newcommand{\attacker}[1]{\ensuremath{\mathcal{#1}}}
\newcommand{\Adv}{\attacker{A}}
\newcommand{\AdvA}{\attacker{A}}
\newcommand{\AdvB}{\attacker{B}}
% games
\newcommand{\privk}{\ensuremath{\textsf{PrivK}_{\attacker{A},\Pi}^{\textsf{eav}}}} %
\newcommand{\privkn}{\ensuremath{\textsf{PrivK}_{\attacker{A},\Pi}^{\textsf{eav}}(n)}} %
\newcommand{\privcpa}[1]{\ensuremath{\textsf{PrivK}_{#1}^{\textsf{cpa}}(n)}} %

%=======Main document==============%
\begin{document}

%----Specs: change accordingly-----%
\newif\ifstudent % comment out false
\studenttrue 
%\studentfalse

\def\hwnum{4}
\def\issuedate{Feb. 15, 2018} % 
\def\duedate{Mar. 1, 2018} % 
\def\yourname{your name} % put your name here
%------------------------------%
\ifstudent
\hw{\hwnum}{\issuedate}{Student: \yourname}{Due: \duedate}%
\else
\hw{\hwnum}{\issuedate}{\prof}{Due: \duedate}%
\fi
\noindent \textbf{Instructions.}  Your solutions will be graded on
\emph{correctness} and \emph{clarity}. You should only submit work
that you believe to be correct; if you cannot solve a problem
completely, you will get significantly more partial credit if you
clearly identify the gap(s) in your solution. It is good practice to
start any long solution with an informal (but accurate) ``proof
summary'' that describes the main idea. %For this problem set, a
% random subset of problems will be graded.
Problems marked with ``\mG'' are required for graduate
students. Undergraduate students will get bonus points for solving
them. The .tex source is provided on course webpage as a template if
you want to typeset your solutions in Latex.

\medskip
\noindent You may collaborate with others on this problem
set% and consult external sources
.  However, you must \textbf{\emph{write up your own solutions}} and
\textbf{\emph{list your collaborators}} for each problem.

\begin{questions}

  \question (Number Theory) Exercises. Do not turn in. 
  \begin{parts}
    \part Prove the following for the relation $ = \mod n$.
    \begin{enumerate}[label=\roman*.]
    \item If $a = b\bmod n$ and $b = c \mod n$, then $a = c\bmod n$.
    \item If $a = a' \bmod n$ and $b = b' \bmod n$, then $a+b = a'+b'
      \bmod n$.
    \item If $a = a' \bmod n$ and $b = b' \bmod n$, then
      $ab= a'b' \bmod n$.
    \end{enumerate}
    \part Find $gcd(60,17)$ and $(\alpha,\beta)$ such that
    $\alpha\cdot 60 + \beta\cdot 17 = gcd(60,17)$ by (extended)
    Euclidean algorithm.
    \part Prove that for $p,q$ co-prime, the Euler function
    $\phi(pq) = \phi(p)\cdot \phi(q)$.  Let $p$ be prime and $k>0$ an
    integer, prove that $\phi(p^k) = (p-1)p^{k-1}$. 
  \end{parts}

  \question (Factoring and RSA)
  \begin{parts}
    \part[6] Let $N=pq$ be a product of two distinct primes. Show that if
    $\phi(N)$ and $N$ are known, then it is possible to compute $p$
    and $q$ in polynomial time. (Hint: derive a quadratic equation
    over the integers in the unknown p.)
    
    \part[6] (Attacking plain-RSA encryption when encrypting short
    messages using small $e$.) Let $e=3$ and $N$ be a modulus of length
    1024 bits. Consider encrypting a message $m$ of length 100-bit
    using plain-RSA. Describe an attack that recovers the plaintext
    $m$ from its ciphertext. Which messages does your attack apply as
    well?

    \part[5] (Homomorphism.) Suppose you know ciphertexts $c_1$ and $c_2$
    for two messages $m_1$ and $m_2$ under plain-RSA encryption. Can
    you get the ciphertext for $m : = m_1\cdot m_2 \bmod N$? 
    
  \end{parts}
  \question (Discrete log, Diffie-Hellman, and El Gamal)
  \begin{parts}
    \part[5] Let $G$ be a finite group. The \emph{order} of an
    element $g\in G$ is defined as the smallest positive integer $i$
    such that $g^i = 1$. Prove that $g^x = g^y$ if and only if
    $x = y \bmod i$.
    
    \part[10] \klbk{Exercise 11. 6}
    
    \part[6] \klbk{Exercise 11.10}

    \part (Bonus 5pts) Read \klbk{Claim 11.7}. Let $\Pi = (G,E,D)$ be
    a CCA-secure public-key encryption scheme. Construct an encryption
    scheme $\Pi' = (G',E',D')$ such that
    $E'_{pk}(m_1,\ldots,m_\ell): =
    E_{pk}(m_1),\ldots,E_{pk}(m_\ell)$. Is $\Pi'$ CCA-secure? Justify
    your answer.
  \end{parts}

  \question (Bonus 10pts. Hybrid Encryption) Let $\Pi_1 = (G,E,D)$ be
  a CCA-secure public-key encryption scheme and $\Pi'= (G',E',D')$ a
  CCA-secure private-key encryption. Construct hybrid public-key
  encryption $\tilde \Pi = (\tilde G,\tilde E,\tilde D)$
  \begin{itemize}
  \item $\tilde G = G$: run $(pk,sk)\gets G(1^n)$.
  \item $\tilde E$: to encrypt $m$, run $k\gets G'(1^n)$, and compute
    $c_1 = E_{pk}(k)$ and $c_2 = E'_k(m)$. Output $c = (c_1,c_2)$. 
  \item $\tilde D$: to decrypt $c = (c_1,c_2)$, compute
    $k := D_{pk}(c_1)$, and output $m := D'_k(c_2)$.
  \end{itemize}
  Prove that $\tilde \Pi$ is CCA-secure. (Hint: read the CPA-security
  proof \klbk{Theorem 11.12})

  
  \question (Attacking plain-RSA signature) Consider the signature
  scheme (plain-RSA): run $\textbf{GenRSA}(1^n)$ to generate $(e,d)$
  with $ed = 1 \bmod N$, and make $e$ public and keep $d$ secret; to
  sign a message $m \in \mathbb{Z}_N^*$, compute
  $\sigma: = [m^d \bmod N]$; to verify $(m,\sigma)$, accept iff.
  $m = [\sigma^e \bmod N]$. 

  \begin{parts}
    \part[6] Show how to forge a valid signature on an arbitrary message
    $m$, by querying a signing oracle \emph{twice}.

    \part[6] Show how to forge a valid signature on an arbitrary message
    $m$, by querying a signing oracle \emph{once}.
  \end{parts}
  
\end{questions}


\end{document}
