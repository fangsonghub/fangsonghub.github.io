% Created by Fang Song on April 4 2017. Instructions: you don't need
% to change anything in the macros, but feel free to define new
% commands as you wish. Starting from the main body, change the specs
% (e.g., your name). use \begin{solution} \end{solution} environment
% to write your solutions. Don't forget to list your collaborators.

\documentclass[12pt,answers]{exam}
%============Macros==================%
\usepackage{amsmath,amsfonts,amssymb,amsthm}
\usepackage{qcircuit}
\usepackage[margin=1in]{geometry}
%--------------Cosmetic----------------%
\usepackage{mathtools}
\usepackage{hyperref}
% \usepackage{ulem}
\usepackage{cancel}
\usepackage{fullpage}
\usepackage{microtype}
\usepackage{xspace}
\usepackage[svgnames]{xcolor}
\usepackage[sc]{mathpazo}
\usepackage{enumitem}
\setlist[enumerate]{itemsep=1pt,topsep=2pt}
\setlist[itemize]{itemsep=1pt,topsep=2pt}
%----------Header--------------------%
\def\course{Winter 2018 CS 485/585 Introduction to Cryptography}
\def\term{Portland State University}
\def\prof{Lecturer: Fang Song}
\newcommand{\handout}[5]{
   \renewcommand{\thepage}{\arabic{page}}
   \begin{center}
   \framebox{
      \vbox{
    \hbox to 5.78in { \hfill \large{\course} \hfill }
       \vspace{2mm}
       \hbox to 5.78in { {\Large \hfill #5  \hfill} }
       \vspace{2mm}
       \hbox to 5.78in { \term \hfill \emph{#2}}
       \hbox to 5.78in { {#3 \hfill \emph{#4}}}
      }
   }
   \end{center}
   \vspace*{4mm}
}
\newcommand{\hw}[4]{\handout{#1}{#2}{#3}{#4}{Homework #1}}

%-----defs and commands-----%
\def\mG{[\textbf{G}]\xspace}
\def\veps{\varepsilon}
\newcommand{\bit}{\{0,1\}}
\newcommand{\klbk}[1]{[\texttt{KL}: #1]}
\newcommand{\N}{\ensuremath{\mathbb{N}}}
\newcommand{\Q}{\ensuremath{\mathbb{Q}}}
\newcommand{\R}{\ensuremath{\mathbb{R}}}
\newcommand{\T}{\ensuremath{\mathbb{T}}}
\newcommand{\Z}{\ensuremath{\mathbb{Z}}}
\newcommand{\algo}[1]{\ensuremath{\mathsf{#1}}}
% asymptotics
\DeclareMathOperator{\poly}{poly}
\DeclareMathOperator{\polylog}{polylog}
\DeclareMathOperator{\negl}{negl}
% pseudorandom stuff
\newcommand{\prg}{\algo{PRG}}
\newcommand{\prf}{\algo{PRF}}
\newcommand{\prp}{\algo{PRP}}
\newcommand{\owf}{\algo{OWF}}
% ADVERSARIES
\newcommand{\chg}{\ensuremath{CH}}
\newcommand{\attacker}[1]{\ensuremath{\mathcal{#1}}}
\newcommand{\Adv}{\attacker{A}}
\newcommand{\AdvA}{\attacker{A}}
\newcommand{\AdvB}{\attacker{B}}
% games
\newcommand{\privk}{\ensuremath{\textsf{PrivK}_{\attacker{A},\Pi}^{\textsf{eav}}}} %
\newcommand{\privkn}{\ensuremath{\textsf{PrivK}_{\attacker{A},\Pi}^{\textsf{eav}}(n)}} %
\newcommand{\privcpa}[1]{\ensuremath{\textsf{PrivK}_{#1}^{\textsf{cpa}}(n)}} %

%=======Main document==============%
\begin{document}

%----Specs: change accordingly-----%
\newif\ifstudent % comment out false
\studenttrue 
%\studentfalse

\def\hwnum{3}
\def\issuedate{Feb. 1, 2018} % 
\def\duedate{Feb. 15, 2018} % 
\def\yourname{your name} % put your name here
%------------------------------%
\ifstudent
\hw{\hwnum}{\issuedate}{Student: \yourname}{Due: \duedate}%
\else
\hw{\hwnum}{\issuedate}{\prof}{Due: \duedate}%
\fi
\noindent \textbf{Instructions.}  Your solutions will be graded on
\emph{correctness} and \emph{clarity}. You should only submit work
that you believe to be correct; if you cannot solve a problem
completely, you will get significantly more partial credit if you
clearly identify the gap(s) in your solution. It is good practice to
start any long solution with an informal (but accurate) ``proof
summary'' that describes the main idea. %For this problem set, a
% random subset of problems will be graded.
Problems marked with ``\mG'' are required for graduate
students. Undergraduate students will get bonus points for solving
them. The .tex source is provided on course webpage as a template if
you want to typeset your solutions in Latex.

\medskip
\noindent You may collaborate with others on this problem
set% and consult external sources
.  However, you must \textbf{\emph{write up your own solutions}} and
\textbf{\emph{list your collaborators}} for each problem.

\begin{questions}

  \question (More MACs)
  \begin{parts}
    \part[10] Let $F$ be a pseudorandom function. Decide if the
    following MACs are secure for authenticating fixed-length
    messages. Prove it or show an attack.

    
    \begin{enumerate}
    \item To authenticate $m = m[1], \ldots, m[\ell]$ where
      $m[i] \in \{0,1\}^n$, compute
      $t: = F_k(m[1])\oplus \cdots \oplus F_k(m[\ell])$.
    \item To authenticate $m = m[1], \ldots, m[\ell]$ where
      $m[i] \in \bit^{n/2}$, compute
      $t:= F_k(\langle 1 \rangle \| m[1])\oplus \cdots \oplus
      F_k(\langle \ell \rangle \| m[\ell])$. $\langle i \rangle$
      denotes the encoding of integer $i$ to a $n/2$-bit string.
    \end{enumerate}

    \part[10] Read the basic construction of CBC-MAC \klbk{Section
      4.4.1}, and do \klbk{Exercise 4.13}. 
  
    \end{parts}
    \question (Hash functions)
    \begin{parts}
      \part[10] \klbk{Exercise 5.2}

      \part[10]  Merkle-Damg{\aa}rd.

      \begin{enumerate}[label=\roman*.]
      \item Modify the Merkle-Damg{\aa}rd construction so that in the
        last block, instead of outputting $z: = h(z_B \| L)$, output
        $z_B\| L$. Is this construction collision resistant? Prove it
        or show an attack.
      \item \klbk{Exercise 5.8} Prove or disprove: if $h$ is
        preimage-resistant, then the hash function $H$ by applying the
        Merkle-Damg{\aa}rd transformation on $h$ is also
        preimage-resistant. 
      \end{enumerate}
      \part (5 Bonus points. Proof of work) Let
      $H: \bit^{2n} \to \bit^n$ be a hash function modeled as a truly
      random function. Suppose one can evaluate $H$ on one input each
      computer cycle. Given some $id \in \bit^n$ and an integer
      $T \leq 2^n$, how many cycles does it need to find an $x$ such
      that the first $\lceil \log T \rceil $ bits of $H(id \| x)$ are
      0?
    \end{parts}

    \question[10] (Authenticated Encryption) \klbk{Exercise 4.22}.

    \question (Foundations)
    \begin{parts}
      \part[5] (Computational indistinguishability) Recall the
      definition of computational indistinguishability
      \klbk{Def. 7.30}. Two probability ensembles
      $\mathcal{X} = \{X_n\}_{n\in\mathbb{N}}$ and
      $\mathcal{Y} = \{Y_n\}_{n\in\mathbb{N}}$ are computationally
      indistinguishable, denoted $\mathcal{X} \approx_c \mathcal{Y}$,
      if for any probabilistic polynomial-time distinguisher $D$,

    \begin{equation*}
      \left| \Pr_{x\gets X_n}[D(1^n,x) = 1] - \Pr_{y\gets
          Y_n}[D(1^n,y) = 1]\right| \leq \negl(n) \, .
    \end{equation*}

    Prove that the relation of computational indistinguishability is
    transitive: if $\mathcal{X}\approx_c \mathcal{Y}$ and
    $\mathcal{Y}\approx_c\mathcal{Z}$, then
    $\mathcal{X}\approx_c \mathcal{Z}$. 
   
    \part[10] Let $G$ be a pseudorandom generator with expansion
    factor $\ell(n) = n +1$.
    \begin{enumerate}[label=\roman*.]
    \item Prove that $G$ is a one-way function.
    \item (Exercise. Do not turn it.) Construct
      $G'(s) = G(s_1) \| G(s_2)$. Is $G'$ a secure pseudorandom
      generator?
    \item Let $G(s)= t_0,\ldots, t_{n-1} t_n$. Construct
      $G''(s) = G(t_0,\ldots,t_{n-1}) \| t_{n}$, i.e., use the first
      $n$-bit output as the seed in the second application of $G$. Is
      $G''$ a PRG? Prove or disprove it.

    \end{enumerate}
    \end{parts}

\end{questions}


\end{document}
