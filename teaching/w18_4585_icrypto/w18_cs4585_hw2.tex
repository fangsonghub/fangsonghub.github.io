% Created by Fang Song on April 4 2017. Instructions: you don't need
% to change anything in the macros, but feel free to define new
% commands as you wish. Starting from the main body, change the specs
% (e.g., your name). use \begin{solution} \end{solution} environment
% to write your solutions. Don't forget to list your collaborators.

\documentclass[12pt,answers]{exam}
%============Macros==================%
\usepackage{amsmath,amsfonts,amssymb,amsthm}
\usepackage{qcircuit}
\usepackage[margin=1in]{geometry}
%--------------Cosmetic----------------%
\usepackage{mathtools}
\usepackage{hyperref}
% \usepackage{ulem}
\usepackage{cancel}
\usepackage{fullpage}
\usepackage{microtype}
\usepackage{xspace}
\usepackage[svgnames]{xcolor}
\usepackage[sc]{mathpazo}
\usepackage{enumitem}
\setlist[enumerate]{itemsep=1pt,topsep=2pt}
\setlist[itemize]{itemsep=1pt,topsep=2pt}
%----------Header--------------------%
\def\course{Winter 2018 CS 485/585 Introduction to Cryptography}
\def\term{Portland State University}
\def\prof{Lecturer: Fang Song}
\newcommand{\handout}[5]{
   \renewcommand{\thepage}{\arabic{page}}
   \begin{center}
   \framebox{
      \vbox{
    \hbox to 5.78in { \hfill \large{\course} \hfill }
       \vspace{2mm}
       \hbox to 5.78in { {\Large \hfill #5  \hfill} }
       \vspace{2mm}
       \hbox to 5.78in { \term \hfill \emph{#2}}
       \hbox to 5.78in { {#3 \hfill \emph{#4}}}
      }
   }
   \end{center}
   \vspace*{4mm}
}
\newcommand{\hw}[4]{\handout{#1}{#2}{#3}{#4}{Homework #1}}

%-----defs and commands-----%
\def\mG{[\textbf{G}]\xspace}
\def\veps{\varepsilon}
\newcommand{\bit}{\{0,1\}}
\newcommand{\klbk}[1]{[\texttt{KL}: #1]}
\newcommand{\N}{\ensuremath{\mathbb{N}}}
\newcommand{\Q}{\ensuremath{\mathbb{Q}}}
\newcommand{\R}{\ensuremath{\mathbb{R}}}
\newcommand{\T}{\ensuremath{\mathbb{T}}}
\newcommand{\Z}{\ensuremath{\mathbb{Z}}}
\newcommand{\algo}[1]{\ensuremath{\mathsf{#1}}}
% asymptotics
\DeclareMathOperator{\poly}{poly}
\DeclareMathOperator{\polylog}{polylog}
\DeclareMathOperator{\negl}{negl}
% pseudorandom stuff
\newcommand{\prg}{\algo{PRG}}
\newcommand{\prf}{\algo{PRF}}
\newcommand{\prp}{\algo{PRP}}
\newcommand{\owf}{\algo{OWF}}
% ADVERSARIES
\newcommand{\chg}{\ensuremath{CH}}
\newcommand{\attacker}[1]{\ensuremath{\mathcal{#1}}}
\newcommand{\Adv}{\attacker{A}}
\newcommand{\AdvA}{\attacker{A}}
\newcommand{\AdvB}{\attacker{B}}
% games
\newcommand{\privk}{\ensuremath{\textsf{PrivK}_{\attacker{A},\Pi}^{\textsf{eav}}}} %
\newcommand{\privkn}{\ensuremath{\textsf{PrivK}_{\attacker{A},\Pi}^{\textsf{eav}}(n)}} %
\newcommand{\privcpa}[1]{\ensuremath{\textsf{PrivK}_{#1}^{\textsf{cpa}}(n)}} %

%=======Main document==============%
\begin{document}

%----Specs: change accordingly-----%
\newif\ifstudent % comment out false
\studenttrue 
%\studentfalse

\def\hwnum{2}
\def\issuedate{Jan. 18, 2018, Update: 01/25/18} % 
\def\duedate{Feb. 1, 2018} % 
\def\yourname{your name} % put your name here
%------------------------------%
\ifstudent
\hw{\hwnum}{\issuedate}{Student: \yourname}{Due: \duedate}%
\else
\hw{\hwnum}{\issuedate}{\prof}{Due: \duedate}%
\fi
\noindent \textbf{Instructions.}  Your solutions will be graded on
\emph{correctness} and \emph{clarity}. You should only submit work
that you believe to be correct; if you cannot solve a problem
completely, you will get significantly more partial credit if you
clearly identify the gap(s) in your solution. It is good practice to
start any long solution with an informal (but accurate) ``proof
summary'' that describes the main idea. %For this problem set, a
% random subset of problems will be graded.
Problems marked with ``\mG'' are required for graduate
students. Undergraduate students will get bonus points for solving
them. The .tex source is provided on course webpage as a template if
you want to typeset your solutions in Latex.

\medskip
\noindent You may collaborate with others on this problem
set% and consult external sources
.  However, you must \textbf{\emph{write up your own solutions}} and
\textbf{\emph{list your collaborators}} for each problem.

\begin{questions}
  \question (Pseudorandom Generators) Let $G$ be a pseudorandom
  generator with expansion factor $\ell(n)> 2n$. Let $F$ be a length
  preserving pseudorandom function. In each of the following cases,
  decide whether $G'$ is necessarily a pseudorandom generator. If yes,
  give a proof; if not, show a counterexample. Note: identify integers
  $\{0, 1, \ldots, 2^n-1\}$ with binary strings $\{0,1\}^n$, e.g.,
  $1 = 00\ldots01$. 
  
    
  \begin{parts}
    \part[6]
    $G'(s) := G(s_1,\ldots, \cancel{s_{\lfloor n/2 \rfloor}},
    s_{{\color{red}\lceil} n/2 {\color{red}\rceil}})$, where
    $s = s_1,\ldots, s_n$.
    \part[6] $G'(s) := G(0^{|s|} \| s)$.
    \part[6] $G'(s) := F_s(0) \| F_s(1)$.
    \part (Bonus 5pts) $G'(s) : = F_s(G(0)) \| F_s(G(1))$. 
  \end{parts}

  \question (Pseudorandom functions and pseudorandom permutations)

  \begin{parts}
    \part[15] \klbk{Exercise 3.10} Let $F$ be a length-preserving
    pseudorandom function. For the following constructions of a keyed
    function $F': \bit^n \times \bit^{n-1} \to \bit^{cn}$, $c = 1$ or
    $2$, decide if $F'$ is a pseudorandom function. If yes, prove it;
    if not, show an attack.
    \begin{enumerate}[label=\roman*.]
    \item $F'_k(x) : = F_k(0 \| x)$.
    \item $F'_k(x) : = F_k(0\| x) \| F_k(1 \| x)$.
    \item $F'_k(x): = F_k(0 \| x) \| F_k(x\|1 )$. 
    \end{enumerate}
    
    \part[5] \klbk{Exercise 3.16} Prove that a pseudorandom
    permutaiton is also a pseudorandom function when the block length
    $\ell(n) \geq n$.
  \end{parts}
  
  \question (Encryption against chosen-plaintext attacks)
  \begin{parts}
    \part[10] Read \klbk{Section 3.6.2} on CBC-mode and work out the
    following.

    \begin{enumerate}[label=\roman*.]
    \item Consider a stateful variant of CBC-mode where the sender
      simply increments the $IV$ by 1 each time a message is encrypted
      (rather than choosing it at random each time). Is this scheme
      CPA-secure? If yes, prove it; if not, describe an attack.
    \item Describe the effect of a single-bit error in the ciphertext
      when using CBC mode.

    \end{enumerate}
    
    \part[15] Let $F$ be a pseudorandom permutation on of block length
    $n$, and $G$ be a pseudorandom generator with 1-bit expansion
    $\ell(n) =n+1$. For each of the following encryption schemes,
    decide whether it is computationally secret or CPA-secure.
    \begin{enumerate}[label=\roman*.]
      \item To encrypt $m \in \bit^{n+1}$, choose uniform
      $r \gets \bit^n$, and output $(r, G(r) \oplus m)$.
    \item To encrypt $m \in \bit^n$, output $m \oplus F_k(0^n)$.
    \item To encryption $m \in \bit^n$, output $F_k(m)$. 
    \end{enumerate}
    \part[5] \mG \klbk{Exercise 3.29} 
  \end{parts}

  \question (Message authentication)
  \begin{parts}
    \part[12] Consider the definitions of a secure MAC (i.e.,
    existentially unforgeable under adpative chosen-message attacks
    \klbk{Def.4.2}) and a strongly-secure MAC (\klbk{Def.4.3}).
    \begin{enumerate}[label=\roman*.]
    \item \klbk{Exercise 4.4} Prove that for canonical MAC schemes, a
      secure MAC is also strongly secure.
    \item \klbk{Exercise 4.5} Assume secure MACs exist. Prove that
      there is a secure MAC, which is NOT strongly secure. 
    \end{enumerate}
    \part[5] \klbk{Exercise 4.6}
  \end{parts}
\end{questions}


\end{document}
