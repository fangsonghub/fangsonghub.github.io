\documentclass[12pt,answers,addpoints]{exam}

%============Macros==================%
\usepackage{amsmath,amsfonts,amssymb,amsthm}
\usepackage{qcircuit}
\usepackage[margin=1in]{geometry}
%--------------Cosmetic----------------%
\usepackage{mathtools}
\usepackage{hyperref}
\hypersetup{
    colorlinks=true,
    linkcolor=blue,
    filecolor=magenta,      
    urlcolor=cyan,
}
\usepackage{fullpage}
\usepackage{microtype}
\usepackage{xspace}
\usepackage[svgnames]{xcolor}
\usepackage[sc]{mathpazo}
\usepackage{enumitem}
\setlist[enumerate]{itemsep=1pt,topsep=2pt}
\setlist[itemize]{itemsep=1pt,topsep=2pt}
%-----defs and commands-----%
\def\mG{[\textbf{G}]\xspace}
\def\veps{\varepsilon}
\def\tr{\mathrm{tr}}
\newcommand{\bit}{\{0,1\}}
\newcommand{\bra}[1]{\langle #1 \rvert}
\newcommand{\ket}[1]{\lvert #1 \rangle}
\newcommand{\kera}[1]{\ket{#1}\bra{#1}}
\newcommand{\negl}{\text{negl}}
\newcommand{\complex}{\mathbb{C}}
\newcommand{\integer}{\mathbb{Z}}
\newcommand{\srd}[2]{\textsf{SR}_{#1}^{#2}(X)}
\newcommand{\corr}[1]{{\color{blue}{#1}}}
% \input{../head}
%----------Header--------------------%

\newcommand{\classn}{CSCE 440/640 Quantum Algorithms}
\newcommand{\classnabbr}{CSCE 440/640}
\newcommand{\school}{Texas A\&M U}
\newcommand{\term}{Spring 2019}
\newcommand{\examdate}{March 18, 2019}
\newcommand{\duedate}{March 20, 2019, 11:59pm, AoE}
\newcommand{\examnum}{Mid-term Exam}
\newcommand{\studentname}{\makebox[1.5in]{\hrulefill}} % change it to your name
\pagestyle{head}
\firstpageheader{}{}{}
\runningheader{\classnabbr}{\examnum\ - Page \thepage\ of
  \numpages}{\term}
\vskip 1ex
\setlength{\headsep}{10pt}
\runningheadrule

%\qzheader                       % execute quiz commands

\begin{document}

\noindent
\begin{tabular*}{\textwidth}{l @{\extracolsep{\fill}} r
    @{\extracolsep{6pt}} r}
  {\Large\textbf{\examnum}} & \Large{\textbf{Name:}} & \studentname\\
  {\term}, {\classn} & &  {\examdate}\\
  \school && Prof. Fang Song
\end{tabular*}\\

\rule[2ex]{\textwidth}{1pt}

\subsection*{Instructions (please read carefully before start!)}

\begin{itemize}
\item This take-home exam contains \numpages\ pages (including this
  cover page) and \numquestions\ questions. Total of points is
  \numpoints.
\item You will have till \textbf{\duedate}~(Anywhere on Earth) to
  finish the exam. You must work on your own, and no collaboration or
  help from any resources other than those made available in class
  (lecture notes, recommended texts, homework problems, etc.)  is
  permitted.

\item Email me your solutions in PDF before the deadline, either
  scanned or typeset in \LaTeX. Name your PDF and email subject as:
  \textbf{Lastname\_Firstname\_s19\_mt}. If you choose to hand-write
  and scan, \emph{print out this exam sheet and write your solutions
    on it}. Do your best to fit your answers into the space provided,
  and attach extra papers only if necessary. If you typeset in \LaTeX,
  \emph{use the provided TeX file}. No other formats are accepted.

\item Your work will be graded on correctness and clarity. Make sure
  your hand writing is legible.
\item Don't forget to write your name on top (or update the
  ``{\textbackslash}studentname'' command in the TeX file)!
\end{itemize}

\begin{center}
\textbf{Grade Table} (for instructor use only)\\
\smallskip
\addpoints
\gradetable[v][questions]
\end{center}

\newpage

\begin{questions}
  \question \emph{Short answers}. Answer the following, and briefly
  justify your answer.
  \begin{parts}
    \part[0] (Sample problem) Is
    $\sqrt{i/3} \ket{0} + \sqrt{2/5}\ket{1}$ a valid quantum state?

    \begin{solution}
      Answer: No.\\
      Justification: Because $|\sqrt{i/3}|^2 + |\sqrt{2/5}|^2 \neq 1$.
    \end{solution}

    \part[5] Is $\frac{1}{\sqrt{2}}(\ket{000} + \ket{111})$ an
    entangled state?

    \vskip 3cm 
    \part[5] Quantum computers can solve NP-Complete problems in
    polynomial time. Is this statement True/False/Unknown?

   \vskip 3cm % comment this line when typing your answer

   \part[5] Recall in the quantum superdense coding protocol, Alice
   wants to send two classical bits to Bob by sending one
   qubit. Suppose a third party (Eve) intercepts Alice's qubit on the
   way. Can Eve infer anything about which of the four possible bit
   strings 00, 01, 10, 11 Alice was trying to send?

   \vskip 4cm

   \part[5] What is the Quantum Fourier Transform $F_{2^4}$ on a
   four-qubit state
   $ \frac{1}{4}\ket{0000} + \frac{i}{4}\ket{0010} +
   \sqrt{\frac{5}{8}}\ket{1111}$?

 \end{parts}
  \newpage
  \question (Quantum circuits)

  \begin{parts}
    \part[10] Suppose you have an unlimited supply of qubits in the state
    $\ket{\psi} = \alpha\ket{0}+\beta\ket{1}$, and qubits in the state
    $\ket{0}$. Give quantum circuits and specify the inputs for
    producing the following quantum states:

    \begin{enumerate}[label=\roman*)]
    \item
      $\alpha^2\ket{00} - \alpha\beta\ket{01} + \alpha\beta\ket{10} -
      \beta^2\ket{11}$.

      \vskip 3cm 
    \item $\alpha\ket{00} - \beta\ket{11}$.  \vskip 3cm
    \end{enumerate}
    \part[20] For each pair of the circuits below, prove or disprove that
    they are equivalent.

    \begin{enumerate}[label=\roman*)]
      \item 
        \[ \Qcircuit @C=1em @R=.7em { & \gate{H} & \gate{X} &
            \gate{H}&\qw & \overset{?}{=} & & \gate{Z}
          & \qw } \] \vskip 4cm
      \item
        \[ \Qcircuit @C=1em @R=.7em {
            & \ctrl{2} & \qw  &  & & \gate{-Z}& \qw \\
            &&& \overset{?}{=} & & &\\
            & \gate{Z} & \qw &&&\ctrl{-2}& \qw} 
        \]
        \newpage
      \item
        \[ \Qcircuit @C=1em @R=.7em {
            &  \gate{H} & \ctrl{2} & \gate{H}& \qw  &  & & \targ & \qw \\
            &&&&& \overset{?}{=} & & &\\
            &\gate{H} & \targ & \gate{H} & \qw &&&\ctrl{-2}& \qw}
        \]
        \vskip 5cm
      \item Let $U,V$ be unitary, and $V^2 =U$. The left-hand-side is
        $U$ controlled by two qubits $\ket{a}\ket{b}$ such that $U$ is
        applied to the third qubit iff. $a = b = 1$.
        \[ \Qcircuit @C=1em @R=.7em { & \ctrl{2} & \qw & & & \qw &
            \ctrl{1}& \qw & \ctrl{1} &
            \ctrl{2} & \qw \\
            &\ctrl{1}& \qw & \overset{?}{=} & &\ctrl{1}& \targ& \ctrl{1}&\targ &\qw &\qw\\
            &\gate{U} & \qw & & &\gate{V}&\qw &\gate{V^\dagger}& \qw
            &\gate{V} & \qw }
        \]
        \vskip 5cm
      \end{enumerate}
      \part[5] Construct a CNOT gate from one controlled-Z gate and two
      Hadamard gates.
      \newpage
      \part[5] (Phase estimation: alternative) Let $U$ be an $n$-qubit
      unitary operator and $\ket{\psi}$ be an eigenvector with
      $U\ket{\psi} = e^{i \theta } \ket{\psi}$. Analyze the circuit
      below and derive the probability that the measurement outcome is
      $0$.

      \begin{figure}[h!]\label{fig:ape}
        \centerline{\Qcircuit @C=1em @R=.7em {
            \lstick{\ket{0}}&  \gate{H} & \ctrl{1} & \gate{H}& \meter &\cw  \\
            \lstick{\ket{\psi}} & \qw & \gate{U} &\qw & \qw & \qw
          }}
        \caption{Alternative phase estimation algorithm}
      \end{figure}

      \vskip 5cm
      \part (15 Bonus points) Continue from part (d). 
      \begin{enumerate}[label=\roman*)]
      \item How many times do we need to repeat the circuit in
        Figure~\ref{fig:ape} to get an estimate $\tilde \theta$ so
        that $|\theta - \tilde \theta | \leq \varepsilon$ with
        probability at least $1 - \delta$?  \vskip 5cm
      \item Suppose you can replace $U$ by $U^k$ for an arbitrary
        integer $k$ of your choice (still controlled by one
        qubit). Show how to approximate $\theta$.
      \end{enumerate}
    \end{parts}
    
  
  \newpage 
  \question (Quantum algorithms and permutations) A bijection
  $P: \bit^m\to \bit^m $ is called a permutation on $\bit^m$.

  \begin{parts}
    \part[5] How many permutations are there in total on $\bit^m$?
    \vskip 3cm

    \part[6] Let $S$ be the set of all permutations on
    $\bit^m$. Consider a subset $\{P_k\}\subseteq S$ which are indexed
    by $n$-bit keys $k\in \bit^n$ for some $n \leq m$. We are now
    given $(x_1,\ldots, x_t)$ and $(y_1,\ldots, y_t)$ with the promise
    that there is a \emph{unique} $k^*\in \bit^n$ such that
    $P_k(x_i) = y_i$ for all $i =1,\ldots, t$. The goal is to identify
    this key $k^*$.  Let us consider classical algorithms
    first. Suppose we have access to an oracle
    $O: (k,x)\mapsto P_k(x)$. How many queries are sufficient to
    determine $k^*$ in the worst case?  How many are necessary?
    Justify your answer.

      \vfill

      
    \part[14] Continue from above. Suppose $O$ can be queried in quantum
      superposition, i.e., it is given as a black-box quantum circuit
      implementing the unitary
      \[ U: \ket{k}\ket{x}\ket{y} \mapsto \ket{k}\ket{x}\ket{y\oplus
          P_k(x)}, \forall k\in \bit^n, x,y\in \bit^m \, .\]

      \begin{enumerate}[label=\roman*)]
      
      \item Show that one can implement another quantum oracle
        \[U': \ket{k} \ket{b} \mapsto \ket{k}\ket{b\oplus f(k)}\, ,\]
        where $f:\bit^n \to \bit$ is such that $f(k) = 1$ iff.
        $P_k(x_i) = y_i$ for all $i = 1,\ldots, t$. How many calls to
        $U$ are needed to answer one query to $U'$?
        \newpage
        \begin{center}
(Problem 3.c continued)
        \end{center}

        \vskip 8cm
      \item Give a quantum algorithm for finding $k^*$. Describe its
        cost in terms of \# of queries to $U$ and the circuit size.
      \end{enumerate}
      \vskip 6cm
      \part (15 Bonus points) Let $k_1,k_2\in \bit^n$ be two secret
      strings and $\pi: \bit^n\to \bit^n$ be a permutation. Define
      another permutation\footnote{This is the Even-Mansour block
        cipher.} $P_{k_1,k_2}: \bit^n \to \bit^n$ as
      \[ P_{k_1,k_2}: x\mapsto \pi(x\oplus k_1)\oplus k_2 \, .\]

      \begin{enumerate}[label=\roman*)]
      \item Define a function $f$ from $\pi$ and $P_{k_1,k_2}$ such
        that for all $x$, $f(x\oplus k_1) = x$.
        \newpage
      \item Suppose $\pi$ is sampled uniformly at random among all
        permutations on $\bit^n$. Given quantum oracles for $\pi$ and
        $P_{k_1,k_2}$, describe a quantum algorithm that recovers
        $k_1,k_2$ efficiently. 
      \end{enumerate}
  \end{parts}
    
\end{questions}
\newpage
\begin{center}
Scrap paper -- no exam questions here.  
\end{center}


\end{document}

%%% Local Variables: 
%%% mode: latex
%%% TeX-master: t
%%% End: 
