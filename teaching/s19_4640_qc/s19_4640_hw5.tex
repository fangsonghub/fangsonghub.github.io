% Instructions: you don't need to change anything in the macros, but
% feel free to define new commands as you wish. Starting from the main
% body, change the specs (e.g., your
% name). use \begin{solution} \end{solution} environment to write your
% solutions. Don't forget to list your collaborators.

\documentclass[12pt,answers]{exam}
%============Macros==================%
\usepackage{amsmath,amsfonts,amssymb,amsthm}
\usepackage{qcircuit}
\usepackage[margin=1in]{geometry}
%--------------Cosmetic----------------%
\usepackage{mathtools}
\usepackage{hyperref}
\hypersetup{
    colorlinks=true,
    linkcolor=blue,
    filecolor=magenta,      
    urlcolor=cyan,
}
\usepackage{fullpage}
\usepackage{microtype}
\usepackage{xspace}
\usepackage[svgnames]{xcolor}
\usepackage[sc]{mathpazo}
\usepackage{enumitem}
\setlist[enumerate]{itemsep=1pt,topsep=2pt}
\setlist[itemize]{itemsep=1pt,topsep=2pt}
%----------Header--------------------%
\def\course{CSCE 440/640 Quantum Algorithms}
\def\term{Texas A\&M U, Spring 2019}
\def\prof{Lecturer: Fang Song}
\newcommand{\handout}[5]{
   \renewcommand{\thepage}{\arabic{page}}
   \begin{center}
   \framebox{
      \vbox{
    \hbox to 5.78in { \hfill \large{\course} \hfill }
       \vspace{2mm}
       \hbox to 5.78in { {\Large \hfill \textbf{#5}  \hfill} }
       \vspace{2mm}
       \hbox to 5.78in { \term \hfill \emph{#2}}
       \hbox to 5.78in { {#3 \hfill \emph{#4}}}
      }
   }
   \end{center}
   \vspace*{4mm}
}
\newcommand{\hw}[4]{\handout{#1}{#2}{#3}{#4}{Homework #1}}
\newcommand{\ex}[4]{\handout{#1}{#2}{#3}{#4}{Exercise #1}}

%-----defs and commands-----%
\def\mG{[\textbf{G}]\xspace}
\def\veps{\varepsilon}
\def\tr{\mathrm{tr}}
\newcommand{\bit}{\{0,1\}}
\newcommand{\bra}[1]{\langle #1 \rvert}
\newcommand{\ket}[1]{\lvert #1 \rangle}
\newcommand{\kera}[1]{\ket{#1}\bra{#1}}
\newcommand{\negl}{\text{negl}}
\newcommand{\complex}{\mathbb{C}}
\newcommand{\integer}{\mathbb{Z}}
\newcommand{\srd}[2]{\textsf{SR}_{#1}^{#2}(X)}
\newcommand{\corr}[1]{{\color{blue}{#1}}}
%=======Main document==============%
\begin{document}

%----Specs: change accordingly-----%
\newif\ifstudent % comment out false
\studenttrue 
%\studentfalse

\def\issuedate{April 10, 2019}
\def\duedate{April 29, 2019, before class} %
\def\yourname{your name} % put your name here

%------------------------------%
\ifstudent
\hw{5}{\issuedate}{Student: \yourname}{Due: \duedate}%
\else
\hw{5}{\issuedate}{\prof}{Due: \duedate}%
\fi
\noindent \textbf{Instructions.} Only PDF format is accepted (type it
or scan clearly). Your solutions will be graded on \emph{correctness}
and \emph{clarity}. You should only submit work that you believe to be
correct; if you cannot solve a problem completely, you will get
significantly more partial credit if you clearly identify the gap(s)
in your solution. For this problem set, a random subset of problems
will be graded. Problems marked with ``\mG'' are required for graduate
students. Undergraduate students will get bonus points for solving
them.  \medskip

\noindent You may collaborate with others on this problem set. 
% and consult external sources.
However, you must \textbf{\emph{write up your own solutions}} and
\textbf{\emph{list your collaborators}} for each problem.

\begin{questions}

  \question (Quantum error-correcting) 
  \begin{parts}
    \part[15] Let $E$ be an arbitrary 1-qubit unitary, and $I,X,Y,Z$
    are the four $2\times 2$ Pauli matrices.
    \begin{enumerate}[label=\roman*)]
    \item Show that it can be written as
      $E = \alpha_0 I + \alpha_1X + \alpha_2 Y + \alpha_3 Z$, for some
      complex coefficients $\alpha_i$ with
      $\sum_{i=0}^3 |\alpha_i|^2 = 1$. (Hint: compute the trace
      $Tr(E^\dagger E)$ in two ways, and use the fact that
      $Tr(AB) = 0$ if $A$ and $B$ are distinct Pauli matricies, and
      $Tr(AB) = Tr(I) = 2$ if $A$ and $B$ are the same Pauli.)
    \item Write the 1-qubit Hadamard transform H as a linear combination
      of the four Pauli matrices.
    \item Suppose an $H$-error happens on the first qubit of
      $\alpha\ket{\bar 0} + \beta\ket{\bar 1}$ using the 9-qubit code.
      Give the various steps in the error-correction procedure that
      corrects this error.

      {Note:} $\ket{\bar b}$ represents a logical qubit, which is the
      encoded state of $\ket{b}$ under the considered code.
    \end{enumerate}
    \part[10] Show that there cannot be a quantum code that encodes
    one logical qubit by $2k$ physical qubits while being able to
    correct errors on up to $k$ of the qubits. (Hint: No-cloning
    theorem)
  \end{parts}
  % \question (Algorithms)
  % \begin{parts}
  %   \part search / decision factoring
  % \end{parts}

  \question (Learning parities) Let $s\in \bit^n$ be a secret $n$-bit
  string. Suppose $f: \bit^n \to \bit$ computes the dot product
  $f(x) = s \cdot x = \sum_{i = 1}^{n} s_i x_i \pmod 2$ (i.e., the
  parity of the bits in $s$ chosen by the non-zero positions of
  $x$). In this problem, we will (mainly) consider the query
  complexity of learning $s$.
  %by algorithms with zero-error, meaning that they always output the
  %correct answer.
  \begin{parts}
    \part[8] How many queries are needed to classically learn $s$ with
    zero-error (i.e., always outputting the correct answer)? Give an
    algorithm for this problem, and show that it is optimal.
    \part[7] Explain why even if we allow the classical algorithm to
    fail with some fixed probability (e.g., probability $1/3$), it
    requires the same asymptotic query complexity as the zero-error
    case.
    \part[10] How many queries are needed by a \emph{quantum} algorithm to
    learn $s$ with zero-error? Give an algorithm for this problem, and
    show that it is optimal. As usual, we assume a quantum oracle
    $O_f$: $\ket{x}\ket{y}\mapsto \ket{x}\ket{x\cdot s \pmod 2}$ is
    given. (Hint: Deutsch-Josza)
  \part (Bonus 10pts) Now consider a \emph{noisy} version $\tilde f$
    of $f$: $\tilde f(x) = x\cdot s + e_x \pmod 2$ where
    $e_x \in \bit$ is a random bit independently drawn for each $x$,
    and $b_x = 1$ with probability $\eta < 1/2$ (i.e., a biased coin
    and we denote it $\text{COIN}_\eta$). Given oracle access to
    $\tilde f$, how many queries are needed by a quantum algorithm for
    finding $s$ with probability at least $\Omega((1-2\eta)^2)$?
  \part (Bonus 15pts) Suppose that we no longer have oracle access to
    $f$. Instead we are given a sequence of classical samples
    $(x_i, y_i), i=1,\ldots, m$, where $x_i\gets \bit^n$ chosen
    uniformly at random and $y_i = s \cdot x_i + e_{x_i} \pmod 2$ with
    independent $e_{x_i}\gets \text{COIN}_\eta$. Let $m$ be a
    polynomial in $n$. Give a (quantum or classical) algorithm that
    runs in time polynomial in $n$ for finding $s$ assuming constant
    $\eta$ (e.g., $\eta = 1/4$).
  \end{parts}
  \question (Testing entanglement) Suppose that Alice and Bob share a
  two-qubit state, and they want to test if it is the EPR pair
  $\ket{\phi^+} := \frac{1}{\sqrt 2}(\ket{00}+\ket{11})$ with local
  measurements and classical communication. Consider the following
  procedure: they randomly select a measurement basis: with
  probability $1/2$, they both measure in the standard basis
  $\{\ket{0},\ket{1}\}$; and, with probability $1/2$, they both
  measure in the Hadamard (diagonal) basis $\{\ket{+},\ket{-}\}$. Then
  they perform the measurement and they accept if and only if their
  outcomes are the same.
  \begin{parts}
  \part[5] Show that the state $\ket{\phi^+}$ is always accepted by
    this test with zero-error.
    \part[8] Show that, for an arbitrary 2-qubit state $\ket{\mu}$, the
    probability that it passes the test is at most
    \begin{equation*}
      \frac{1 + |\langle{\mu}|{\phi^+}\rangle|^2}{2}   \, .
    \end{equation*}
    (Hint: decomposing $\ket{\mu}$ under the four Bell states.)
    \part[7] Now consider another (malicious) party Eve, who may have
    intervened with the state that Alice shares with Bob. Let
    $\rho_{ABE}$ be their joint state. Now assume that Alice and Bob
    are certain that they two perfectly share $\ket{\phi^+}$, show
    that Alice and Eve's state cannot be in $\ket{\phi^+}$ as well.

    Note: we can actually show that $\rho_{ABE}$ must be of form
    $\kera{\phi^+}_{AB}\otimes \rho_E$ (i.e., Eve's state is
    uncorrelated with that of Alice and Bob). This is an example of
    \emph{monogamy of entanglement}: the more system $A$ is entangled
    with $B$, the less $A$ is entangled with another system $C$.
  \end{parts}

\question (Distinguishing states)
  \begin{parts}
  \part[10] Explain how a device which, upon input of one of two
    non-orthogonal quantum states $\ket{\psi}$ or $\ket{\phi}$
    correctly identified the state, could be used to build a device
    which cloned the states $\ket{\psi}$ and $\ket{\phi}$. Conversely,
    explain how a device for cloning could be used to distinguish
    non-orthogonal quantum states.
  \part[5] Suppose Bob is given $\ket{\psi_0}= \ket{0}$ or
    $\ket{\psi_1} = \frac{1}{\sqrt 2} (\ket{0}+\ket{1})$. Help Bob
    design a POVM which distinguishes the states some of the time, but
    never makes an error of mis-identification.
  \part[5] \mG Suppose Bob is given a quantum state chosen from a set
    of linearly independent states
    $\{ \ket{\psi_0}, \ldots \ket{\psi_{m-1}}\}$. Construct a POVM
    $\{E_0, E_2,\ldots,E_{m}\}$ such that if outcome $E_i$ occurs,
    $0 \leq i \leq m-1$, then Bob knows with certainty that he was
    given the state $\ket{\psi_i}$. (The POVM must be such that
    $\langle \psi_i|E_i|\psi_i\rangle > 0$ for each $i$.)
  \end{parts}
\end{questions}
\end{document}


