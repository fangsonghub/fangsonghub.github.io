% Instructions: you don't need to change anything in the macros, but
% feel free to define new commands as you wish. Starting from the main
% body, change the specs (e.g., your
% name). use \begin{solution} \end{solution} environment to write your
% solutions. Don't forget to list your collaborators.

\documentclass[12pt,answers]{exam}
%============Macros==================%
\usepackage{amsmath,amsfonts,amssymb,amsthm}
\usepackage{qcircuit}
\usepackage[margin=1in]{geometry}
%--------------Cosmetic----------------%
\usepackage{mathtools}
\usepackage{hyperref}
\hypersetup{
    colorlinks=true,
    linkcolor=blue,
    filecolor=magenta,      
    urlcolor=cyan,
}
\usepackage{fullpage}
\usepackage{microtype}
\usepackage{xspace}
\usepackage[svgnames]{xcolor}
\usepackage[sc]{mathpazo}
\usepackage{enumitem}
\setlist[enumerate]{itemsep=1pt,topsep=2pt}
\setlist[itemize]{itemsep=1pt,topsep=2pt}
%----------Header--------------------%
\def\course{CSCE 440/640 Quantum Algorithms}
\def\term{Texas A\&M U, Spring 2019}
\def\prof{Lecturer: Fang Song}
\newcommand{\handout}[5]{
   \renewcommand{\thepage}{\arabic{page}}
   \begin{center}
   \framebox{
      \vbox{
    \hbox to 5.78in { \hfill \large{\course} \hfill }
       \vspace{2mm}
       \hbox to 5.78in { {\Large \hfill \textbf{#5}  \hfill} }
       \vspace{2mm}
       \hbox to 5.78in { \term \hfill \emph{#2}}
       \hbox to 5.78in { {#3 \hfill \emph{#4}}}
      }
   }
   \end{center}
   \vspace*{4mm}
}
\newcommand{\hw}[4]{\handout{#1}{#2}{#3}{#4}{Homework #1}}
\newcommand{\ex}[4]{\handout{#1}{#2}{#3}{#4}{Exercise #1}}

%-----defs and commands-----%
\def\mG{[\textbf{G}]\xspace}
\def\veps{\varepsilon}
\def\tr{\mathrm{tr}}
\newcommand{\bit}{\{0,1\}}
\newcommand{\bra}[1]{\langle #1 \rvert}
\newcommand{\ket}[1]{\lvert #1 \rangle}
\newcommand{\kera}[1]{\ket{#1}\bra{#1}}
\newcommand{\negl}{\text{negl}}
\newcommand{\complex}{\mathbb{C}}
\newcommand{\integer}{\mathbb{Z}}
\newcommand{\srd}[2]{\textsf{SR}_{#1}^{#2}(X)}
\newcommand{\corr}[1]{{\color{blue}{#1}}}
%=======Main document==============%
\begin{document}

%----Specs: change accordingly-----%
\newif\ifstudent % comment out false
\studenttrue 
%\studentfalse

\def\issuedate{Jan. 28, 2019}
\def\duedate{Feb. 13, 2019} % 
\def\yourname{your name} % put your name here

%------------------------------%
\ifstudent
\hw{2}{\issuedate}{Student: \yourname}{Due: \duedate}%
\else
\hw{2}{\issuedate}{\prof}{Due: \duedate}%
\fi
\noindent \textbf{Instructions.} Only PDF format is accepted (type it
or scan clearly). Your solutions will be graded on \emph{correctness}
and \emph{clarity}. You should only submit work that you believe to be
correct; if you cannot solve a problem completely, you will get
significantly more partial credit if you clearly identify the gap(s)
in your solution. For this problem set, a random subset of problems
will be graded. Problems marked with ``\mG'' are required for graduate
students. Undergraduate students will get bonus points for solving
them.  \medskip

\noindent You may collaborate with others on this problem set. 
% and consult external sources.
However, you must \textbf{\emph{write up your own solutions}} and
\textbf{\emph{list your collaborators}} for each problem.

\begin{questions}
  
\question (Quantum states and gates)   
\begin{parts}
    \part[6] Let $X = \left(
    \begin{array}{lr}
      0 & 1\\
      1 & 0
    \end{array} \right)$ and $Z = \left(
    \begin{array}{lr}
      1 & 0\\
      0 & -1
    \end{array} \right)$.
  \begin{enumerate}[label=\roman*)]
    \item Suppose we have a qubit and we first apply $X$ and then
      $Z$. Is it equivalent to first applying $Z$ and then $X$?
    \item Suppose we have two qubits, and we apply $X$ to both and
      then $Z$ to both. Is it equivalent to applying $Z$ to both and
      then applying $X$ to both? Justify your answer. 
    \end{enumerate}
    \part[6] (SWAP gate) A SWAP gate takes two inputs $a$ and $b$ and
    outputs $b$ and $a$; i.e., it swaps the values of two input
    registers. Show how to build a SWAP gate using only CNOT gates.
    (Hint: you’ll need 3 of them.)
        \part[5] $\mG$ Show that every unitary one-qubit gate with real entries can
    be written as a rotation matrix, possibly preceded and followed by
    Z-gates. In other words, show that for every $2 \times 2$ real
    unitary U, there exist signs $s_1, s_2, s_3 \in \{1, -1\}$ and
    angle $\theta \in [0, 2\pi)$ such that

    \begin{equation*}
      U =  s_1   \left(\begin{array}{lr}
      1 & 0\\
      0 & s_2
              \end{array} \right) \left( \begin{array}{lr}
      \cos \theta & - \sin\theta\\
      \sin\theta & \cos\theta
                                         \end{array} \right) \left(    \begin{array}{lr}
                                                                1 & 0\\
                                                                0 & s_3
                                                                       \end{array} \right) \, . 
    \end{equation*}

  \end{parts}
  \question (Product states versus entangled states) In each of the
  following, either express the 2-qubit state as a tensor product of
  1-qubit states or prove that it cannot be expressed this way. 
  \begin{parts}
    \part[4] $\frac 1 2 \ket{00} + \frac 1 2 \ket{01} +\frac 1 2
    \ket{10} - \frac{1}{2} \ket{11}$
  \part[4]
    $\frac{3}{4}\ket{00} + \frac{\sqrt 3}{4}\ket{01}+\frac{\sqrt
      3}{4}\ket{10}+\frac{1}{4}\ket{11}$
  \end{parts}

  \question (Distinguishing states by local measurements) Suppose
  Alice and Bob are physically separated from each other, and are each
  given one of the qubits of some 2-qubit state. They are required to
  distinguish between State I and State II with only local
  measurements. Namely they can each perform a local (one-qubit)
  unitary operation and then a measurement (in the computational
  basis) of their own qubit. After their measurements, they can send
  only classical bits to each other. (This is usually referred to as
  LOCC: local operation and classical communication.)  In each case
  below, either give a perfect distinguishing procedure (that never
  errs) or explain why there is no perfect distinguishing procedure
  (i.e., that for any procedure the success probability must be less
  than 1).

  \begin{parts}
    \part[5] State I: $\frac{1}{\sqrt 2}(\ket{00} + \ket{11})$; State II: $\frac{1}{\sqrt 2}(\ket{01} + \ket{10})$ 
    \part[5] State I: $\frac{1}{\sqrt 2} (\ket{00} + \ket{11})$; State
    II: $\frac{1}{\sqrt 2} (\ket{00}-\ket{11})$
    \part[5] \mG State I: $\frac{1}{\sqrt 2}(\ket{00} + i \ket{11})$;
    State II: $\frac{1}{\sqrt 2}(\ket{00}-i\ket{11})$
  \end{parts}

\question (Linear algebra)
  \begin{parts}
  \part[15] (Tensor product)
    \begin{enumerate}[label=\roman*)]
    \item Show that
      $(A \otimes B)\cdot (C \otimes D) = (AC) \otimes (BD)$.
    \item Show that
      $(A\otimes B)^\dagger = A^\dagger \otimes B^\dagger$.
      \item If $A$ and $B$ are both invertible, show that so is
        $A\otimes B$.
      \item Show that
        $\tr(A\otimes B) = \tr(A)\cdot \tr(B)$. 
      \item Show that if $A$ and $B$ are unitary matrices, then so is
        $A \otimes B$.

    \end{enumerate}

  \part[6] Let $U=(v_1,\ldots,v_n)$ be a unitary matrix and each
    $v_i\in\complex^n$.
    \begin{enumerate}[label=\roman*)]
    \item Show that $\{v_1,\ldots,v_n\}$ form an orthonormal basis of
      $\complex^n$. 
    \item Show that the eigenvalues of any unitary $U$ are of the form
      $e^{i\theta}$ for some $\theta \in [0,2\pi)$.
    \end{enumerate}    
    \part[4] Show that for any $x\in \bit^n$,
    $H^{\otimes n} \ket{x} = {\frac{1}{\sqrt{2^n}}}\sum_{y\in \bit^n} (-1)^{x\cdot
      y}\ket{y}$. $x\cdot y : = \sum_{i = 1}^n x_iy_i$ is the dot
    product over $\integer_2^n$.

    \part[4] Let $x,y\in \bit^n$ and let $s = x\oplus y$. Show that
    \begin{equation*}
      H^{\otimes n} \frac{1}{\sqrt 2}(\ket{x} + \ket{y}) =
      \frac{1}{\sqrt {2^{n-1}}}\sum_{z: z\cdot s = 0} (-1)^{x\cdot z}
      \ket{z} \, .
    \end{equation*}

    % \part[5] Suppose that
    %   $\ket{v_1}, \ket{v_2}, \ldots \ket{v_k} \in \complex^k$ form
    %   an
    %   orthonormal basis. Show that $\sum_{i=1}^k \ket{v_i}\bra{v_i}$
    %   is
    %   the identity matrix.

  \part[8] For a vector $v = (v_0, \ldots, v_{k-1})\in \complex^k$,
    let $\|v\|:=\sqrt{\sum_{i=0}^{k-1} |v_i|^2}$, which is the usual
    Euclidean length of $v$. For any $k \times k$ matrix
    $M\in \complex^{k\times k}$, define its \emph{spectral norm}
    $\|M \|$ as $\|M\| = \max_{\ket{\psi}} \| M \ket{\psi}\|$, where
    the maximum is taken over quantum states (i.e., vectors
    $\ket{\psi}$ such that $\| \ket{\psi}\| = 1$).  Define the
    distance between two $k \times k$ unitary matrices $M_1$ and $M_2$
    as $\|M_1 - M_2\|$. Show that
    \begin{enumerate}[label=\roman*)]
    \item $\|A - B\| \leq \|A - C\| + \| C - B \|$, for any three
      $k \times k$ matrices A, B, and C. (Namely, this distance
      measure satisfies the \emph{triangle inequality}.)
    \item Show that, for any two $k \times k$ unitary matrices $U_1$
      and $U_2$, and any matrix $A$, $\|U_1AU_2\| = \|A\|$.
    \end{enumerate}
  \end{parts}

\question (Errors in randomized algorithms) Suppose you want to write
  a computer program $C$ to compute a Boolean function
  $f: \{0, 1\}^n \to \{0, 1\}$, mapping $n$ bits to 1 bit. If $C$ is a
  deterministic algorithm, then ``$C$ successfully computes $f$'' has
  a clear meaning that that $C(x) = f(x)$ for all inputs
  $x \in \bit^n$. But what if $C$ is a probabilistic algorithm?

  \begin{parts}
  \part[8] The best thing is if $C$ is a \emph{zero-error} algorithm with
    failure probability $p$. Namely
    \begin{itemize}
    \item on every input $x$, the output of $C(x)$ is either $f(x)$ or
      $\perp$ (denoting failure).
    \item on every input $x$ we have $\Pr[C(x) = \perp] \leq p$
      (NB. the probability is only over the internal randomness of
      $C$, not the random choice of $x$.).
    \end{itemize}

    \begin{enumerate}[label=\roman*)]
    \item If you have a zero-error algorithm $C$ for $f$ with failure
      probability $90\%$,
      show how to convert it to a zero-error algorithm $C'$ with
      failure probability at most $2^{-500}$. The ``slowdown'' should
      only be a factor of a few thousand.
    \item Alternatively, show how to convert $C$ to an algorithm $C''$
      for $f$ which: (i) always outputs the correct answer, meaning
      $C''(x) = f(x)$ for all $x$; (ii) has expected running time only
      a few powers of 2 worse than that of $C$. (Hint: look up the
      mean of a geometric random variable.)
    \end{enumerate}
  \part[5] The second best thing is if $C$ is a one-sided error algorithm
    for $f$, with failure probability $p$.  There are two kinds of
    such algorithms, ``no-false-positives'' and
    ``no-false-negatives''. For simplicity, let’s just consider ``no
    false-negatives'' (the other case is symmetric);
    \begin{itemize}
    \item on every input $x$, the output $C(x)$ is either $0$ or $1$;
    \item on every input $x$ such that $f(x) = 1$, the output $C(x)$
      is also 1;
    \item on every input $x$ such that $f(x) = 0$, we have
      $\Pr[C(x) = 1] \leq p$. 
    \end{itemize}

    Show how to convert a no-false-negatives algorithm $C$ for $f$
    with failure probability $90\%$ to another no-false-negatives
    algorithm $C'$ for $f$ with failure probability at most
    $2^{-500}$. The ``slowdown'' should only be a factor of a few
    thousand.

  \part[5] The third possibility (which is rare in practice) is if $C$ is
    a two-sided error algorithm for $f$, with failure probability
    $p$. Namely,
    \begin{itemize}
  \item on every input $x$, the output $C(x)$ is either 0 or 1.
  \item on every input $x$, we have $\Pr[C(x) \neq f(x)] \leq p$. 
  \end{itemize}

  If you have a two-sided error algorithm $C$ for $f$ with failure
  probability $40\%$, show how to convert it to a two-sided error
  algorithm $C'$ for $f$ with failure probability at most $2^{-500}$.
  The “slowdown” should only be a factor of a few dozen
  thousand. (Hint: look up the Chernoff bound.)
  \end{parts}

  \question (Simple search algorithms) In the context of this
  question, we are interested in exact solutions (with failure
  probability zero).

  \begin{parts}
    \part[6] (1-out-of-4 search) Consider a black-box function
    $f:\bit^2 \to \bit$ with the property that there is a unique
    $x\in \bit^{\corr{2}}$ such that $f(x) = 1$ and the goal is to
    determine $x$. How many classical queries are necessary to solve
    this problem? Design a quantum algorithm that finds $x$ using 1
    quantum query.
    \part[6] \mG (2-out-of-4 search) Given a black-box for a function
    $f:\bit^2 \to \bit$ with exactly two $x\in \bit^2$ such that
    $f(x) = 1$ and the goal is to determine both $x$'s. Prove that 3
    classical queries are necessary to solve this problem and that 2
    quantum queries are sufficient to solve this problem.
  \end{parts}

\question (Simulating classical circuits) Let $f: \bit^2 \to \bit^2$
  be a function such that $f(ab) = 0$ if $a=b=1$ and $f(ab) =1$
  otherwise.
  \begin{parts}
    \part[3] Design a circuit using your favorite gate set (e.g., AND,
    OR, NOT) to compute $f$.
    \part[3] Turn your circuit into a reversible circuit using Toffoli
    gate $T: a,b,c \mapsto a,b,a\wedge b \oplus c$ and other
    reversible gates. You may need to introduce ancilla bits
    \part[4] Turn your reversible circuit into a unitary quantum
    circuit that implements the unitary
    $U_f: \ket{x}\ket{y} \mapsto \ket{x}\ket{f(x)\oplus y}$. 
  \end{parts}

\question (Playing with quantum circuits)

  \begin{parts}

  \part (Exercise) Play around both the graphic composer and QASM
    editor
    \href{https://quantumexperience.ng.bluemix.net/qx/editor}{IBM Q
      experience} (or some other tools, e.g.,
    \href{https://www.quantum-quest.nl/quirky/}{Quirky} and
    \href{http://www.quantumplayground.net/#/playground/5080491044634624}{Quantum
      playground}). Test the teleportation protocol.
  \part[10] Determine the behavior of the following quantum circuit by
    implementing it (in graphic interface or programming it):
        \begin{figure}[ht]
          \centerline{ \Qcircuit @C=1em @R=0.75em {
              & \qw & \qw & \qw & \ctrl{2} & \qw&\qw&\qw&
              \ctrl{2}&\qw&\ctrl{1}& \qw &\ctrl{1}& \gate{T} &\qw \\
              & \qw & \ctrl{1} & \qw & \qw & \qw & \ctrl{1} &\qw &\qw
              & \gate{T^\dagger} & \targ & \gate{T^\dagger} & \targ
              &\gate{S} & \qw\\
              & \gate{H} & \targ & \gate{T^\dagger} & \targ & \gate{T}
              & \targ &\gate{T^\dagger} & \targ & \gate{T} & \gate{H}
              &\qw & \qw & \qw &\qw }}
\end{figure}
    You’ll want to precede this circuit by all 8 possible ways of
    doing or not doing NOT gates on the relevant 3 qubits, so as to
    see what this circuit does to each of the basic states
    $\ket{000},\ket{001}, \ldots, \ket{111}$. 
  \end{parts}
\question (Watch in your leisure time. No grades.)
  \begin{parts}
  \part \href{https://www.youtube.com/watch?v=dzKWfw68M5U}{Many Worlds
      Interpretation}.
  \part
    \href{https://www.youtube.com/watch?v=YCPnXtk8bMw&index=24&list=PLm3J0oaFux3aafQm568blS9blxtA_EWQv}{Fast
      Fourier Transform}.
  \part The enormity of the
    \href{https://www.youtube.com/watch?v=S9JGmA5_unY}{number
      $2^{256}$}.
  \end{parts}
\end{questions}
\end{document}


