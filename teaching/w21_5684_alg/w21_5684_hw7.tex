
% Instructions: you don't need to change anything in the macros, but
% feel free to define new commands as you wish. Starting from the main
% body, change the specs (e.g., your
% name). use \begin{solution} \end{solution} environment to write your
% solutions. Don't forget to list your collaborators.

\documentclass[12pt,answers]{exam}
%============Macros==================%
\usepackage{amsmath,amsfonts,amssymb,amsthm}
\usepackage[margin=1in]{geometry}
%--------------Cosmetic----------------%
\usepackage{mathtools}
\usepackage{hyperref}
\usepackage{fullpage}
\usepackage{microtype}
\usepackage{xspace}
\usepackage[svgnames]{xcolor}
\usepackage[sc]{mathpazo}
\usepackage{enumitem}
\setlist[enumerate]{itemsep=1pt,topsep=2pt}
\setlist[itemize]{itemsep=1pt,topsep=2pt}
%----------Header--------------------%
\def\course{CS 584/684 Algorithm Design and Analysis}
\def\term{Portland State U, Winter 2021}
\def\prof{Lecturer: Fang Song}
\newcommand{\handout}[5]{
   \renewcommand{\thepage}{\arabic{page}}
   \begin{center}
   \framebox{
      \vbox{
    \hbox to 5.78in { \hfill \large{\course} \hfill }
       \vspace{2mm}
       \hbox to 5.78in { {\Large \hfill \textbf{#5}  \hfill} }
       \vspace{2mm}
       \hbox to 5.78in { \term \hfill \emph{#2}}
       \hbox to 5.78in { {#3 \hfill \emph{#4}}}
      }
   }
   \end{center}
   \vspace*{4mm}
}
\newcommand{\hw}[4]{\handout{#1}{#2}{#3}{#4}{Homework #1}}

%-----defs and commands-----%
\def\veps{\varepsilon}
\newcommand{\bit}{\{0,1\}}
\newcommand{\negl}{\text{negl}}
\newcommand{\corr}[1]{{\color{blue}{#1}}}
\newcommand{\alg}[1]{\textsf{#1}}
%=======Main document==============%
\begin{document}

%----Specs: change accordingly-----%
\newif\ifstudent % comment out false
\studenttrue 
% \studentfalse

\def\texbp{5} % bonus for typing in latex
\def\hwnum{7} %
\def\issuedate{03/02/21} % 
\def\duedate{03/11/21} % 
\def\yourname{your name} % type your name here

%------------------------------%
\ifstudent
\hw{\hwnum}{\issuedate}{Student: \yourname}{Due: \duedate}%
\else
\hw{\hwnum}{\issuedate}{\prof}{Due: \duedate}%
\fi

\noindent \textbf{Instructions.} This problem set contains \numpages\
pages (including this cover page) and \numquestions\ questions. A
random subset of problems will be graded.

\begin{itemize}
\item Your solutions will be graded on \emph{correctness} and
  \emph{clarity}. You should only submit work that you believe to be
  correct, and you will get significantly more partial credit if you
  clearly identify the gap(s) in your solution. It is good practice to
  start any long solution with an informal (but accurate) summary that
  describes the main idea. You may opt for the ``I take 15\%'' option.

\item You need to submit a PDF file before the deadline. Either a
  clear scan of you handwriting or a typeset document is accepted. You
  will get $\texbp$ bonus points for typing in LaTeX (Download and use
  the accompany TeX file).

\item You may collaborate with others on this problem set. However,
  you must \textbf{{write up your own solutions}} and \textbf{{list
      your collaborators and any external sources}} for each
  problem. Be ready to explain your solutions orally to a course staff
  if asked.

\item For problems that require you to provide an algorithm, you must
  give a precise description of the algorithm, together with a proof
  of correctness and an analysis of its running time. You may use
  algorithms from class as subroutines. You may also use any facts
  that we proved in class or from the book.

\item \textbf{If you describe a Greedy algorithm, you will get no
    credit without a formal proof of correctness, even if your
    algorithm is correct.}

\item A proof of NP-completeness should include two parts: 1) the
    problem is in NP; and 2) the problem is NP-hard.

  \end{itemize}

  \paragraph{Exercises. Do not turn in.}
\begin{questions}
  \question A clique in an undirected graph $G =(V, E)$ is a subset
  $V' \subseteq V$ of vertices, each pair of which is connected by an
  edge in $E$. (In other words, a clique is a \emph{complete} subgraph
  of $G$. The size of a clique is the number of vertices it
  contains. The Clique problems asks to decide whether a clique of a
  given size $k$ exists in the graph. Show that Clique is NP-complete.
\end{questions}
\newpage 
\paragraph{Problems to turn in.}

\begin{questions}

\question (Demand) Suppose instead of capacities, we consider networks
  where each edge $u\to v$ has a non-negative \emph{demand}
  $d(u\to v)$. Now an $(s,t)$-flow $f$ is \emph{feasible} if and only
  if $f(u\to v)\ge d(u\to v)$ for every edge $u\to v$. (Feasible flow
  values can now be arbitrarily large.) A natural problem in this
  setting is to find a feasible $(s,t)$-flow of \emph{minimum} value.

  \begin{parts}
  \part[10] Describe an efficient algorithm to compute a feasible
    $(s,t)$-flow, given the graph, the demand function, and vertices
    $s$ and $t$ as input. (Hint: find a flow that is non-zero
    everywhere, and then scale it up to make it feasible.)
    
  \part[10] Suppose you have access to a subroutine \alg{MaxFlow} that
    computes \emph{maximum} flows in networks with edge
    capacities. Describe an efficient algorithm to compute a
    \emph{minimum} flow in a given network with edge demands; your
    algorithm should call \alg{MaxFlow} exactly once.

  \part[10] State and prove an analogue of the max-flow min-cut theorem
    for this setting. (Do minimum flows correspond to maximum cuts?)
  \end{parts}


  \newpage 

  \question (Integer linear programming) An \emph{integer
    linear-programming} problem is a linear-programming problem with
  the additional constraint that the variables $x$ must take on
  \emph{integral} values.  Specifically, given an integer $m\times n$
  matrix $A$, an integer vector $b$ of dimension $m$ and integer
  vector $c$ of dimension $n$, we want to maximize $c^T\cdot x$ under
  the constraints $Ax\le b$ and $x\ge 0$.


  \begin{parts}
    \part[7] Show that \emph{weak duality} (CLRS Lemma 29.8) holds for an
    integer linear program.
    \part[8] Show that \emph{strong duality} (CLRS Theorem 29.10) does
    not always hold for an integer linear program.
\newpage
    \bonuspart[10] Given a primal linear program in standard form, let us
    define $P$ to be the optimal objective value for the primal linear
    program, $D$ to be the optimal objective value for its dual, $IP$
    to be the optimal objective value for the integer version of the
    primal (that is, the primal with the added constraint that the
    variables take on integer values), and $ID$ to be the optimal
    objective value for the integer version of the dual. Assuming that
    both the primal integer program and the dual integer program are
    feasible and bounded, show that
    \[ IP\le P = D \le ID \, .\]


    \newpage
    
  \part[10] Consider further \emph{0-1 integer programming}, where one
  needs to decide if there exists an integer vector $x$ of dimension
  $n$ with elements in the set \bit such that $Ax \le b$. Prove that
  0-1 integer programming is NP-complete. 
  \end{parts}
  \newpage
  
  \question (Randomized and approximate algorithms)

  
  \begin{parts}
    
  \part[10] (Hat-check) Each of $n$ customers gives a hat to a hat-check
  person at a restaurant. The hat-check person gives the hats back to
  the customers in a uniformly random order. What is the expected
  number of customers who get back their own hat?
  \newpage

  \part[10] (3-Coloring) Suppose you are given a graph $G=(V,E)$, and we
  want to color each node with one of three colors, even if we aren't
  necessarily able to give different colors to every pair of adjacent
  nodes. We say an edge $(u,v)$ is satisfied if the colors assigned to
  $u$ and $v$ are different.

  Consider a coloring scheme that \emph{maximizes} the number of
  satisfied edges, and let $c^*$ denote this number. Give a poly-time
  algorithm that produces a coloring that satisfies at least
  $\frac 2 3 c^*$ edges. If you want to use an randomized algorithm,
  the \emph{expected} number of edges it satisfies should be at least
  $\frac 2 3 c^*$.
  
  \newpage

  \part[10] You have received a present containing $n$ pieces of candy
  with weights $W[1,\ldots,n]$ (in ounces). You want to load the candy
  into as many boxes as possible, so that each box contains \emph{at
    least} $L$ ounces of candy. Describe an efficient
  $2$-approximation algorithm for this problem. (Hint: First consider
  the case where every piece of candy weighs less than $L$ ounces.)
  \end{parts}


\end{questions}
\end{document}
