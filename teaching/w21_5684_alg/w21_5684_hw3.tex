% Instructions: you don't need to change anything in the macros, but
% feel free to define new commands as you wish. Starting from the main
% body, change the specs (e.g., your
% name). use \begin{solution} \end{solution} environment to write your
% solutions. Don't forget to list your collaborators.

\documentclass[12pt,answers]{exam}
%============Macros==================%
\usepackage{amsmath,amsfonts,amssymb,amsthm}
\usepackage[margin=1in]{geometry}
%--------------Cosmetic----------------%
\usepackage{mathtools}
\usepackage{hyperref}
\usepackage{fullpage}
\usepackage{microtype}
\usepackage{xspace}
\usepackage[svgnames]{xcolor}
\usepackage[sc]{mathpazo}
\usepackage{enumitem}
\setlist[enumerate]{itemsep=1pt,topsep=2pt}
\setlist[itemize]{itemsep=1pt,topsep=2pt}
%----------Header--------------------%
\def\course{CS 584/684 Algorithm Design and Analysis}
\def\term{Portland State U, Winter 2021}
\def\prof{Lecturer: Fang Song}
\newcommand{\handout}[5]{
   \renewcommand{\thepage}{\arabic{page}}
   \begin{center}
   \framebox{
      \vbox{
    \hbox to 5.78in { \hfill \large{\course} \hfill }
       \vspace{2mm}
       \hbox to 5.78in { {\Large \hfill \textbf{#5}  \hfill} }
       \vspace{2mm}
       \hbox to 5.78in { \term \hfill \emph{#2}}
       \hbox to 5.78in { {#3 \hfill \emph{#4}}}
      }
   }
   \end{center}
   \vspace*{4mm}
}
\newcommand{\hw}[4]{\handout{#1}{#2}{#3}{#4}{Homework #1}}

%-----defs and commands-----%
\def\veps{\varepsilon}
\newcommand{\bit}{\{0,1\}}
\newcommand{\negl}{\text{negl}}
\newcommand{\corr}[1]{{\color{blue}{#1}}}
\newcommand{\alg}[1]{\textsf{#1}}
%=======Main document==============%
\begin{document}

%----Specs: change accordingly-----%
\newif\ifstudent % comment out false
 \studenttrue 
% \studentfalse

\def\texbp{5} % bonus for typing in latex
\def\hwnum{3} %
\def\issuedate{01/19/21} % 
\def\duedate{01/26/21} % 
\def\yourname{your name} % type your name here

%------------------------------%
\ifstudent
\hw{\hwnum}{\issuedate}{Student: \yourname}{Due: \duedate}%
\else
\hw{\hwnum}{\issuedate}{\prof}{Due: \duedate}%
\fi

\noindent \textbf{Instructions.} This problem set contains \numpages\
pages (including this cover page) and \numquestions\ questions. A
random subset of problems will be graded.

\begin{itemize}
\item Your solutions will be graded on \emph{correctness} and
  \emph{clarity}. You should only submit work that you believe to be
  correct, and you will get significantly more partial credit if you
  clearly identify the gap(s) in your solution. It is good practice to
  start any long solution with an informal (but accurate) summary that
  describes the main idea. You may opt for the ``I take 15\%'' option.

\item You need to submit a PDF file before the deadline. Either a
  clear scan of you handwriting or a typeset document is accepted. You
  will get $\texbp$ bonus points for typing in LaTeX (Download and use
  the accompany TeX file).

\item You may collaborate with others on this problem set.  However,
  you must \textbf{{write up your own solutions}} and \textbf{{list
      your collaborators and any external sources}} for each
  problem. Be ready to explain your solutions orally to a course staff
  if asked.

\item For problems that require you to provide an algorithm, you must
  give a precise description of the algorithm, together with a proof
  of correctness and an analysis of its running time. You may use
  algorithms from class as subroutines. You may also use any facts
  that we proved in class or from the book.

\end{itemize}

\newpage

\begin{questions}

  \question (Sumerians' multiplication algorithm) The clay tablets
  discovered in Sumer led some scholars to conjecture that ancient
  Sumerians performed multiplication by reduction to \emph{squaring},
  using an identity like \[x\cdot y = (x^2 + y^2 - (x - y)^2)/2 \, .\]
  In this problem, we will investigate how to actually square large
  numbers.

  \begin{parts}
    \part[10] Describe a variant of Karatsuba’s algorithm that squares
    any $n$-digit number in $O(n^{\log 3})$ time, by reducing to
    squaring three $\lceil n/2 \rceil$-digit numbers. (Karatsuba
    actually did this in 1960.)

    \part[10] Describe a recursive algorithm that squares any $n$-digit
    number in $O(n^{\log_3{6}})$ time, by reducing to squaring six
    $\lceil n/3 \rceil$-digit numbers.

    \bonuspart[10] Describe a recursive algorithm that squares any
    $n$-digit number in $O(n^{log_3{5}})$ time, by reducing to
    squaring only five $(n/3 + O(1))$-digit numbers. [Hint: What is
    $(a + b + c)^2 + (a- b + c)^2$?]
    \end{parts}
    % \begin{solution}
    %   uncomment the environment and write your solution here
    % \end{solution}

  \newpage 
  
  \question[10] (Cycles) Give an algorithm to detect whether a given
  undirected graph contains a cycle. If the graph contains a cycle,
  then your algorithm should output one (not all cycles, just one of
  them). The running time of your algorithm should be $O(m + n)$ for a
  graph with $n$ nodes and $m$ edges.


  \newpage
\question An \emph{Euler tour} of a strongly connected, directed graph
  $G = (V, E)$ is a cycle that traverses each \emph{edge} of $G$
  exactly once, although it may visit a vertex more than once.

  \begin{parts}
  \part[10] Show that $G$ has an Euler tour if and only if
    $\text{in-degree}(v) = \text{out-degree}(v)$ for each vertex 
    $v \in V$.
  \part[10] Describe an $O(|E|)$-time algorithm to find an Euler tour
    of $G$ if one exists.
  \end{parts}

\end{questions}


\end{document}
