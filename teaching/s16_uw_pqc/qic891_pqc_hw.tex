%% HOW TO USE THIS TEMPLATE:

%% Ensure that you replace "YOUR NAME HERE" with your own name.  Type
%% your solution to each problem part within
%% the \begin{solution} \end{solution} environment immediately

% the "answers" option causes the solutions to be printed; make sure
% you change the variable \answers to 1. 
\def\answers{0}
\ifnum\answers=1
\documentclass[12pt,answers,addpoints]{exam}
\else
\documentclass[12pt,addpoints]{exam}
\fi

% header macros, ignore
\newcommand{\handout}[5]{
%   \renewcommand{\thepage}{#1-\arabic{page}}
   \renewcommand{\thepage}{\arabic{page}} 
  \noindent
   \begin{center}
   \framebox{
      \vbox{
    \hbox to 5.78in { \hfill {QIC 891 {Topics in Quantum Safe Cryptography}} \hfill }
       \vspace{2mm}
        \hbox to 5.78in { \hfill {Module 1: {\em Post-Quantum Cryptography}} \hfill  }
	\vspace{2mm}
	       % \hbox to 5.78in { {Lecturer: { Fang Song, University of Waterloo}} \hfill  }
	%\vspace{2mm}
       \hbox to 5.78in { {\Large \hfill #5  \hfill} }
       \vspace{2mm}
       \hbox to 5.78in { {\it #3 \hfill #4} }
      }
   }
   \end{center}
   \vspace*{4mm}
}

\newcommand{\hw}[4]{\handout{#1}{#2}{Student:#3}{Due date:
    #4}{Homework #1}}
\usepackage[utf8]{inputenc}
\usepackage[T1]{fontenc}
\usepackage{fixltx2e}
\usepackage{enumitem}
\usepackage{color,graphicx}
\usepackage{amssymb,amsmath,amsthm}
\usepackage{hyperref}
\usepackage{fourier} % charter, fourier, mathpazo, times
\usepackage[hmargin=1in,vmargin=1in]{geometry} %

\begin{document}
\hw{}{b}{{Your Name Here}}{June 2, 2016} % Put YOUR NAME HERE

{\sf You are encouraged to write up your solution using \LaTeX. A
  template .tex file is available at the course webpage.} 

\begin{questions}
  \question Review a few major directions for post-quantum
  cryptography: \emph{lattice}-based, \emph{code}-based,
  \emph{multivariate quadratic polynomial} (MQ)-based.
\begin{parts}
  \part[3] For each direction, give examples of proposed schemes for
  public-key \textbf{encryption} and \textbf{signature}.
  \part[3] For each direction, give examples of some hard problems and
  popular algorithms for them.
\end{parts}
  % \begin{itemize}
  % \item Lattice-based.
  % \item Code-based. 
  % \item MQ-based. 
  % \end{itemize}
\begin{solution}
... Answer here ... 
\end{solution}

\question Digital signature.
\begin{parts}
  \part[2] Hash-based signature scheme is based on a generic
  construction from one-way functions. Describe the two steps of the
  generic construction.
\begin{solution}
... Answer here ... 
\end{solution}
  \part[2] Given a 3-round identification scheme $(G,(P,V))$. Describe
  how to turn it into a signature scheme by the Fiat-Shamir
  transformation.
\begin{solution}
... Answer here ... 
\end{solution}

  \part[2] Suppose that we want to prove the security of an
  identification scheme and the Fiat-Shamir transformation against
  quantum attackers. Describe two (or more) challenges.
\begin{solution}
... Answer here ... 
\end{solution}

\end{parts}

\question Public-key encryption.  

\begin{parts}
  \part[4] Based on lattices, codes and MQ each, describe a candidate
  function $f$ and a corresponding trapdoor $f^{-1}$.
\begin{solution}
... Answer here ... 
\end{solution}
\part[4] Describe a method of constructing an IND-CPA encryption
scheme from a injective trapdoor one-way function $(f,f^-1)$ and a
random oracle $\mathcal{O}$. Give your justification why the
construction is secure and explain possible difficulties of proving
security against quantum attackers.
\begin{solution}
... Answer here ... 
\end{solution}

\end{parts}

\begin{solution}
... Answer here ... 
\end{solution}

\end{questions}

% \newpage
%  \bibliographystyle{alpha}
%  \small{\bibliography{../pqc.bib}}

\end{document}