% Instructions: you don't need to change anything in the macros, but
% feel free to define new commands as you wish. Starting from the main
% body, change the specs (e.g., your
% name). use \begin{solution} \end{solution} environment to write your
% solutions. Don't forget to list your collaborators.

\documentclass[12pt,answers]{exam}
%============Macros==================%
\usepackage{amsmath,amsfonts,amssymb,amsthm}
\usepackage[margin=1in]{geometry}
%--------------Cosmetic----------------%
\usepackage{mathtools}
\usepackage{hyperref}
\usepackage{fullpage}
\usepackage{microtype}
\usepackage{xspace}
\usepackage[svgnames]{xcolor}
\usepackage[sc]{mathpazo}
\usepackage{enumitem}
\setlist[enumerate]{itemsep=1pt,topsep=2pt}
\setlist[itemize]{itemsep=1pt,topsep=2pt}
%----------Header--------------------%
\def\course{CSCE629 Analysis of Algorithms}
\def\term{Texas A\&M U, Fall 2019}
\def\prof{Lecturer: Fang Song}
\newcommand{\handout}[5]{
   \renewcommand{\thepage}{\arabic{page}}
   \begin{center}
   \framebox{
      \vbox{
    \hbox to 5.78in { \hfill \large{\course} \hfill }
       \vspace{2mm}
       \hbox to 5.78in { {\Large \hfill \textbf{#5}  \hfill} }
       \vspace{2mm}
       \hbox to 5.78in { \term \hfill \emph{#2}}
       \hbox to 5.78in { {#3 \hfill \emph{#4}}}
      }
   }
   \end{center}
   \vspace*{4mm}
}
\newcommand{\hw}[4]{\handout{#1}{#2}{#3}{#4}{Homework #1}}

%-----defs and commands-----%
\def\mG{[\textbf{G}]\xspace}
\def\veps{\varepsilon}
\newcommand{\bit}{\{0,1\}}
\newcommand{\negl}{\text{negl}}
\newcommand{\corr}[1]{{\color{blue}{#1}}}
%=======Main document==============%
\begin{document}

%----Specs: change accordingly-----%
\newif\ifstudent % comment out false
\studenttrue 
%\studentfalse

\def\issuedate{08/30/19}
\def\duedate{09/06/19} % 
\def\yourname{your name} % put your name here

%------------------------------%
\ifstudent
\hw{1}{\issuedate}{Student: \yourname}{Due: \duedate}%
\else
\hw{1}{\issuedate}{\prof}{Due: \duedate}%
\fi

% This assignment contains \numquestions\ questions, \numpages\ pages
% for the total of \numpoints \ marks.
\noindent \textbf{Instructions.} Typeset your submission by \LaTeX,
and submit in PDF format. Your solutions will be graded on
\emph{correctness} and \emph{clarity}. You should only submit work
that you believe to be correct, and you will get significantly more
partial credit if you clearly identify the gap(s) in your
solution. You may opt for the ``I’ll take 15\%'' option (details in
Syllabus). A random subset of the problems will be graded. \medskip

\noindent You may collaborate with others on this problem set.
However, you must \textbf{{write up your own solutions}} and
\textbf{{list your collaborators and any external sources}} for each
problem.

\medskip

\begin{questions}

  \question (Piazza) \textbf{Attention: this problem is due Friday,
    August 30, 11:59pm CDT.}
  \begin{parts}
    \part[2] Enroll on Piazza
    \url{https://piazza.com/tamu/spring2019/csce440640/}.
    \part[3] Post a note on Piazza describing: 1) a few words about
    yourself; 2) your strengths in CS (e.g., programming, algorithm,
    ...); 3) what you hope to get out of this course; and 4) anything
    else you feel like sharing. See instructions on how to post a note
    \url{https://support.piazza.com/customer/en/portal/articles/1564004-post-a-note}.

    The purpose is to help me know you all, and also get you known to
    your fellow students. 
\end{parts}

\question[10] (Order of growth rate) Take the following list of functions and arrange them in
  ascending order of growth rate. That is, if function $g(n)$
  immediately follows function $f(n)$ in your list, then it should be
  the case that $f(n)$ is $O(g(n))$.
  \begin{itemize}
  \item $f_1(n) = n^{2.5}$
  \item $f_2(n) = \sqrt{2n}$
  \item $f_3(n)=n+10$
  \item $f_4(n) = 10n$
  \item $f_5(n) = 100^n$
  \item $f_6(n) = n^2 \log n$
  \end{itemize}

  % \begin{solution}
  %  uncomment the environment and write your solution here
  %\end{solution}

\question[10] (Order of growth rate) Take the following list of functions and arrange them in
  ascending order of growth rate.
  \begin{itemize}
  \item $g_1(n) = 2^{\sqrt{n}}$
  \item $g_2(n) = 2^n$
  \item $g_3(n) = n(\log n)^3$
  \item $g_4(n) = n^{4/3}$
  \item $g_5(n) = n^{\log n}$
  \item $g_6(n) = 2^{2^n}$
  \item $g_7(n) = 2^{n^2}$
  \item $g_8(n) = n!$
  \end{itemize}

\question[15] (Understanding big-$O$ notation) Assume you have functions $f$ and $g$ such that $f(n)$ is
  $O(g(n))$. For each of the following statements, decide whether you
  think it is true or false and give a proof or counterexample.
  \begin{parts}
  \part $\log_2f(n)$ is $O(\log_2 g(n))$.
  \part $2^{f(n)}$ is $O(2^{g(n)})$.
  \part $f(n)^2$ is $O(g(n)^2)$.
  \end{parts}
\question (Basic proof techniques) Read the chapter on \texttt{Proof
    by Induction} by Erickson
  (\url{http://jeffe.cs.illinois.edu/teaching/algorithms/notes/98-induction.pdf}),
  and do the following.

  \begin{parts}
  \part[10] Prove that every integer (positive, negative, or zero) can be
    written in the form $\sum k \pm 3^k$, where the exponents $k$ are
    distinct non-negative integers. For example:
    \begin{center}
      \begin{align*}
      42 &= 3^4 - 3^3 - 3^2 - 3^1 \\
      25 &= 3^3 - 3^1 + 3^0 \\
      17 &= 3^3 - 3^2 - 3^0
    \end{align*}
    \end{center}
  \part[10] A binary tree is \emph{full} if every node has either two
    children (an internal node) or no children (a leaf). Give two
    proofs (proof by induction and proof by contradiction) of the
    following statement: in any full binary tree, the number of leaves
    is exactly one more than the number of internal nodes.
  \end{parts}
\question (Reduction) Given a set $X$ of $n$ Boolean variables
  $x_1,\ldots,x_n$; each can take the value 0 or 1 (equivalently,
  ``false'' or ``true''). By a term over $X$, we mean one of the
  variables $x_i$ or its negation $\bar x_i$. Finally, a clause is
  simply a disjunction of distinct terms
  $t_1 \vee t_2 \vee \ \ldots \vee t_\ell $.  (Again, each
  $t_i \in \{x_1, x_2,\ldots,x_n, \overline{x_1},\ldots,
  \overline{x_n}\}$.) We say the clause has length l if it contains l
  terms.

  A truth assignment for $X$ is an assignment of the value 0 or 1 to
  each $x_i$; in other words, it is a function $v: X \to \bit$. The
  assignment $v$ implicitly gives $x_i$ the opposite truth value from
  $x_i$. An assignment satisfies a clause $C$ if it causes $C$ to
  evaluate to 1 under the rules of Boolean logic; this is equivalent
  to requiring that at least one of the terms in $C$ should receive
  the value 1. An assignment satisfies a collection of clauses
  $C_1,\ldots, C_k$ if it causes all of the $C_i$ to evaluate to 1; in
  other words, if it causes the conjunction
  $C_1 \wedge C_2 \wedge \ldots C_k$ to evaluate to 1. In this case,
  we will say that $v$ is a satisfying assignment with respect to
  $C_1,\ldots,C_k$; and that the set of clauses $C_1,\ldots,C_k$ is
  \emph{satisfiable}. For example, consider the three clauses
  \[(x_1 \vee \overline{x_2}), (\overline{x_1} \vee \overline{x_3}),
    (x_2 \vee \overline{x_3}) \, .\]

  A truth assignment $v$ that sets all variables to 1 is not a
  satisfying assignment, because it does not satisfy the second of
  these clauses; but the truth assignment $v'$ that sets all variables
  to 0 is a satisfying assignment.

  We can now state the \texttt{Satisfiability} Problem, also referred
  to as SAT:
  \begin{center}
    Given a set of clauses $C_1, \ldots, C_k$ over a set of variables
    $X = \{ x_1, \ldots, x_2\}$, does there exist a satisfying truth
    assignment?
    \end{center}

    Suppose we are given an oracle $\mathcal{O}$ which can solve the
    \texttt{Satisfiability} problem. Namely if we feed a set of
    clauses to $\mathcal{O}$, it will tell us YES if there is a
    satisfying assignment for all clauses or NO if there exists
    none. Show that you can actually find a satisfying truth
    assignment $v$ for $C_1,\ldots,C_k$ by asking questions to
    $\mathcal{O}$. Describe your procedure, and analyze how many
    questions you need to ask.
\end{questions}


\end{document}
