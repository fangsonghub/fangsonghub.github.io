% Instructions: you don't need to change anything in the macros, but
% feel free to define new commands as you wish. Starting from the main
% body, change the specs (e.g., your
% name). use \begin{solution} \end{solution} environment to write your
% solutions. Don't forget to list your collaborators.

\documentclass[12pt,addpoints,answers]{exam}
%============Macros==================%
\usepackage{amsmath,amsfonts,amssymb,amsthm}
\usepackage[margin=1in]{geometry}
%--------------Cosmetic----------------%
\usepackage{mathtools}
\usepackage{hyperref}
\usepackage{fullpage}
\usepackage{microtype}
\usepackage{xspace}
\usepackage[svgnames]{xcolor}
\usepackage[sc]{mathpazo}
\usepackage{enumitem}
\setlist[enumerate]{itemsep=1pt,topsep=2pt}
\setlist[itemize]{itemsep=1pt,topsep=2pt}
%----------Header--------------------%
\def\course{CSCE629 Analysis of Algorithms}
\def\term{Texas A\&M U, Fall 2019}
\def\prof{Lecturer: Fang Song}
\newcommand{\handout}[5]{
   \renewcommand{\thepage}{\arabic{page}}
   \begin{center}
   \framebox{
      \vbox{
    \hbox to 5.78in { \hfill \large{\course} \hfill }
       \vspace{2mm}
       \hbox to 5.78in { {\Large \hfill \textbf{#5}  \hfill} }
       \vspace{2mm}
       \hbox to 5.78in { \term \hfill \emph{#2}}
       \hbox to 5.78in { {#3 \hfill \emph{#4}}}
      }
   }
   \end{center}
   \vspace*{4mm}
}
\newcommand{\hw}[4]{\handout{#1}{#2}{#3}{#4}{Homework #1}}

%-----defs and commands-----%
%\def\mG{[\textbf{G}]\xspace}
\def\veps{\varepsilon}
\newcommand{\bit}{\{0,1\}}
\newcommand{\negl}{\text{negl}}
\newcommand{\corr}[1]{{\color{blue}{#1}}}
\newcommand{\alg}[1]{\textsf{#1}}
%=======Main document==============%
\begin{document}

%----Specs: change accordingly-----%
\newif\ifstudent % comment out false
\studenttrue 
% \studentfalse

\def\hwnum{3}
\def\issuedate{09/13/19}
\def\duedate{10am, 09/20/19} % 
\def\yourname{your name} % put your name here
%------------------------------%

\ifstudent
\hw{\hwnum}{\issuedate}{Student: \yourname}{Due: \duedate}%
\else
\hw{\hwnum}{\issuedate}{\prof}{Due: \duedate}%
\fi

\noindent \textbf{Instructions.}

\begin{itemize} 
\item Typeset your submission by \LaTeX, and submit in PDF
  format. Your solutions will be graded on \emph{correctness} and
  \emph{clarity}. You should only submit work that you believe to be
  correct, and you will get significantly more partial credit if you
  clearly identify the gap(s) in your solution. You may opt for the
  ``I’ll take 15\%'' option (details in Syllabus).
\item You may collaborate with others on this problem set.  However,
  you must \textbf{{write up your own solutions}} and \textbf{{list
      your collaborators and any external sources}} for each
  problem. Be ready to explain your solutions orally to a course staff
  if asked.
\item For problems that require you to provide an algorithm, you must
  give a precise description of the algorithm, together with a proof
  of correctness and an analysis of its running time. You may use
  algorithms from class as subroutines. You may also use any facts
  that we proved in class or from the book.
\end{itemize}

\noindent This assignment contains \numquestions\ questions,
\numpages\ pages for the total of \numpoints \ points and
\numbonuspoints\ bonus points. A random subset of the problems will be
graded. \medskip

\newpage

\begin{questions}

  \question (Akinator's trick) Play the game \alg{Akinator} online
  (\url{https://en.akinator.com/}), and answer the questions below.

  \begin{parts}
    \part[10] Given a \emph{sorted} array $A$ with distinct numbers,
    we want to find out an $i$ such that $A[i] = i$ if exists. Give an
    $O(\log n)$ algorithm.
    \part[10] Consider a sorted array with distinct numbers. It is
    then rotated $k$ ($k$ is unknown) positions to the right, and call
    the resulting array $A$. (Example: (8,9,2,3,5,7) is the sorted
    array (2,3,5,7,8,9) rotated to the right by 2 positions) Design as
    efficient an algorithm as you can to find out if $A$ contains a
    number $x$.

    Exercise (do not turn in). Can you think of some real-world
    problems that the techniques in your algorithms could be useful?
  \end{parts}

  \newpage
  \question (Counting inversions) Given a sequence of $n$
  \emph{distinct} numbers $a_1,\ldots,a_n$, we call $(a_i,a_j)$ an
  \emph{inversion} if $i< j$ but $a_i > a_j$. For instance, the
  sequence $(2,4,1,3,5)$ contains three inversions $(2,1)$, $(4,1)$
  and $(4,3)$. 

  \begin{parts}
    \part[15] Given an algorithm running in time $O(n\log n)$ that
    counts the number of inversions. (Hint: does Merge-sort help?) Can
    you also output all inversions?

    \bonuspart[10] Let's call a pair a \emph{significant inversion} if
    $i<j$ and $a_i > 2a_j$. Given an $O(n\log n)$ algorithm to count
    the number of significant inversions. 
  \end{parts}

  \newpage 

  
  \question[10] (Cycles) Give an algorithm to detect whether a given
  undirected graph contains a cycle. If the graph contains a cycle,
  then your algorithm should output one (not all cycles, just one of
  them). The running time of your algorithm should be $O(m + n)$ for a
  graph with $n$ nodes and $m$ edges.

  \newpage
  
  \question[15] (Shortest cycles containing a given edge) Give an
  algorithm that takes as input an undirected graph $G = (V, E)$ and
  an edge $e \in E$, and outputs a shortest cycle that contains $e$
  (if no cycle containing $e$ exists, the algorithm should output
  ``none'').

  (Note: Give as efficient an algorithm as you can.)

  
\end{questions}


\end{document}
