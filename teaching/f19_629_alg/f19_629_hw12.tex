
% Instructions: you don't need to change anything in the macros, but
% feel free to define new commands as you wish. Starting from the main
% body, change the specs (e.g., your
% name). use \begin{solution} \end{solution} environment to write your
% solutions. Don't forget to list your collaborators.

\documentclass[12pt,addpoints,answers]{exam}
%============Macros==================%
\usepackage{amsmath,amsfonts,amssymb,amsthm}
\usepackage[margin=1in]{geometry}
%--------------Cosmetic----------------%
\usepackage{mathtools}
\usepackage{hyperref}
\usepackage{fullpage}
\usepackage{microtype}
\usepackage{xspace}
\usepackage[svgnames]{xcolor}
\usepackage[sc]{mathpazo}
\usepackage{enumitem}
\setlist[enumerate]{itemsep=1pt,topsep=2pt}
\setlist[itemize]{itemsep=1pt,topsep=2pt}
%----------Header--------------------%
\def\course{CSCE629 Analysis of Algorithms}
\def\term{Texas A\&M U, Fall 2019}
\def\prof{Lecturer: Fang Song}
\newcommand{\handout}[5]{
   \renewcommand{\thepage}{\arabic{page}}
   \begin{center}
   \framebox{
      \vbox{
    \hbox to 5.78in { \hfill \large{\course} \hfill }
       \vspace{2mm}
       \hbox to 5.78in { {\Large \hfill \textbf{#5}  \hfill} }
       \vspace{2mm}
       \hbox to 5.78in { \term \hfill \emph{#2}}
       \hbox to 5.78in { {#3 \hfill \emph{#4}}}
      }
   }
   \end{center}
   \vspace*{4mm}
}
\newcommand{\hw}[4]{\handout{#1}{#2}{#3}{#4}{Homework #1}}

%-----defs and commands-----%
%\def\mG{[\textbf{G}]\xspace}
\def\veps{\varepsilon}
\newcommand{\bit}{\{0,1\}}
\newcommand{\negl}{\text{negl}}
\newcommand{\corr}[1]{{\color{blue}{#1}}}
\newcommand{\alg}[1]{\textsf{#1}}
\newcommand{\class}[1]{\text{#1}}
\newcommand{\NP}{\class{NP}}
%=======Main document==============%
\begin{document}

%----Specs: change accordingly-----%
\newif\ifstudent % comment out false
 \studenttrue 
% \studentfalse

\def\hwnum{12}
\def\issuedate{11/12/19}
\def\duedate{10am, 12/02/19} % 
\def\yourname{your name} % put your name here
%------------------------------%

\ifstudent
\hw{\hwnum}{\issuedate}{Student: \yourname}{Due: \duedate}%
\else
\hw{\hwnum}{\issuedate}{\prof}{Due: \duedate}%
\fi

\noindent \textbf{Instructions.}

\begin{itemize} 
\item Typeset your submission by \LaTeX, and submit in PDF
  format. Your solutions will be graded on \emph{correctness} and
  \emph{clarity}. You should only submit work that you believe to be
  correct, and you will get significantly more partial credit if you
  clearly identify the gap(s) in your solution. You may opt for the
  ``I’ll take 15\%'' option.
\item You may collaborate with others on this problem set.  However,
  you must \textbf{{write up your own solutions}} and \textbf{{list
      your collaborators and any external sources}} for each
  problem. Be ready to explain your solutions orally to a course staff
  if asked.
\item For problems that require you to provide an algorithm, you must
  give a precise description of the algorithm, together with a proof
  of correctness and an analysis of its running time. You may use
  algorithms from class as subroutines. You may also use any facts
  that we proved in class or from the book.
\item \textbf{If you describe a Greedy algorithm, you will get no
    credit without a formal proof of correctness, even if your
    algorithm is correct.}
\end{itemize}

\noindent This assignment contains \numquestions\ questions,
\numpages\ pages for the total of \numpoints \ points and
\numbonuspoints\ bonus points. A random subset of the problems will be
graded. \medskip

\paragraph{Exercises. Do not turn in.}
\begin{questions}
  \question (Hat-check) Each of $n$ customers gives a hat to a
  hat-check person at a restaurant. The hat-check person gives the
  hats back to the customers in a uniformly random order. What is the
  expected number of customers who get back their own hat?  \question
  (Streaks) Suppose you flip a fair coin $n$ times. What is the
  longest streak of consecutive heads that you expect to see?
\end{questions}
\newpage 
\paragraph{Problems to turn in.}

\begin{questions}
\question (Happy holidays) This time of year is here, you known,
  holidays, and everyone is facing various (joyful) challenges.
  \begin{parts}
  \part[10] For Kevin (character in movie ``Home
    Alone''), he is handed $n$ pieces of candy with weights
    $W[1,\ldots,n]$ (in ounces) that he needs to load into boxes. The
    goal is to load the candy into as many boxes as possible, so that
    each box contains at least $L$ ounces of candy. Help Kevin solve
    this by describing an efficient $2$-approximation algorithm for
    this problem. Prove that the approximation ratio of your algorithm
    is 2. [Hint: First consider the case where every piece of candy
    weighs less than $L$ ounces.]

  \part[10] For Juliet, this means that she needs to fulfill Romeo's
    wishlist - $n$ Nintendo\textsuperscript{\textregistered} Switch
    games. She happened to notice that in a personal care and beauty
    store \textsf{AROHPES}, her favorite lip balm includes a 99\%
    discount code for one of the $n$ games Romeo wants. However, the
    code is hidden at the bottom of the package, which Juliet cannot
    tell before purchasing and opening it, and the code comes in
    random so that it is equally likely to be good for any of the $n$
    games. She intends to purchase sufficient lip balms, which she
    likes and needs anyways, so she gets discount codes for all $n$
    games. How many lip balms does she need to buy \emph{in
      expectation} before getting a code for each game?

    \bonuspart[3] Continuing from Part (b), it's 7 days till
    Thanksgiving now. Due to popularity, \textsf{AROHPES} limits the
    number of this particular lip balms one can purchase in a day to
    $6$. She convinced Romeo to reduce his demand to 10 games. Can
    Juliet succeed in collecting all discount codes by Thanksgiving?
    [Hint: look up Markov's inequality]
    
  \end{parts}

  \newpage

\question (3-Coloring) Suppose you are given a graph $G=(V,E)$, and we
  want to color each node with one of three colors, even if we aren't
  necessarily able to give different colors to every pair of adjacent
  nodes. We say an edge $(u,v)$ is satisfied if the colors assigned to
  $u$ and $v$ are different.

  Consider a coloring scheme that maximizes the number of satisfied
  edges, and let $c^*$ denote this number. Give a poly-time algorithm
  that produces a coloring that satisfies at least $\frac 2 3 c^*$
  edges. If you want to use an randomized algorithm, the
  \emph{expected} number of edges it satisfies should be at least
  $\frac 2 3 c^*$. 
  
  \newpage
    
  \question (Errors in randomized algorithms) Suppose you want to
    write a computer program $C$ to compute a Boolean function
    $f: \{0, 1\}^n \to \{0, 1\}$, mapping $n$ bits to 1 bit. If $C$ is
    a deterministic algorithm, then ``$C$ successfully computes $f$''
    has a clear meaning that that $C(x) = f(x)$ for all inputs
    $x \in \bit^n$. But what if $C$ is a randomized algorithm?

  \begin{parts}
  \part[8] The best thing is if $C$ is a \emph{zero-error} algorithm with
    failure probability $p$. Namely
    \begin{itemize}
    \item on every input $x$, the output of $C(x)$ is either $f(x)$ or
      $\perp$ (denoting failure).
    \item on every input $x$ we have $\Pr[C(x) = \perp] \leq p$
      (NB. the probability is only over the internal randomness of
      $C$, not the random choice of $x$.).
    \end{itemize}

    \begin{enumerate}[label=\roman*)]
    \item If you have a zero-error algorithm $C$ for $f$ with failure
      probability $90\%$,
      show how to convert it to a zero-error algorithm $C'$ with
      failure probability at most $2^{-500}$. The ``slowdown'' should
      only be a factor of a few thousand.
    \item Alternatively, show how to convert $C$ to an algorithm $C''$
      for $f$ which: (i) always outputs the correct answer, meaning
      $C''(x) = f(x)$ for all $x$; (ii) has expected running time only
      a few powers of 2 worse than that of $C$. (Hint: look up the
      mean of a geometric random variable.)
    \end{enumerate}
  \part[5] The second best thing is if $C$ is a one-sided error algorithm
    for $f$, with failure probability $p$.  There are two kinds of
    such algorithms, ``no-false-positives'' and
    ``no-false-negatives''. For simplicity, let’s just consider ``no
    false-negatives'' (the other case is symmetric);
    \begin{itemize}
    \item on every input $x$, the output $C(x)$ is either $0$ or $1$;
    \item on every input $x$ such that $f(x) = 1$, the output $C(x)$
      is also 1;
    \item on every input $x$ such that $f(x) = 0$, we have
      $\Pr[C(x) = 1] \leq p$. 
    \end{itemize}

    Show how to convert a no-false-negatives algorithm $C$ for $f$
    with failure probability $90\%$ to another no-false-negatives
    algorithm $C'$ for $f$ with failure probability at most
    $2^{-500}$. The ``slowdown'' should only be a factor of a few
    thousand.

  \part[5] The third possibility (which is rare in practice) is if $C$ is
    a two-sided error algorithm for $f$, with failure probability
    $p$. Namely,
    \begin{itemize}
  \item on every input $x$, the output $C(x)$ is either 0 or 1.
  \item on every input $x$, we have $\Pr[C(x) \neq f(x)] \leq p$. 
  \end{itemize}

  If you have a two-sided error algorithm $C$ for $f$ with failure
  probability $40\%$, show how to convert it to a two-sided error
  algorithm $C'$ for $f$ with failure probability at most $2^{-500}$.
  The “slowdown” should only be a factor of a few dozen
  thousand. (Hint: look up the Chernoff bound.)
  \end{parts}


\end{questions}
\end{document}

