
% Instructions: you don't need to change anything in the macros, but
% feel free to define new commands as you wish. Starting from the main
% body, change the specs (e.g., your
% name). use \begin{solution} \end{solution} environment to write your
% solutions. Don't forget to list your collaborators.

\documentclass[12pt,addpoints,answers]{exam}
%============Macros==================%
\usepackage{amsmath,amsfonts,amssymb,amsthm}
\usepackage[margin=1in]{geometry}
%--------------Cosmetic----------------%
\usepackage{mathtools}
\usepackage{hyperref}
\usepackage{fullpage}
\usepackage{microtype}
\usepackage{xspace}
\usepackage[svgnames]{xcolor}
\usepackage[sc]{mathpazo}
\usepackage{enumitem}
\setlist[enumerate]{itemsep=1pt,topsep=2pt}
\setlist[itemize]{itemsep=1pt,topsep=2pt}
%----------Header--------------------%
\def\course{CSCE629 Analysis of Algorithms}
\def\term{Texas A\&M U, Fall 2019}
\def\prof{Lecturer: Fang Song}
\newcommand{\handout}[5]{
   \renewcommand{\thepage}{\arabic{page}}
   \begin{center}
   \framebox{
      \vbox{
    \hbox to 5.78in { \hfill \large{\course} \hfill }
       \vspace{2mm}
       \hbox to 5.78in { {\Large \hfill \textbf{#5}  \hfill} }
       \vspace{2mm}
       \hbox to 5.78in { \term \hfill \emph{#2}}
       \hbox to 5.78in { {#3 \hfill \emph{#4}}}
      }
   }
   \end{center}
   \vspace*{4mm}
}
\newcommand{\hw}[4]{\handout{#1}{#2}{#3}{#4}{Homework #1}}

%-----defs and commands-----%
%\def\mG{[\textbf{G}]\xspace}
\def\veps{\varepsilon}
\newcommand{\bit}{\{0,1\}}
\newcommand{\negl}{\text{negl}}
\newcommand{\corr}[1]{{\color{blue}{#1}}}
\newcommand{\alg}[1]{\textsf{#1}}
%=======Main document==============%
\begin{document}

%----Specs: change accordingly-----%
\newif\ifstudent % comment out false
\studenttrue 
% \studentfalse

\def\hwnum{6}
\def\issuedate{10/07/19}
\def\duedate{10am, 10/18/19} % 
\def\yourname{your name} % put your name here
%------------------------------%

\ifstudent
\hw{\hwnum}{\issuedate}{Student: \yourname}{Due: \duedate}%
\else
\hw{\hwnum}{\issuedate}{\prof}{Due: \duedate}%
\fi

\noindent \textbf{Instructions.}

\begin{itemize} 
\item Typeset your submission by \LaTeX, and submit in PDF
  format. Your solutions will be graded on \emph{correctness} and
  \emph{clarity}. You should only submit work that you believe to be
  correct, and you will get significantly more partial credit if you
  clearly identify the gap(s) in your solution. You may opt for the
  ``I’ll take 15\%'' option.
\item You may collaborate with others on this problem set.  However,
  you must \textbf{{write up your own solutions}} and \textbf{{list
      your collaborators and any external sources}} for each
  problem. Be ready to explain your solutions orally to a course staff
  if asked.
\item For problems that require you to provide an algorithm, you must
  give a precise description of the algorithm, together with a proof
  of correctness and an analysis of its running time. You may use
  algorithms from class as subroutines. You may also use any facts
  that we proved in class or from the book.
\item \textbf{If you describe a Greedy algorithm, you will get no
    credit without a formal proof of correctness, even if your
    algorithm is correct.}
\end{itemize}

\noindent This assignment contains \numquestions\ questions,
\numpages\ pages for the total of \numpoints \ points and
\numbonuspoints\ bonus points. A random subset of the problems will be
graded. \medskip

\newpage
\begin{questions}

  \question[15] (Print neatly) Consider the problem of neatly printing
  a paragraph with a mono-spaced font (all characters having the same
  width) on a printer. The input text is a sequence of $n$ words of
  length $\ell_1,\ell_2,\ldots,\ell_n$ measured in characters. We want
  to print this paragraph neatly on a number of lines that hold a
  maximum of $M$ characters each. Our criterion of ``neatness'' is as
  follows.

  If a given line contains words $i$ through $j$, where $i\le j$, and
  we leave exactly one space between words, the number of extra space
  characters at the end of line is $M - j +i + \sum_{k=i}^j \ell_k$,
  which must be non-negative to fit the words on the line. We wish to
  \emph{minimize} the sum, over all lines except the last. Describe
  and analyze an algorithm (both time and space) to print a paragraph
  of $n$ words neatly. 

  \newpage
  \question (Shortest path with bounded negative edges) Suppose we
  are given a directed graph $G$ with weighted edges and two vertices
  $s$ and $t$.
\begin{parts}
    \part[10] Describe and analyze an algorithm to find the shortest path
    from $s$ to $t$ when exactly one edge in $G$ has negative weight.

    \part[15] Describe and analyze an algorithm to find the shortest
    path from $s$ to $t$ when exactly $k$ edges in $G$ have negative
    weights. How does the running time of your algorithm depend on
    $k$?
\end{parts}
\newpage
\question (Arbitrage) Arbitrage is the use of discrepancies in
currency exchange rates to transform one unit of a currency into more
than one unit of the same currency. For example (exchange rates not
up to date), suppose 1 US dollar buys 71 Indian rupees, 1 Indian rupee
buys 1.6 Japanese yen, and 1 Japanese yen buys 0.0093 US dollars. Then
by converting currencies, a trader can start with 1 US dollar and buy
$71\times 1.6 \times 0.0093 = 1.0565$ US dollars, thus making a profit
of 5.65 percent.

Suppose that you are given $n$ currencies $c_1,c_2,\ldots,c_n$ and an
$n\times n$ table $R$ of exchange rates, such that one unit of
currency $c_i$ buys $R[i,j]$ units of currency $j$.

\begin{parts}
  \part[10] Describe and analyze an algorithm to determine whether or not
  there exists a sequence of currencies
  $\langle c_{i_1},\ldots, c_{i_k}\rangle $ such that
  \[ R[i_1,i_2] \cdot R[i_2,i_3] \cdots R[i_{k-1},i_k]\cdot R[i_k,
    i_1] > 1 \, .\]
  \part[10] Describe and analyze an algorithm to print out such a
  sequence if one exists.   
\end{parts}

\end{questions}
\end{document}
