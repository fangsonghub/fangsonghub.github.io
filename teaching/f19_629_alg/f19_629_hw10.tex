% Instructions: you don't need to change anything in the macros, but
% feel free to define new commands as you wish. Starting from the main
% body, change the specs (e.g., your
% name). use \begin{solution} \end{solution} environment to write your
% solutions. Don't forget to list your collaborators.

\documentclass[12pt,addpoints,answers]{exam}
%============Macros==================%
\usepackage{amsmath,amsfonts,amssymb,amsthm}
\usepackage[margin=1in]{geometry}
%--------------Cosmetic----------------%
\usepackage{mathtools}
\usepackage{hyperref}
\usepackage{fullpage}
\usepackage{microtype}
\usepackage{xspace}
\usepackage[svgnames]{xcolor}
\usepackage[sc]{mathpazo}
\usepackage{enumitem}
\setlist[enumerate]{itemsep=1pt,topsep=2pt}
\setlist[itemize]{itemsep=1pt,topsep=2pt}
%----------Header--------------------%
\def\course{CSCE629 Analysis of Algorithms}
\def\term{Texas A\&M U, Fall 2019}
\def\prof{Lecturer: Fang Song}
\newcommand{\handout}[5]{
   \renewcommand{\thepage}{\arabic{page}}
   \begin{center}
   \framebox{
      \vbox{
    \hbox to 5.78in { \hfill \large{\course} \hfill }
       \vspace{2mm}
       \hbox to 5.78in { {\Large \hfill \textbf{#5}  \hfill} }
       \vspace{2mm}
       \hbox to 5.78in { \term \hfill \emph{#2}}
       \hbox to 5.78in { {#3 \hfill \emph{#4}}}
      }
   }
   \end{center}
   \vspace*{4mm}
}
\newcommand{\hw}[4]{\handout{#1}{#2}{#3}{#4}{Homework #1}}

%-----defs and commands-----%
%\def\mG{[\textbf{G}]\xspace}
\def\veps{\varepsilon}
\newcommand{\bit}{\{0,1\}}
\newcommand{\negl}{\text{negl}}
\newcommand{\corr}[1]{{\color{blue}{#1}}}
\newcommand{\alg}[1]{\textsf{#1}}
\newcommand{\class}[1]{\text{#1}}
\newcommand{\NP}{\class{NP}}
%=======Main document==============%
\begin{document}

%----Specs: change accordingly-----%
\newif\ifstudent % comment out false
 \studenttrue 
% \studentfalse

\def\hwnum{10}
\def\issuedate{11/08/19}
\def\duedate{10am, 11/15/19} % 
\def\yourname{your name} % put your name here
%------------------------------%

\ifstudent
\hw{\hwnum}{\issuedate}{Student: \yourname}{Due: \duedate}%
\else
\hw{\hwnum}{\issuedate}{\prof}{Due: \duedate}%
\fi

\noindent \textbf{Instructions.}

\begin{itemize} 
\item Typeset your submission by \LaTeX, and submit in PDF
  format. Your solutions will be graded on \emph{correctness} and
  \emph{clarity}. You should only submit work that you believe to be
  correct, and you will get significantly more partial credit if you
  clearly identify the gap(s) in your solution. You may opt for the
  ``I’ll take 15\%'' option.
\item You may collaborate with others on this problem set.  However,
  you must \textbf{{write up your own solutions}} and \textbf{{list
      your collaborators and any external sources}} for each
  problem. Be ready to explain your solutions orally to a course staff
  if asked.
\item For problems that require you to provide an algorithm, you must
  give a precise description of the algorithm, together with a proof
  of correctness and an analysis of its running time. You may use
  algorithms from class as subroutines. You may also use any facts
  that we proved in class or from the book.
\item {If you describe a Greedy algorithm, you will get no
    credit without a formal proof of correctness, even if your
    algorithm is correct.}
\end{itemize}

\noindent This assignment contains \numquestions\ questions,
\numpages\ pages for the total of \numpoints \ points and
\numbonuspoints\ bonus points. A random subset of the problems will be
graded. \medskip

% \paragraph{Exercises. Do not turn in.}
\newpage 

\begin{questions}

  \question (Integer linear programming) An \emph{integer
    linear-programming} problem is a linear-programming problem with
  the additional constraint that the variables $x$ must take on
  \emph{integral} values.

  \begin{parts}
    \part[7] Show that \emph{weak duality} (CLRS Lemma 29.8) holds for an
    integer linear program.

    \part[8] Show that \emph{strong duality} (CLRS Theorem 29.10) does
    not always hold for an integer linear program.

    \part[10] Given a primal linear program in standard form, let us
    define $P$ to be the optimal objective value for the primal linear
    program, $D$ to be the optimal objective value for its dual, $IP$
    to be the optimal objective value for the integer version of the
    primal (that is, the primal with the added constraint that the
    variables take on integer values), and $ID$ to be the optimal
    objective value for the integer version of the dual. Assuming that
    both the primal integer program and the dual integer program are
    feasible and bounded, show that
    \[ IP\le P = D \le ID \, .\]
  \end{parts}

  \newpage
  \question ($\NP$ under regular operations) Given any sets
  $X\subseteq \bit^*$ and $Y \subseteq \bit^*$, we can define new sets
  under regular operations. For example, the \emph{union} is
  $X \cup Y = \{x: x\in X \text{ or }x\in Y\}$; and the
  \emph{concatenation} is
  $X \circ Y = \{xy: x\in X \text{ and } y \in Y \}$. We consider
  applying these operations on problems in NP.
  \begin{parts}
  \part[7] Prove or disprove that $\NP$ is closed under union. Namely
    for any $X\in \NP$ and $Y\in \NP$, does $X\cup Y \in \NP$ always
    hold?

    \part[8] Prove or disprove that $\NP$ is closed under
    concatenation.
  \part (Exercise. Do not turn in.) Prove or disprove that $\NP$ is
    closed under complement.
  \end{parts}

  \newpage
 
  \question[15] (2-SAT) Consider 2-SAT, in which the input is a formula
  with at most 2 literals per clause. Show that 2-SAT is in P. (Hint:
  you may want to consider an algorithm for testing if a graph is
  bipartite).
\end{questions}
\end{document}
