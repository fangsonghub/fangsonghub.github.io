
% Instructions: you don't need to change anything in the macros, but
% feel free to define new commands as you wish. Starting from the main
% body, change the specs (e.g., your
% name). use \begin{solution} \end{solution} environment to write your
% solutions. Don't forget to list your collaborators.

\documentclass[12pt,addpoints,answers]{exam}
%============Macros==================%
\usepackage{amsmath,amsfonts,amssymb,amsthm}
\usepackage[margin=1in]{geometry}
%--------------Cosmetic----------------%
\usepackage{mathtools}
\usepackage{hyperref}
\usepackage{fullpage}
\usepackage{microtype}
\usepackage{xspace}
\usepackage[svgnames]{xcolor}
\usepackage[sc]{mathpazo}
\usepackage{enumitem}
\setlist[enumerate]{itemsep=1pt,topsep=2pt}
\setlist[itemize]{itemsep=1pt,topsep=2pt}
%----------Header--------------------%
\def\course{CSCE629 Analysis of Algorithms}
\def\term{Texas A\&M U, Fall 2019}
\def\prof{Lecturer: Fang Song}
\newcommand{\handout}[5]{
   \renewcommand{\thepage}{\arabic{page}}
   \begin{center}
   \framebox{
      \vbox{
    \hbox to 5.78in { \hfill \large{\course} \hfill }
       \vspace{2mm}
       \hbox to 5.78in { {\Large \hfill \textbf{#5}  \hfill} }
       \vspace{2mm}
       \hbox to 5.78in { \term \hfill \emph{#2}}
       \hbox to 5.78in { {#3 \hfill \emph{#4}}}
      }
   }
   \end{center}
   \vspace*{4mm}
}
\newcommand{\hw}[4]{\handout{#1}{#2}{#3}{#4}{Homework #1}}

%-----defs and commands-----%
%\def\mG{[\textbf{G}]\xspace}
\def\veps{\varepsilon}
\newcommand{\bit}{\{0,1\}}
\newcommand{\negl}{\text{negl}}
\newcommand{\corr}[1]{{\color{blue}{#1}}}
\newcommand{\alg}[1]{\textsf{#1}}
%=======Main document==============%
\begin{document}

%----Specs: change accordingly-----%
\newif\ifstudent % comment out false
\studenttrue 
%\studentfalse

\def\hwnum{5}
\def\issuedate{09/27/19}
\def\duedate{10am, 10/04/19} % 
\def\yourname{your name} % put your name here
%------------------------------%

\ifstudent
\hw{\hwnum}{\issuedate}{Student: \yourname}{Due: \duedate}%
\else
\hw{\hwnum}{\issuedate}{\prof}{Due: \duedate}%
\fi

\noindent \textbf{Instructions.}

\begin{itemize} 
\item Typeset your submission by \LaTeX, and submit in PDF
  format. Your solutions will be graded on \emph{correctness} and
  \emph{clarity}. You should only submit work that you believe to be
  correct, and you will get significantly more partial credit if you
  clearly identify the gap(s) in your solution. You may opt for the
  ``I’ll take 15\%'' option (details in Syllabus).
\item You may collaborate with others on this problem set.  However,
  you must \textbf{{write up your own solutions}} and \textbf{{list
      your collaborators and any external sources}} for each
  problem. Be ready to explain your solutions orally to a course staff
  if asked.
\item For problems that require you to provide an algorithm, you must
  give a precise description of the algorithm, together with a proof
  of correctness and an analysis of its running time. You may use
  algorithms from class as subroutines. You may also use any facts
  that we proved in class or from the book.
\end{itemize}

\noindent This assignment contains \numquestions\ questions,
\numpages\ pages for the total of \numpoints \ points and
\numbonuspoints\ bonus points. A random subset of the problems will be
graded. \medskip

\newpage
\begin{questions}
  \question[20] (Longest forward-backward contiguous substring) Describe
  and analyze an efficient algorithm to find the length of the longest
  \emph{contiguous} substring that appears both \emph{forward} and
  \emph{backward} in an input string $T[1,\ldots,n]$. The forward and
  backward substrings must NOT overlap.  Here are several examples.
  \begin{itemize}
  \item Given the input string \texttt{ALGORITHM}, your algorithm
    should return 0.

  \item Given the input string
    \texttt{\underline{R}ECU\underline{R}SION}, your algorithm should
    return 1, for the substring \texttt{R}.
  \item Given the input string
    \texttt{R\underline{EDI}V\underline{IDE}}, your algorithm should
    return 3, for the substring \texttt{EDI}. (The forward and backward
    substrings must not overlap!)
  \item Given the input string
    \texttt{D\underline{YNAM}ICPROGRAMMING\underline{MANY}TIMES}, your
    algorithm should return 4, for the substring \texttt{YNAM}. (It
    should not return 6, for the subsequence \texttt{YNAMIR}, because
    it's not contiguous.).
  \end{itemize}
\newpage

  \question[20] (Word segmentation) If English were written without
  spaces, then we need to infer likely boundaries between consecutive
  words in the text. This is called word segmentation. For example,
  given \texttt{meetateight}, you can probably decide that the best
  segmentation is \texttt{meet{\char32}at{\char32}eight}. (and not
  \texttt{me{\char32}et{\char32}at{\char32}eight}, or
  \texttt{meet{\char32}ate{\char32}ight}. This is all the more
  relevant in languages like Chinese and Japanese, which are written
  without spaces between the words.

  How could we automate this process?  A simple approach that is at
  least reasonably effective is to find a segmentation that maximizes
  the cumulative ``quality'' of its individual constituent
  words. Thus, suppose you are given a black box that, for any string
  of letters $x = x_1x_2\ldots x_n$, will return a number
  $quality(x)$. This number can be either positive or negative; larger
  numbers correspond to more plausible English words. (So
  $quality(\texttt{me})$ would be positive, while
  $quality(\texttt{ght})$ would be negative.)

  Given a long string of letters $y = y_1y_2\ldots y_n$, a
  segmentation of $y$ is a partition of its letters into contiguous
  blocks of letters; each block corresponds to a word in the
  segmentation. The total quality of a segmentation is determined by
  adding up the qualities of each of its blocks. (So we’d get the
  right answer above provided that
  $quality(\texttt{meet}) + quality(\texttt{at}) +
  quality(\texttt{eight})$ was greater than the total quality of any
  other segmentation of the string.)

  Descrive and analyze an efficient algorithm that takes a string $y$
  and computes a segmentation of maximum total quality. (You can treat
  a single call to the black box computing $quality()$ as a single
  computational step.)
  \newpage

  \bonusquestion[10] (Greedy matrix-chain multiplication?) As
  suggested in class, one approach to choosing the matrix $A_k$ at
  which to split the subproduct $A_iA_{i+1}\ldots A_j$ is by selecting
  $k$ to minimize the quantity $d_{i-1}d_kd_j$. Does this always give
  an optimal solution. Prove it or give a counterexample (i.e., an
  instance of the matrix-chain multiplication problem for which this
  greedy approach yields a suboptimal solution.)

\end{questions}
\end{document}
