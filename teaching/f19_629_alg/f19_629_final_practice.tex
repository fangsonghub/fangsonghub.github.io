\documentclass[12pt,answers,addpoints]{exam}

%============Macros==================%
\usepackage{amsmath,amsfonts,amssymb,amsthm}
\usepackage{qcircuit}
\usepackage[margin=1in]{geometry}
%--------------Cosmetic----------------%
\usepackage{mathtools}
\usepackage{hyperref}
\hypersetup{
    colorlinks=true,
    linkcolor=blue,
    filecolor=magenta,      
    urlcolor=cyan,
}
\usepackage{fullpage}
\usepackage{microtype}
\usepackage{xspace}
\usepackage[svgnames]{xcolor}
\usepackage[sc]{mathpazo}
\usepackage{enumitem}
\setlist[enumerate]{itemsep=1pt,topsep=2pt}
\setlist[itemize]{itemsep=1pt,topsep=2pt}

%----------commands--------------------%
\newcommand{\class}[1]{\text{#1}}
\newcommand{\NP}{\class{NP}}

%----------Header--------------------%
\newcommand{\classn}{CSCE629 Analysis of Algorithms}
\newcommand{\classnabbr}{CSCE 629}
\newcommand{\school}{Texas A\&M U}
\newcommand{\term}{Fall 2019}
\newcommand{\examdate}{XX/XX/2019}
%\newcommand{\duedate}{}
\newcommand{\examnum}{Final Exam: Practice}
\newcommand{\studentname}{\makebox[1.5in]{\hrulefill}} % change it to your name
\pagestyle{head}
\firstpageheader{}{}{}
\runningheader{\classnabbr}{\examnum\ - Page \thepage\ of
  \numpages}{\term}
\vskip 1ex
\setlength{\headsep}{10pt}
\runningheadrule

%\qzheader                       % execute quiz commands

\begin{document}

\noindent
\begin{tabular*}{\textwidth}{l @{\extracolsep{\fill}} r
    @{\extracolsep{6pt}} r}
  {\Large\textbf{\examnum}} & \Large{\textbf{Name:}} & \studentname\\
  {\term}, {\classn} & &  {\examdate}\\
  \school && Prof. Fang Song
\end{tabular*}\\

\noindent\rule[2ex]{\textwidth}{1pt}

\subsection*{Instructions (please read carefully before start!)}

\begin{itemize}
\item This exam contains \numpages\ pages (including this cover page)
  and \numquestions\ questions, for the total of \numpoints \ points
  and \numbonuspoints\ bonus points.

\item You will have two hours (xx:xx - xx:xx) to finish the exam. Be
  strategic and allocate your time wisely.

\item You may bring two double-sided letter size (8.5-by-11'') study
  sheet. Any other resources and electronic devices are NOT permitted.
  
\item Your work will be graded on correctness and clarity. Make sure
  your hand writing is legible.
  
\item Don't forget to write your name on the top.
\end{itemize}

\begin{center}
\textbf{Grade Table} (for instructor use only)\\
\smallskip
\addpoints
\gradetable[v][questions]
\end{center}

\newpage

\begin{questions}
  
\question \emph{Short answers}. Joker and Batman agreed to give up
  violence and compete peacefully on algorithmic problems. Answer the
  following, and \textbf{always} justify your answers briefly.
\begin{parts}

\part[5] Joker goes first. He gives $n$ equal-sized intervals, and
  asks for the maximum number of compatible intervals.  Batman
  proposes a greedy algorithm that picks the intervals based on
  earliest start time. Is Batman's algorithm correct?

    \vskip 4cm

  \part[5] Batman's turn.  He describes a \emph{directed acyclic
      graph} where the edges have arbitrary weights and ask Joker to
    find the simple path from node $s$ to node $t$ of \emph{maximum}
    total weight. Joker decides to negate the weight on each edge and
    running a shortest path algorithm. Is Joker's algorithm correct?
    
    \vskip 4cm

  \part[5] In this round they change the rule. Each will make a
    statement and the other needs to decide if the statement is true
    or false. Joker states that if all capacities in a flow network
    are integers, then the value of the maximum flow must also be an
    integer. Help Batman tell true or false.


    \newpage
    % \vskip 4cm 

  \part[5] With your help, Batman won on last question. Now he makes
    the following statement. If a graph has a unique shortest path $P$
    from node $s$ to node $t$, and has a unique minimum spanning tree
    $T$, then every edge in $P$ must also be in $T$. Help Joker this
    time. 

    \vskip 4cm
    

  \part[5] In the last round, Joker challenges Batman with a
    tax-collecting task at the city council of Gotham, and asks Batman
    to resign forever unless he manages to design an $O(n^2\log n)$
    algorithm. Batman comes up with a divide-and-conquer algorithm
    where he can divide a problem into two subproblems both of size
    half of the original, and the combining step takes $O(n^2)$
    time. Does Batman need to resign?

    \vskip 4cm
    
  \part[5] Likewise, Batman wants to stop Joker from crimes, and Joker
    agrees not to commit any crime as long as Batman's question keeps
    him busy. After thinking hard for a while, Batman tells Joker the
    3-SAT problem, the perfect bipartite matching problem, and the
    notion of Karp reductions. Batman asks Joker to find a Karp
    reduction from 3-SAT to perfect bipartite matching. Given the
    state-of-art, how long do you think that Joker will be kept busy?
    
  \end{parts}

  \newpage

  \question (Network flow)
  \begin{parts}
  \part[8] Given a maximum $(s,t)$-flow for a flow network $G$, how to
    find a minimum $(s,t)$-cut and how much time does it take?

    \vskip 8cm
    
  \part[5] Suppose $G$ is a directed graph with integer edge
    capacities. Let $(A,B)$ be a minimum $(s,t)$-cut. Prove or
    disprove: if $G'$ is obtained by adding 1 to the capacity of every
    edge in $G$, then $(A,B)$ is a minimum $(s,t)$-cut in $G'$.

    \vskip 6cm

    \newpage 
    
  \part[12] You are given maximum $(s,t)$-flow $f$ for a flow network
    $G$. Subsequently, the capacity of a specific edge $e^* \in G$ is
    reduced by 1. 

    \begin{enumerate}[label=\roman*)]
    \item Given an algorithm that finds the maximum flow in the
      resulting graph $G'$ in $O(m+n)$ time. You may assume the flow
      $f$ is \emph{acyclic}: there is no cycle in $G$ on which all
      edges carry positive flow.

      \vskip 8cm
    \item Prove the correctness and the $O(m+n)$ running time of your
      algorithm.
    \end{enumerate}
  \end{parts}
  \newpage

\question (Coin changing) Consider the problem of making change for
  $n$ cents using the fewest number of coins. Assume that each coin’s
  value is an integer. 

  \begin{parts}
  \part[5] Prof. Greedy proposes an algorithm that always uses coins
    of the largest value as many as possible and then moves to the
    next largest one. Prove that if the coins available consist of
    quarters (25 cents), dimes (10 cents), nickels (5 cents), and
    pennies (1 cent), then Prof. Greedy's algorithm yields an optimal
    solution.

    \vskip 8cm 
  \part[5] Consider other sets of coin denominations. Does this
    algorithm always yield an \emph{optimal} solution? We assume the
    set always contains pennies so that there is a solution for every
    value of $n$.
    
    \newpage
    
  \part[10] Give an $O(nk)$-time algorithm that makes change for any
    set of $k$ different coin denominations. Again we assume pennies
    are always included.

  \end{parts}
\newpage
  

\question ($k$-spanning tree) In a (minimum) spanning tree of a graph,
  it could occur that some vertex is a bottleneck in the sense that
  its degree is high. In \emph{$k$-spanning tree} problem, we intend
  to keep the degrees low in a spanning tree. More precisely, we are
  given an undirected graph $G=(V,E)$, and the goal is to decide if
  there is a spanning tree in $G$ in which each vertex has degree at
  most $k$.

  \begin{parts}
  \part[7] Show that $k$-spanning tree is in $\NP$ for any $k\ge 2$.
    \vskip 5cm
  \part[10] Show that $2$-spanning tree is \NP-complete. [Hint: consider
    the relation with Hamiltonian cycle.]

    \newpage 
  \part[8] Show that, in fact, $k$-spanning tree is
    \NP-complete for any $k\ge 2$. 

  \end{parts}
  
\newpage
\end{questions}

\begin{center}
  Scratch paper -- no exam questions here.
\end{center}


\newpage
\begin{center}
  Scratch paper -- no exam questions here.
\end{center}

\newpage

\begin{center}
  \Large{Guideline on grading rubrics}
\end{center}

We will refer to these rubrics for certain types of questions. They
are general guidelines and are subject to change. They are stated on a
10-point basis, and will be scaled accordingly.

\paragraph{Dynamic Programming}

\begin{itemize}
\item \textbf{6 points for a correct recurrence}, described either
  using functional notation or as pseudocode for a recursive
  algorithm.

  \begin{itemize}[label=+]
  \item 1 point for a clear English description of the function you
    are trying to evaluate.  (Otherwise, we don’t even know what
    you’re trying to do.) Automatic zero if the English description is
    missing.
  \item 1 point for stating how to call your recursive function to
    get the final answer.
  \item 1 point for the base case(s). 
  \item 3 points for the recursive case(s). 
  \end{itemize}
\item 4 points for iterative details

  \begin{itemize}[label=+]
  \item 1 point for describing the memoization data structure; a clear
    picture may be sufficient.
  \item 2 points for describing a correct evaluation order; a clear
    picture may be sufficient. If you use nested loops, be sure to
    specify the nesting order.
  \item 1 point for running time
  \end{itemize}
\item Proofs of correctness are not required for full credit on exams,
  unless the problem specifically asks for one.

\item For problems that ask for an algorithm that computes an optimal
  structure—such as a subset, partition, subsequence, or tree, an
  algorithm that computes only the value or cost of the optimal
  structure is sufficient for full credit, unless the problem says
  otherwise. 

\end{itemize}

\paragraph{Graph reductions}

For problems solved by reducing them to a standard graph algorithm
covered in class (for example: shortest paths, topological sort,
minimum spanning trees, maximum flows, bipartite maximum matching):

\begin{itemize}
\item 1 point for listing the vertices of the graph. (If the original
  input is a graph, describing how to modify that graph is
  fine.)

\item 1 point for listing the edges of the graph, including whether
  the edges are directed or undirected. (If the original input is a
  graph, describing how to modify that graph is fine.)

\item 1 point for describing appropriate weights and/or lengths and/or
  capacities and/or costs and/or demands and/or whatever for the
  vertices and edges.

\item 2 points for an explicit description of the problem being solved
  on that graph. (For example: ``We compute the shortest path in $G$
  from $u$ to $v$.'')

\item 3 points for other algorithmic details, assuming the rest of the
  reduction is correct.

  \begin{itemize}[label=+]
    
  \item 1 point for describing how to build the graph from the
    original input (for example: ``by brute force'').

  \item 1 point for describing the algorithm you use to solve the
    graph problem (for example: ``Dijstra's algorithm'')

  \item 1 point for describing how to extract the output for the
    original problem from the output of the graph algorithm.

  \end{itemize}

\item 2 points for the running time, expressed in terms of the
  original input parameters.
  
\item If the problem explicitly asks for a proof of correctness,
  divide all previous points in half and add 5 points for proof of
  correctness.
  
\end{itemize}
\paragraph{NP-Complete reductions}

For problems that ask ''Prove that $X$ is \NP-Complete'':

\begin{itemize}

\item \textbf{4 points for the polynomial-time reduction}:

  \begin{itemize}[label=+]
  \item 1 point for explicitly naming the \NP-hard problem $Y$ to
    reduce from. You may use any of the problems listed in the lecture
    notes; a list of NP-hard problems will appear on the back page of
    the exam.
  \item 2 points for describing the polynomial-time algorithm to
    transform arbitrary instances of $Y$ into inputs to instances of black-box
    algorithm for $X$

  \item 1 point for describing the polynomial-time algorithm to
    transform the output of the black-box algorithm for $X$ into the
    output for $Y$.

  \item Reductions that call a black-box algorithm for $X$ more than
    once are acceptable. You do not need to analyze the running time
    of your resulting algorithm for $Y$, but it must be polynomial in
    the size of the input instance of $Y$.
    
\end{itemize}
\item \textbf{6 points for the proof of correctness}. This is the
  entire point of the problem. These proofs always have two parts; for
  example:
  \begin{itemize}[label=+]
  \item 3 points for proving that your reduction transforms positive
    instances of $Y$ into positive instances of $X$.

  \item 3 points for proving that your reduction transforms negative
    instances of $Y$ into negative instances of $X$.

  \item These proofs do not need to be as detailed as in the book or
    homework; however, it must be clear that you have at least
    considered all possible cases. We are really just looking for
    compelling evidence that you understand why your reduction is
    correct.

  \item It is still possible to get partial credit for an incorrect
    reduction. For example, if you describe a reduction that sometimes
    reports false positives, but you prove that all False answers are
    correct, you would still get 3 points for half of the correctness
    proof.
  \end{itemize}
\item \textbf{2 points for showing $X\in\NP$}, if this has not been
  shown in a previous problem. Relocate 1 point each from the
  reduction and the correctness proof.

\item Zero points for reducing $X$ to some NP-hard problem $Y$.

\item Zero points for attempting to solve $X$.

\end{itemize}

\end{document}

%%% Local Variables: 
%%% mode: latex
%%% TeX-master: t
%%% End: 




