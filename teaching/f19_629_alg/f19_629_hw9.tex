% Instructions: you don't need to change anything in the macros, but
% feel free to define new commands as you wish. Starting from the main
% body, change the specs (e.g., your
% name). use \begin{solution} \end{solution} environment to write your
% solutions. Don't forget to list your collaborators.

\documentclass[12pt,addpoints,answers]{exam}
%============Macros==================%
\usepackage{amsmath,amsfonts,amssymb,amsthm}
\usepackage[margin=1in]{geometry}
%--------------Cosmetic----------------%
\usepackage{mathtools}
\usepackage{hyperref}
\usepackage{fullpage}
\usepackage{microtype}
\usepackage{xspace}
\usepackage[svgnames]{xcolor}
\usepackage[sc]{mathpazo}
\usepackage{enumitem}
\setlist[enumerate]{itemsep=1pt,topsep=2pt}
\setlist[itemize]{itemsep=1pt,topsep=2pt}
%----------Header--------------------%
\def\course{CSCE629 Analysis of Algorithms}
\def\term{Texas A\&M U, Fall 2019}
\def\prof{Lecturer: Fang Song}
\newcommand{\handout}[5]{
   \renewcommand{\thepage}{\arabic{page}}
   \begin{center}
   \framebox{
      \vbox{
    \hbox to 5.78in { \hfill \large{\course} \hfill }
       \vspace{2mm}
       \hbox to 5.78in { {\Large \hfill \textbf{#5}  \hfill} }
       \vspace{2mm}
       \hbox to 5.78in { \term \hfill \emph{#2}}
       \hbox to 5.78in { {#3 \hfill \emph{#4}}}
      }
   }
   \end{center}
   \vspace*{4mm}
}
\newcommand{\hw}[4]{\handout{#1}{#2}{#3}{#4}{Homework #1}}

%-----defs and commands-----%
%\def\mG{[\textbf{G}]\xspace}
\def\veps{\varepsilon}
\newcommand{\bit}{\{0,1\}}
\newcommand{\negl}{\text{negl}}
\newcommand{\corr}[1]{{\color{blue}{#1}}}
\newcommand{\alg}[1]{\textsf{#1}}
%=======Main document==============%
\begin{document}

%----Specs: change accordingly-----%
\newif\ifstudent % comment out false
 \studenttrue 
% \studentfalse

\def\hwnum{9}
\def\issuedate{11/01/19}
\def\duedate{10am, 11/08/19} % 
\def\yourname{your name} % put your name here
%------------------------------%

\ifstudent
\hw{\hwnum}{\issuedate}{Student: \yourname}{Due: \duedate}%
\else
\hw{\hwnum}{\issuedate}{\prof}{Due: \duedate}%
\fi

\noindent \textbf{Instructions.}

\begin{itemize} 
\item Typeset your submission by \LaTeX, and submit in PDF
  format. Your solutions will be graded on \emph{correctness} and
  \emph{clarity}. You should only submit work that you believe to be
  correct, and you will get significantly more partial credit if you
  clearly identify the gap(s) in your solution. You may opt for the
  ``I’ll take 15\%'' option.
\item You may collaborate with others on this problem set.  However,
  you must \textbf{{write up your own solutions}} and \textbf{{list
      your collaborators and any external sources}} for each
  problem. Be ready to explain your solutions orally to a course staff
  if asked.
\item For problems that require you to provide an algorithm, you must
  give a precise description of the algorithm, together with a proof
  of correctness and an analysis of its running time. You may use
  algorithms from class as subroutines. You may also use any facts
  that we proved in class or from the book.
\item \textbf{If you describe a Greedy algorithm, you will get no
    credit without a formal proof of correctness, even if your
    algorithm is correct.}
\end{itemize}

\noindent This assignment contains \numquestions\ questions,
\numpages\ pages for the total of \numpoints \ points and
\numbonuspoints\ bonus points. A random subset of the problems will be
graded. \medskip

\paragraph{Exercises. Do not turn in.}
\begin{questions}
  \question (Max bipartite matching)

  \begin{parts}
    \part Give a linear-programming formulation of the bipartite
    maximum matching problem. The input is a bipartite graph
    $G = (U \cup V; E)$, where $E \subseteq U \times V$; the output is
    the largest matching in $G$. Your linear program should have one
    variable for each edge.
    
    \part Now dualize the linear program from part (a). What do the
    dual variables represent? What does the objective function
    represent?  What problem is this!?
  \end{parts}
\end{questions}
\newpage 
\paragraph{Problems to turn in.}

\begin{questions}

\question[15] (Vertex cover) A \emph{vertex cover} of an undirected
  graph $G = (V, E)$ is a subset of the vertices which touches every
  edge—that is, a subset $S \subseteq V$ such that for each edge
  ${u,v} \in E$, one or both of $u,v$ are in $S$.  Describe and
  analyze an algorithm, as efficient as you can, to find a minimum
  vertex cover in a bipartite graph.

  \newpage

  \question (Updating max flow) You are given a flow network
  $G = (V,E)$ with source $s$ and sink $t$, and integer capacities.
  \begin{parts}
    \part[10] Suppose that you are given a max flow in $G$. Now we
    increase the capacity of a single edge $(u,v)\in E$ by 1. Given an
    $O(m+n)$-time algorithm to update the max flow. 
    \part[10] Now suppose all edges have unit capacity and you are
    given a parameter $k$. The goal is to delete $k$ edges so as to
    reduce the maximum $s-t$ flow in $G$ as much as possible. In other
    words, you should find a set of edges $F\subseteq E$ so that
    $|F|=k$ and the maximum $s-t$ flow in $G'=(V,E-F)$ is as small as
    possible subject to this. Describe and analyze a polynomial-time
    algorithm to solve this problem.
  \end{parts}

  \newpage
  

  \question[15] (Filling classrooms) Faced with the threat of brutally
  severe budget cuts, Potemkin University has decided to hire actors
  to sit in classes as ``students'', to ensure that every class they
  offer is completely full. Because actors are expensive, the
  university wants to hire as few of them as possible.

  Building on their previous leadership experience at the now-defunct
  Sham-Poobanana University, the administrators at Potemkin have given
  you a directed acyclic graph $G = (V,E)$, whose vertices represent
  classes, and where each edge $i\to j$ indicates that the same
  ``student'' can attend class $i$ and then later attend class $j$. In
  addition, you are also given an array $cap[1,\ldots, V]$ listing the
  maximum number of ``students'' who can take each class. Describe an
  analyze an algorithm to compute the minimum number of
  ``students''that would allow every class to be filled to capacity.

\end{questions}
\end{document}
