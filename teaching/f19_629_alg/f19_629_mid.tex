\documentclass[12pt,answers,addpoints]{exam}

%============Macros==================%
\usepackage{amsmath,amsfonts,amssymb,amsthm}
\usepackage{qcircuit}
\usepackage[margin=1in]{geometry}
%--------------Cosmetic----------------%
\usepackage{mathtools}
\usepackage{hyperref}
\hypersetup{
    colorlinks=true,
    linkcolor=blue,
    filecolor=magenta,      
    urlcolor=cyan,
}
\usepackage{fullpage}
\usepackage{microtype}
\usepackage{xspace}
\usepackage[svgnames]{xcolor}
\usepackage[sc]{mathpazo}
\usepackage{enumitem}
\setlist[enumerate]{itemsep=1pt,topsep=2pt}
\setlist[itemize]{itemsep=1pt,topsep=2pt}
%----------Header--------------------%
\newcommand{\classn}{CSCE629 Analysis of Algorithms}
\newcommand{\classnabbr}{CSCE 629}
\newcommand{\school}{Texas A\&M U}
\newcommand{\term}{Fall 2019}
\newcommand{\examdate}{Oct. 11, 2019}
\newcommand{\duedate}{10am, Oct. 14, 2019}
\newcommand{\examnum}{Mid-term Exam}
\newcommand{\studentname}{\makebox[1.5in]{\hrulefill}} % change it to your name
\pagestyle{head}
\firstpageheader{}{}{}
\runningheader{\classnabbr}{\examnum\ - Page \thepage\ of
  \numpages}{\term}
\vskip 1ex
\setlength{\headsep}{10pt}
\runningheadrule

%\qzheader                       % execute quiz commands

\begin{document}

\noindent
\begin{tabular*}{\textwidth}{l @{\extracolsep{\fill}} r
    @{\extracolsep{6pt}} r}
  {\Large\textbf{\examnum}} & \Large{\textbf{Name:}} & \studentname\\
  {\term}, {\classn} & &  {\examdate}\\
  \school && Prof. Fang Song
\end{tabular*}\\

\rule[2ex]{\textwidth}{1pt}

\subsection*{Instructions (please read carefully before start!)}

\begin{itemize}
\item This take-home exam contains \numpages\ pages (including this
  cover page) and \numquestions\ questions. Total of points is
  \numpoints.
\item You will have till \textbf{\duedate} to finish the exam. You
  must work on your own, and no collaboration or help from any
  resources other than those made available in class (lecture notes,
  texts, homework problems, etc.) is permitted.

\item Submit your solutions in PDF on Gradescope before the deadline,
  either scanned or typeset in \LaTeX. If you choose to hand-write and
  scan, \emph{print out this exam sheet and write your solutions on
    it}. Do your best to fit your answers into the space provided, and
  attach extra papers only if necessary. If you typeset in \LaTeX,
  \emph{use the provided TeX file}. No other formats are accepted.

\item Your work will be graded on correctness and clarity. Make sure
  your hand writing is legible. You may opt for the “I’ll take 15\%”
  option.
  
\item Don't forget to write your name on top (or update the
  ``{\textbackslash}studentname'' command in the TeX file)!
\end{itemize}

\begin{center}
\textbf{Grade Table} (for instructor use only)\\
\smallskip
\addpoints
\gradetable[v][questions]
\end{center}

\newpage

\begin{questions}
  \question \emph{Short answers}. Answer the following, and briefly
  justify your answer.
  \begin{parts}
  \part[5] You are working on a divide-and-conquer algorithm that
    given an input of size $n$, calls itself recursively on 8
    subproblems of size $\lceil n/5 \rceil$. Dividing and combining
    take time $f(n)$. Can the total running time of the algorithm be
    $O(n)$?


    \vskip 3cm
   

  \part[5] Rank the following functions by \emph{ascending} order of
    growth. That is, find an arrangement of $g_1,g_2,\ldots$ such that
    $g_1(n) = O(g_2(n)), g_2(n) = O(g_3(n)),\ldots$. If two function
    satisfy $g_i(n) = \Theta(g_j(n))$, they can be in either
    order. Recall that $\log (\cdot)$ is base 2 logarithm and
    $\log_b(\cdot)$ is base $b$ logarithm.

    \[5^{\log n}, \log^{91}n, \sqrt[5]n, 2^{\sqrt[3]{n}}, \log(\sqrt{n!}),
      3^{log_3 n}, n^{2 \log 7} \, .\]

    \vskip 3cm

  \part[5] Your friend Hulk proposes adapting Dijkstra's algorithm in
    the following way to deal with negative-length edges. Pick the
    edge with the smallest length $\ell < 0$ (e.g., -10), and then
    increase the length of each edge by $|\ell|$ so all edges will
    become non-negative. Then run Dijkstra's on $G$ with this updated
    lengths to find a shortest path from $s$ to $t$. Does this also
    give you a shortest $s-t$ path in the original graph? Justify your
    answer.
    \vskip 3cm

  \part[5] We have a connected graph $G=(V,E)$. Suppose that we run
    both BFS and DFS on a vertex $u\in V$, and obtain the same BFS
    search tree and DFS search tree $T$, which contains all vertices
    of $G$. Prove or disapprove: $G = T$.

  \end{parts}
\newpage

\question Consider a function $f$ that takes two integer inputs $i,j$
  in $\{1,\ldots,n\}$ and returns a real number. A \emph{local
    minimum} of $f$ is a point $(i,j)$ such that $f(i,j) \leq f(a,b)$
  for all pairs $(a,b)\in \{1,\ldots,n\}^2$ where $|a - i| \leq 1$ and
  $|b-j| \leq 1$.

    The goal of this problem is to find an efficient algorithm that
    finds a local minimum of $f$, assuming nothing about the structure
    of $f$ except that for any input $(i,j)$, evaluating $f(i,j)$
    takes \emph{unit} time. 

    We can represent this problem via a grid-like graph $G_n$, where
    the vertices are pairs of integers and two pairs are connected if
    each of their components differs by at most $1$, that is
    $((i,j),(a,b))\in E$ iff. $|a - i| \leq 1$ and $|b - j| \leq
    1$. Note that $G_n$ has $n^2$ vertices and degree at most $8$. We
    can think of $f$ as a function that assigns a real number to each
    vertex. A local minimum is then a vertex $v$ such that
    $f(v) \leq f(v')$ for all adjacent vertices $v'$.
    \begin{parts}
\part[10] Give a recursive algorithm for this problem that cuts the
  graph $G_n$ into four sub-graphs of the form $G_{n/2}$, and gets
  called recursively on at most one of the subgraphs.
\newpage
\part[5] Prove correctness of your algorithm.
\vskip 8cm
\part[10] State a recurrence that describes the worst-case running time
  of your algorithm. Solve it using any method of your choice.

\end{parts}
\newpage
\question You have been hired to plan the locations of rest stops
  along a stretch of highway. There are $n$ potential locations for
  the rest stops. Location $i$ is $d_i$ miles from the start point of
  the highway. Building a rest stop at location $i$ costs $c_i$
  dollars. The cost of building at a set of locations is the sum of
  the individual costs. You need to design an algorithm that takes the
  numbers $(d_i,c_i)$ and returns the minimum-cost set of locations
  such that no two consecutive locations are more than 30miles apart,
  and there is at least one rest stop within 30miles of both the start
  and end of the highway. Assuming that the input comes sorted in
  \emph{ascending} order according to $d_i$.

  \begin{parts}
  \part[10] You realize that dynamic programming is helpful here,
    assuming that all potential locations are at least 5miles
    apart. Describe the subproblems for which your algorithm compute
    solutions.

    \vskip 6cm
    
  \part[7] Give the pseudocode of your algorithm.
    \newpage 
  \part[8] State your algorithm's (worst-case) running time and justify
    your answer. 
  \end{parts}
    \newpage

  \question[20] Suppose you are given a set $L$ of $n$ line segments
    in the plane, where each segment has one endpoint on the vertical
    line $x = 0$ and one endpoint on the vertical line $x = 1$, and
    all $2n$ endpoints are distinct. Describe and analyze an algorithm
    to compute the largest subset of $L$ in which no pair of segments
    intersects.


  
\end{questions}
\newpage
\begin{center}
Scrap paper -- no exam questions here.  
\end{center}


\end{document}

%%% Local Variables: 
%%% mode: latex
%%% TeX-master: t
%%% End: 
