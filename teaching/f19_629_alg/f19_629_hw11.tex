% Instructions: you don't need to change anything in the macros, but
% feel free to define new commands as you wish. Starting from the main
% body, change the specs (e.g., your
% name). use \begin{solution} \end{solution} environment to write your
% solutions. Don't forget to list your collaborators.

\documentclass[12pt,addpoints,answers]{exam}
%============Macros==================%
\usepackage{amsmath,amsfonts,amssymb,amsthm}
\usepackage[margin=1in]{geometry}
%--------------Cosmetic----------------%
\usepackage{mathtools}
\usepackage{hyperref}
\usepackage{fullpage}
\usepackage{microtype}
\usepackage{xspace}
\usepackage[svgnames]{xcolor}
\usepackage[sc]{mathpazo}
\usepackage{enumitem}
\setlist[enumerate]{itemsep=1pt,topsep=2pt}
\setlist[itemize]{itemsep=1pt,topsep=2pt}
%----------Header--------------------%
\def\course{CSCE629 Analysis of Algorithms}
\def\term{Texas A\&M U, Fall 2019}
\def\prof{Lecturer: Fang Song}
\newcommand{\handout}[5]{
   \renewcommand{\thepage}{\arabic{page}}
   \begin{center}
   \framebox{
      \vbox{
    \hbox to 5.78in { \hfill \large{\course} \hfill }
       \vspace{2mm}
       \hbox to 5.78in { {\Large \hfill \textbf{#5}  \hfill} }
       \vspace{2mm}
       \hbox to 5.78in { \term \hfill \emph{#2}}
       \hbox to 5.78in { {#3 \hfill \emph{#4}}}
      }
   }
   \end{center}
   \vspace*{4mm}
}
\newcommand{\hw}[4]{\handout{#1}{#2}{#3}{#4}{Homework #1}}

%-----defs and commands-----%
%\def\mG{[\textbf{G}]\xspace}
\def\veps{\varepsilon}
\newcommand{\bit}{\{0,1\}}
\newcommand{\negl}{\text{negl}}
\newcommand{\corr}[1]{{\color{blue}{#1}}}
\newcommand{\alg}[1]{\textsf{#1}}
\newcommand{\class}[1]{\text{#1}}
\newcommand{\NP}{\class{NP}}
%=======Main document==============%
\begin{document}

%----Specs: change accordingly-----%
\newif\ifstudent % comment out false
 \studenttrue 
% \studentfalse

\def\hwnum{11}
\def\issuedate{11/15/19}
\def\duedate{10am, 11/22/19} % 
\def\yourname{your name} % put your name here
%------------------------------%

\ifstudent
\hw{\hwnum}{\issuedate}{Student: \yourname}{Due: \duedate}%
\else
\hw{\hwnum}{\issuedate}{\prof}{Due: \duedate}%
\fi

\noindent \textbf{Instructions.}

\begin{itemize} 
\item Typeset your submission by \LaTeX, and submit in PDF
  format. Your solutions will be graded on \emph{correctness} and
  \emph{clarity}. You should only submit work that you believe to be
  correct, and you will get significantly more partial credit if you
  clearly identify the gap(s) in your solution. You may opt for the
  ``I’ll take 15\%'' option.
\item You may collaborate with others on this problem set.  However,
  you must \textbf{{write up your own solutions}} and \textbf{{list
      your collaborators and any external sources}} for each
  problem. Be ready to explain your solutions orally to a course staff
  if asked.
\item For problems that require you to provide an algorithm, you must
  give a precise description of the algorithm, together with a proof
  of correctness and an analysis of its running time. You may use
  algorithms from class as subroutines. You may also use any facts
  that we proved in class or from the book.
\item \textbf{If you describe a Greedy algorithm, you will get no
    credit without a formal proof of correctness, even if your
    algorithm is correct.}
\end{itemize}

\noindent This assignment contains \numquestions\ questions,
\numpages\ pages for the total of \numpoints \ points and
\numbonuspoints\ bonus points. A random subset of the problems will be
graded. \medskip

\paragraph{Exercises. Do not turn in.}
\begin{questions}
  \question A clique in an undirected graph $G =(V, E)$ is a subset
  $V' \subseteq V$ of vertices, each pair of which is connected by an
  edge in $E$. (In other words, a clique is a \emph{complete} subgraph
  of $G$. The size of a clique is the number of vertices it
  contains. The Clique problems asks to decide whether a clique of a
  given size $k$ exists in the graph. Show that Clique is
  \NP-complete.

  \question Given an integer $m\times n$ matrix $A$ and an integer
  $m$-vector $b$, the 0-1 integerprogramming problem asks whether
  there exists an integer $n$-vector $x$ with elements in the set \bit
  suchthat $Ax =b$. Prove that 0-1 integer programming is
  \NP-complete. (Hint: Reduce from 3-CNF-SAT.)

\end{questions}
\newpage 
\paragraph{Problems to turn in.}

\begin{questions}

\question (Bonnie and Clyde) Bonnie and Clyde have just robbed a
  bank. They have a bag of money and want to divide it up. For each of
  the following scenarios, either give a polynomial-time algorithm, or
  prove that the problem is \NP-complete. The input in each case is a
  list of the $n$ items in the bag, along with the value of each.
  \begin{parts}
    \part[10] The bag contains $n$ coins, but only 2 different
    denominations: some coins are worth $x$ dollars, and some are
    worth $y$ dollars. Bonnie and Clyde wish to divide the money
    exactly evenly.
    \part[10] The bag contains $n$ coins, with an arbitrary number of
    different denominations, but each denomination is a nonnegative
    integer power of 2, i.e., the possible denominations are 1 dollar,
    2 dollars, 4 dollars, etc. Bonnie and Clyde wish to divide the
    money exactly evenly.

    \part[10] The bag contains $n$ checks, which are, in an amazing
    coincidence, made out to “Bonnie or Clyde.” They wish to divide
    the checks so that they each get the exact same amount of money.
    \part[10] The bag contains $n$ checks as in part (c), but this
    time Bonnie and Clyde are willing to accept a split in which the
    difference is no larger than 100 dollars.
\end{parts}
  \newpage

  \question[15] (Galactic Shortest Path) This problem originates from
  the \emph{Star Wars} movies. Luke, Leia, and friends are trying to
  make their way from the Death Star back to the hidden Rebel base. We
  can view the galaxy as an undirected graph $G = (V, E)$, where each
  node is a star system and an edge $(u, v)$ indicates that one can
  travel directly between $u$ and $v$. The Death Star is represented
  by a node $s$, the hidden Rebel base by a node $t$. Certain edges in
  this graph represent longer distances than others; thus each edge
  $e$ has an integer length $\ell_e\ge 0$. Also, certain edges
  represent routes that are more heavily patrolled by evil Imperial
  spacecraft; so each edge $e$ also has an integer risk $r_e \ge 0$,
  indicating the expected amount of damage incurred from
  special-effects-intensive space battles if one traverses this edge.
  
  It would be safest to travel through the outer rim of the galaxy,
  from one quiet upstate star system to another; but then one's ship
  would run out of fuel long before getting to its
  destination. Alternately, it would be quickest to plunge through the
  cosmopolitan core of the galaxy; but then there would be far too
  many Imperial spacecraft to deal with. In general, for any path $P$
  from $s$ to $t$, we can define its total length to be the sum of the
  lengths of all its edges; and we can define its total risk to be the
  sum of the risks of all its edges.

  So Luke, Leia, and company are looking at a complex type of shortest
  path problem in this graph: they need to get from $s$ to $t$ along a
  path whose total length and total risk are both reasonably small. In
  concrete terms, we can phrase the \emph{Galactic Shortest-Path
    Problem} as follows: Given a setup as above, and integer bounds
  $L$ and $R$, is there a path from $s$ to $t$ whose total length is
  at most $L$, and whose total risk is at most $R$?

  Describe a poly-time algorithm or prove that Galactic Shortest Path
  is \NP-complete.
  
\end{questions}
\end{document}


