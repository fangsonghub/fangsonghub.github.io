\documentclass[12pt,answers,addpoints]{exam}

%============Macros==================%
\usepackage{amsmath,amsfonts,amssymb,amsthm}
\usepackage{qcircuit}
\usepackage[margin=1in]{geometry}
%--------------Cosmetic----------------%
\usepackage{mathtools}
\usepackage{hyperref}
\hypersetup{
    colorlinks=true,
    linkcolor=blue,
    filecolor=magenta,      
    urlcolor=cyan,
}
\usepackage{fullpage}
\usepackage{microtype}
\usepackage{xspace}
\usepackage[svgnames]{xcolor}
\usepackage[sc]{mathpazo}
\usepackage{enumitem}
\setlist[enumerate]{itemsep=1pt,topsep=2pt}
\setlist[itemize]{itemsep=1pt,topsep=2pt}
%----------Header--------------------%
\newcommand{\classn}{CSCE629 Analysis of Algorithms}
\newcommand{\classnabbr}{CSCE 629}
\newcommand{\school}{Texas A\&M U}
\newcommand{\term}{Fall 2019}
\newcommand{\examdate}{XX/XX/2019}
\newcommand{\duedate}{XX/XX/2019}
\newcommand{\examnum}{Practice Mid-term}
\newcommand{\studentname}{\makebox[1.5in]{\hrulefill}} % change it to your name
\pagestyle{head}
\firstpageheader{}{}{}
\runningheader{\classnabbr}{\examnum\ - Page \thepage\ of
  \numpages}{\term}
\vskip 1ex
\setlength{\headsep}{10pt}
\runningheadrule

%\qzheader                       % execute quiz commands

\begin{document}

\noindent
\begin{tabular*}{\textwidth}{l @{\extracolsep{\fill}} r
    @{\extracolsep{6pt}} r}
  {\Large\textbf{\examnum}} & \Large{\textbf{Name:}} & \studentname\\
  {\term}, {\classn} & &  {\examdate}\\
  \school && Prof. Fang Song
\end{tabular*}\\

\rule[2ex]{\textwidth}{1pt}

\subsection*{Instructions (please read carefully before start!)}

\begin{itemize}
\item This take-home exam contains \numpages\ pages (including this
  cover page) and \numquestions\ questions. Total of points is
  \numpoints.
\item You will have till \textbf{\duedate} to finish the exam. You
  must work on your own, and no collaboration or help from any
  resources other than those made available in class (lecture notes,
  texts, homework problems, etc.) is permitted.

\item Submit your solutions in PDF on Gradescope before the deadline,
  either scanned or typeset in \LaTeX. If you choose to hand-write and
  scan, \emph{print out this exam sheet and write your solutions on
    it}. Do your best to fit your answers into the space provided, and
  attach extra papers only if necessary. If you typeset in \LaTeX,
  \emph{use the provided TeX file}. No other formats are accepted.

\item Your work will be graded on correctness and clarity. Make sure
  your hand writing is legible.
  
\item Don't forget to write your name on top (or update the
  ``{\textbackslash}studentname'' command in the TeX file)!
\end{itemize}

\begin{center}
\textbf{Grade Table} (for instructor use only)\\
\smallskip
\addpoints
\gradetable[v][questions]
\end{center}

\newpage

\begin{questions}
  \question \emph{Short answers}. Answer the following, and briefly
  justify your answer.
  \begin{parts}
  \part[5] (Sample problem) You are working on a divide-and-conquer
    algorithm that given an input of size $n$, calls itself
    recursively on 8 subproblems of size $\lceil n/3\rceil$. Dividing
    and combining take time $f(n)$. Can the total running time of the
    algorithm be $O(n^2)$? 


    \begin{solution}
      Answer: Yes.\\
      Justification: The recurrence is
      $T(n) = 8 T(\lceil n/3 \rceil) + f(n)$. Since
      $\log_b^a = \log_3 8 < \log_3 9 =2$. By Master's theorem, if
      $f(n) = n^2$, it grows polynomially faster than $n^{\log_b
        a}$. Moreover, $8f(n/3) \leq c n^2$ for $c = 8/9 <1$
      (regularity condition). Hence the solution to the recurrence is
      $O(n^2)$.
    \end{solution}
  \part[5] You are given a directed graph $G = (V, E)$ with $|V | = n$
    vertices and $|E| = m$ edges. Let $G^R$ be the graph obtained by
    reversing the directions of all the edges in $G$. Prove or
    disprove: you can compute adjacency matrix of $G^R$ given the
    adjacency matrix of $G$ in $o(n^2)$ time. 

    \vskip 3cm
    
    \part[5] Given a set $V$ of $n$ points in the plane, let $\delta$ be
    the distance between two closest points in $V$. Consider the graph
    $G_{2\delta}$ with vertex set $V$ and an edge $(u,v)$ for every
    pair of points $u$ and $v$ in $V$ at distance $2\delta$ or
    less. Prove or disprove: $G_{2\delta}$ can have $\Omega(n\log n)$
    edges.

    \vskip 3cm
  \part[5] Rank the following functions by ascending order of
    growth. That is, find an arrangement of $g_1,g_2,\ldots$ such that
    $g_1(n) = O(g_2(n)), g_2(n) = O(g_3(n)),\ldots$. If two function
    satisfy $g_i(n) = \Theta(g_j(n))$, they can be in either
    order. Recall that $\log (\cdot)$ is base 2 logarithm and
    $\log_b(\cdot)$ is base $b$ logarithm. 

    \[3^{\log n}, \log^{19}n, \sqrt n, 2^{\sqrt[2]{n}}, \log(n!),
      4^{log_4 n}, n^{\log 7} \, .\]

    \vskip 3cm
    
  \end{parts}

  \newpage

\question Given an array of temperatures $t_1, \ldots, t_n$ measured
  on $n$ days, the weather service would like to compare the biggest
  increase in temperature to the biggest decrease in temperature over
  the given period. Specifically, you would like to compute two
  numbers: (a) $\max_{up}$, the maximum over pairs $i < j$ of
  $t_j - t_i$ and (b) $\max_{down}$, the maximum over pairs $i < j$ of
  $t_i - t_j$. For example, for the array $[10, 0, 1, 2, 3, 4, -1]$,
  you would compute $\max_{up} = 4$ and $\max_{down} = 11$.

  \begin{parts}
  \part[20] Give a divide and conquer algorithm that computes these
    numbers in
    $O(n)$ time.  [Hint: Cut the array into two and make a recursive
    call on each half. It may be helpful to have the recursive calls
    pass up some extra information (in addition to
    $\max_{up}$ and $\max_{down}$).]

    \newpage
    
  \part Give a linear-time algorithm for the same problem which makes
    a single pass through the data and uses only a \emph{constant}
    amount of workspace (beyond the space needed to store the input).
    
  \end{parts}


  \newpage

  \question[20] You have reached the inevitable point in the semester
  where it is no longer possible to finish all of your assigned work
  without pulling at least a few all-nighters. The problem is that
  pulling successive all-nighters will burn you out, so you need to
  pace yourself (or something).

  Let’s model the situation as follows. There are $n$ days left in the
  semester. For simplicity, let's say you are taking one class, there
  are no weekends, there is an assignment due every single day until
  the end of the semester (would you like that?), and you will only
  work on an assignment the day before it is due. For each day $i$,
  you know two positive integers:
  \begin{itemize}
  \item $Score[i]$ is the score you will earn on the $i$th assignment
    if you do not pull an all-nighter the night before.
  \item $Bonus[i]$ is the number of additional points you could
    potentially earn if you do pull an all-nighter the night before.
  \end{itemize}
  However, pulling multiple all-nighters in a row has a price. If you
  turn in the $i$th assignment immediately after pulling $k$
  consecutive all-nighters, your actual score for that assignment will
  be $(Score[i] + Bonus[i])/2^{k-1}$.  Design and analyze an algorithm
  that computes the maximum total score you can achieve, given the
  arrays $Score[1,\ldots,n]$ and $Bonus[1, \ldots,n]$ as input.

  \newpage 
  \question[20] (Another algorithm design problem)
\end{questions}
\newpage
\begin{center}
Scrap paper -- no exam questions here.  
\end{center}


\end{document}

%%% Local Variables: 
%%% mode: latex
%%% TeX-master: t
%%% End: 
