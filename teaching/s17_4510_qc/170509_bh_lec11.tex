\documentclass[11pt]{article}
%============Macros==================%
% you don't need to change anything. start editing from main body.
\usepackage{amsmath,amsfonts,amssymb,amsthm}
\usepackage{qcircuit}
\usepackage[margin=1in]{geometry}
%--------------Cosmetic----------------%
\usepackage{mathtools}
\usepackage{hyperref}
\usepackage{fullpage}
\usepackage{microtype}
\usepackage{xspace}
\usepackage[svgnames]{xcolor}
\usepackage[sc]{mathpazo}
\usepackage{enumitem}
\setlist[enumerate]{itemsep=1pt,topsep=2pt}
\setlist[itemize]{itemsep=1pt,topsep=2pt}
%--------------Header------------------%
\def\course{CS 410/510 Introduction to Quantum Computing}
\def\term{Portland State U, Spring 2017}
\def\prof{Lecturer: Fang Song}
\newcommand{\handout}[5]{
   \renewcommand{\thepage}{\arabic{page}}
   \begin{center}
   \framebox{
      \vbox{
    \hbox to 5.78in { \hfill \large{\course} \hfill }
       \vspace{2mm}
       \hbox to 6in { {\Large \hfill #5  \hfill} }
       \vspace{2mm}
       \hbox to 6in { \term \hfill \emph{#2}}
       \hbox to 6in { {#3 \hfill \emph{#4}}}
      }
   }
   \end{center}
   \vspace*{4mm}
}
\newcommand{\lecture}[4]{\handout{#1}{#2}{#3}{#4}{{Lecture #1}}}

\def\C{\mathbb{C}}
\def\I{\mathbb{1}}
\newcommand{\vecket}[2]{\begin{pmatrix} #1 \\ #2 \end{pmatrix}}
\newcommand{\vecbra}[2]{\begin{pmatrix}\bar{#1}&\bar{#2}\end{pmatrix}}
\newcommand{\ket}[1]{| #1 \rangle}
\newcommand{\bra}[1]{\langle #1 |}
\newcommand{\inner}[2]{\langle #1 | #2 \rangle}
\newcommand{\makeroom}{\vspace{5mm}\noindent}
\newcommand{\overwire}[1]{\text{\raisebox{5mm}{$^{#1}$}}}

\newtheorem{theorem}{Theorem}

%=============Main Doc=================%
\begin{document}
%-----Specs: change accordingly--------%
\def\lecdate{May 9, 2017} % put lecture date here
\def\scribe{Scribe: Ben Hamlin} % put your name here
\def\lecnum{11} % change the lecture number

\lecture{\lecnum}{\lecdate}{\prof}{\scribe}%

\begin{center}
{\textsc{Version: \today}}  
\end{center}

%======================================%
\section{Theory of Quantum Information}
%======================================%

A.K.A. Quantum information processing

\begin{itemize}
\item \textit{What it's not}: Quantum computation
\begin{itemize}
\item Algorithms (e.g., search, order-finding)
\item Complexity (e.g., QBP, QMA)
\item In general: \textit{Making quantum states meaningful.}
\end{itemize}
\item \textit{What it is}: Quantum information
\begin{itemize}
\item More fundamental tasks (e.g., create, copy, store, communicate, \dots)
\item Focus on theory (experiments also exist)
\item In general: \textit{Making quantum states availiable.}
\end{itemize}
\end{itemize}

%==============================%
\section{Information theory}
%==============================%

\subsection{Classical setting} 
Claude Shannon, ``A Mathematical Theory of Communication,'' 1948.
\[\Qcircuit @C=2mm @R=3mm {
   \lstick{\text{Alice}}          &&          \rstick{\text{Bob}}
\\ \lstick{m} & \gate{\text{Comm. Channel}} & \rstick{\hat{m}}\qw
\\            & \text{phone line, cable, etc.}
}\]
\textit{Central Questions:}
\begin{enumerate}\setcounter{enumi}{-1}
\item What is information?
\item How many bits do we need to describe a source? (Shannon entropy)
      [Shannon's Source Coding Theorem]
\item How can we transmit messages over a noisy channel?
      [Shannon's Noisy-channel Coding Theorem]
\end{enumerate}

\makeroom
\subsection{Quantum setting}
\[\Qcircuit @C=2mm @R=2mm {
   \lstick{\text{Alice}} && \rstick{\text{Bob}}
\\ \lstick{\ket{m}} & \gate{\text{Quantum Channel}} & \rstick{\ket{\hat m}}\qw
\\ \lstick{\text{Q. data}} & \text{fiber}
}\]

\makeroom
\textit{Central questions:}
\[\begin{tabular}{l|l|l}
    & classical channel     & quantum channel
    \\ \hline         &                       &
    \\ classical data & 1) Source Coding Thm. & 1) How much classical info is in a Q. state [Holevo]
    \\                & 2) Noisy Channel Thm. & 2) Classical capacity of Q. channel (QECC)
    \\ \hline         &                       &
    \\ quantum data   &                     & 1) Min. qubits to describe Q. source [Schumacher]
    \\                &                       & 2) Q. capacity of Q. channel (QECC)
\end{tabular}\]

\makeroom
\textbf{New resource: Entanglement}
\begin{itemize}
\item Teleportation: Transmit quantum states over a classical using two bits
\item Superdense coding: Transmit two classical bits using a single qubit
\item Non-local games
\end{itemize}

\makeroom
\textbf{New tasks:}
\begin{itemize}
\item Getting around the No-cloning Theorem
\item Distinguishing quantum states
\end{itemize}

%==============================%
\section{No-cloning Theorem}
%==============================%

\textbf{Classically...}
\[\Qcircuit @C=2mm @R=2mm {
   \lstick{a \in \{0,1\}} & \ctrl{1} & \qw & \rstick{a}
\\ \lstick{0}             & \targ    & \qw & \rstick{a}
}\]
The above classical circuit ``clones'' the bit $a$ reversibly. I.e., it
calculates $\rangle a,0 \langle \mapsto \rangle a,a \langle$.

\makeroom
\textbf{Quantumly...} \\
Is there a quantum circuit $U$ that calculates
$\ket{a}\ket0 \xmapsto{U} \ket{a}\ket{a}$? \textit{No-cloning Theorem}

\makeroom
\textit{Why does CNOT fail?} \\
Succeeds in some cases, e.g., $\ket0, \ket1$:
\[\Qcircuit @C=2mm @R=3mm {
   \lstick{\ket0} & \ctrl{1} & \qw & \rstick{\ket0}
\\ \lstick{\ket0} & \targ    & \qw & \rstick{\ket0}
} \hspace{1in} \Qcircuit @C=2mm @R=2mm {
   \lstick{\ket1} & \ctrl{1} & \qw & \rstick{\ket1}
\\ \lstick{\ket0} & \targ    & \qw & \rstick{\ket1}
}\]
Fails in others, e.g., $\ket+$:
\[\Qcircuit @C=2mm @R=3mm {
   \lstick{\ket0} & \ctrl{2} & \qw 
\\ & & & \rstick{\frac{\ket{00} + \ket{11}}{\sqrt{2}} \not= \ket+\ket+}
\\ \lstick{\ket0} & \targ    & \qw \gategroup{1}{3}{3}{3}{3mm}{\}}
}\]

\makeroom
\begin{proof}
Suppose for contradiction that there exists a unitary $U$ and a fixed
preparation $\ket{s}$ s.t. for all states $\ket\psi$,
\begin{align*}
\ket\psi \ket{s} \xmapsto{U} \ket\psi \ket\psi
\end{align*}
Then for any distinct quantum states $\ket\psi$ and $\ket{\psi'}$,
\begin{align*}
   \ket\psi    \ket{s} \xmapsto{U} \ket\psi    \ket\psi
\\ \ket{\psi'} \ket{s} \xmapsto{U} \ket{\psi'} \ket{\psi'}
\end{align*}
And since unitary operations preserve inner product,
\begin{align}
   \inner{\psi s}{\psi' s} &= \inner{\psi\psi}{\psi'\psi'}
\\ (\bra\psi \otimes \bra{s})(\ket{\psi'} \otimes \ket{s})
        &= (\bra\psi \otimes \bra\psi)(\ket{\psi'} \otimes \ket{\psi'})
\\ \inner{\psi}{\psi'} \otimes \inner{s}{s}
        &= \inner{\psi}{\psi'} \otimes \inner{\psi}{\psi'}
\\ \inner{\psi}{\psi'} &= \inner{s}{s} \label{contra}
\end{align}
And since $\inner{s}{s}$ is a constant, line \ref{contra} clearly cannot be true
for all distinct $\ket\psi$, $\ket{\psi'}$. Therefore by contradiction, no such
$U$ can exist.
\end{proof}

%======================================%
\section{The Power of Entanglement}
%======================================%

\textbf{Examples of entangled states:}
\begin{itemize}
\item 2-qubit
\[\text{Bell States: } \left\{
\begin{array}{l}
        \left.\begin{array}{l}
                \ket{\Phi^+} = \frac{1}{\sqrt{2}}(\ket{00} + \ket{11})
        \\      \ket{\Phi^-} = \frac{1}{\sqrt{2}}(\ket{00} - \ket{11})
        \end{array} \right\}\text{EPR Pairs}
\\      \left.\begin{array}{l}
                \ket{\Psi^+} = \frac{1}{\sqrt{2}}(\ket{01} + \ket{10})
        \\      \ket{\Psi^-} = \frac{1}{\sqrt{2}}(\ket{01} - \ket{10})
        \end{array}\right.
\end{array}\right.\]
\item 3-qubit
\[\text{GHZ State: } \frac{1}{\sqrt{2}}(\ket{000} + \ket{111})\]
\end{itemize}

\makeroom
\textbf{Teleportation:}\\
Suppose Alice wants to send $\ket\psi = \alpha\ket0 + \beta\ket1$ to Bob.
\begin{itemize}
\item Alice may not know $\alpha$ and $\beta$.
\item Even if she does, she still cant send $\alpha,\beta \in \C$ with infinite
precision.
\end{itemize}

\paragraph{Claim:} If Alice and Bob share an EPR pair, Alice can send 2
classical bits to Bob s.t. Bob can reproduce $\ket\psi$ exactly
[Bennet,~Brassard,~et~al.~'93].

\makeroom
\textbf{Teleportation circuit:}\\
\[\Qcircuit @C=2mm @R=3mm {
   \lstick{\text{Alice}}                    &&   (1)      &&& (2)      & (3)      & (4)    &              &             && \rstick{\text{Bob}}
\\ \lstick{\ket\psi}                        &&&& \lstick{M} & \ctrl{1} & \gate{H} & \meter & \cw          & \control\cw  & \rstick{a}\cw
\\                                          &&&& \lstick{A} & \targ    & \qw      & \meter & \control\cw  & \cw\cwx      & \rstick{b}\cw
\\ \lstick{\raisebox{9mm}{$\ket{\Phi^+}$}}  &&&& \lstick{B} & \qw      & \qw      & \qw    & \gate{X}\cwx & \gate{Z}\cwx & \rstick{\ket{\psi}}\qw
\gategroup{3}{2}{4}{2}{3mm}{\{}
}\]

\begin{enumerate}
\item $\frac{1}{\sqrt{2}}(\alpha(\ket{000} + \ket{011})
        + \beta(\ket{100} + \ket{111}))$
\item $\frac{1}{\sqrt{2}}(\alpha(\ket{000} + \ket{011})
        + \beta(\ket{110} + \ket{101}))$
\item $\frac{1}{2}(\ket{00}(\alpha\ket0 + \beta\ket1)
                   \ket{10}(\alpha\ket0 - \beta\ket1)
                   \ket{01}(\alpha\ket1 + \beta\ket0)
                   \ket{11}(\alpha\ket1 - \beta\ket0))$
\item
\begin{tabular}{c|c|cll}
MA & w.p.          & B                        &                &            \\ \hline
00 & $\frac{1}{4}$ & $\alpha\ket0+\beta\ket1$ & $\xmapsto{\I}$ & $\ket\psi$ \\ \hline
01 & $\frac{1}{4}$ & $\alpha\ket1+\beta\ket0$ & $\xmapsto{X}$  & $\ket\psi$ \\ \hline
10 & $\frac{1}{4}$ & $\alpha\ket0-\beta\ket1$ & $\xmapsto{Z}$  & $\ket\psi$ \\ \hline
11 & $\frac{1}{4}$ & $\alpha\ket1-\beta\ket0$ & $\xmapsto{ZX}$ & $\ket\psi$
\end{tabular}
\end{enumerate}
\end{document}
