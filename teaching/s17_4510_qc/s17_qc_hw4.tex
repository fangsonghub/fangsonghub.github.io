% Created by Fang Song on April 4 2017. Instructions: you don't need
% to change anything in the macros, but feel free to define new
% commands as you wish. Starting from the main body, change the specs
% (e.g., your name). use \begin{solution} \end{solution} environment
% to write your solutions. Don't forget to list your collaborators.

\documentclass[12pt,answers]{exam}
%============Macros==================%
\usepackage{amsmath,amsfonts,amssymb,amsthm}
\usepackage{qcircuit}
\usepackage[margin=1in]{geometry}
%--------------Cosmetic----------------%
\usepackage{mathtools}
\usepackage{hyperref}
\usepackage{fullpage}
\usepackage{microtype}
\usepackage{xspace}
\usepackage[svgnames]{xcolor}
\usepackage[sc]{mathpazo}
\usepackage{soul} %strike
\usepackage{enumitem}
\setlist[enumerate]{itemsep=1pt,topsep=2pt}
\setlist[itemize]{itemsep=1pt,topsep=2pt}
%----------Header--------------------%
\def\course{CS 410/510 Introduction to Quantum Computing}
\def\term{Portland State U, Spring 2017}
\def\prof{Lecturer: Fang Song}
\newcommand{\handout}[5]{
   \renewcommand{\thepage}{\arabic{page}}
   \begin{center}
   \framebox{
      \vbox{
    \hbox to 5.78in { \hfill \large{\course} \hfill }
       \vspace{2mm}
       \hbox to 5.78in { {\Large \hfill #5  \hfill} }
       \vspace{2mm}
       \hbox to 5.78in { \term \hfill \emph{#2}}
       \hbox to 5.78in { {#3 \hfill \emph{#4}}}
      }
   }
   \end{center}
   \vspace*{4mm}
}
\newcommand{\hw}[4]{\handout{#1}{#2}{#3}{#4}{Homework #1}}

%-----defs and commands-----%
\def\mG{[\textbf{G}]\xspace}
\def\veps{\varepsilon}
\def\complex{\mathbb{C}}
\def\integer{\mathbb{Z}}
\newcommand{\bra}[1]{\langle #1 \rvert}
\newcommand{\ket}[1]{\lvert #1 \rangle}
\newcommand{\kera}[1]{\ket{#1}\bra{#1}}
\newcommand{\bret}[2]{\langle{#1}|{#2}\rangle}
\newcommand{\corr}[1]{{\color{blue}{#1}}}
\newcommand{\bool}{\ensuremath{\{0,1\}}}

%=======Main document==============%
\begin{document}

%----Specs: change accordingly-----%
\newif\ifstudent % comment out false
%\studenttrue 
\studentfalse

\def\hwnum{4} %number
\def\issuedate{May 16, 2017}
\def\duedate{May 30, 2017} % 
\def\yourname{your name} % put your name here
%------------------------------%
\ifstudent
\hw{\hwnum}{\issuedate}{Student: \yourname}{Due: \duedate}%
\else
\hw{\hwnum}{\issuedate}{\prof}{Due: \duedate}%
\fi
\noindent \textbf{Instructions.}
%Solutions must be typeset in \LaTeX\ (a template for this homework is available on the course web page).
Your solutions will be graded on \emph{correctness} and
\emph{clarity}. You should only submit work that you believe to be
correct; if you cannot solve a problem completely, you will get
significantly more partial credit if you clearly identify the gap(s)
in your solution. It is good practice to start any long solution with
an informal (but accurate) ``proof summary'' that describes the main
idea. For this problem set, a random subset of problems will be
graded. Problems marked with ``\mG'' are required for graduate
students. Undergraduate students will get bonus points for solving
them. 

\medskip
\noindent You may collaborate with others on this problem set% and consult external sources
.  However, you must \textbf{\emph{write up your own solutions}} and
\textbf{\emph{list your collaborators}} for each problem.

\begin{questions}
  \question[15] (OR gate as a quantum operation) Recall the binary OR
  operation, denoted as $\vee$, defined as $a \vee b = 0$ if
  $a = b = 0$ and $a\vee b = 1$ otherwise. Here we consider operations
  that map the two-qubit state $\ket{a, b}$ to the one-qubit state
  $\ket{a\vee b}$, for all $a,b \in \{0,1\}$. Of course, no unitary
  operation can perform this mapping, since the input and output
  dimension do not match; however, general quantum operations can
  compute this mapping.

  \begin{parts}
    \part Give a sequence of $2\times 4$ matrices $A_1,\ldots, A_k$
    with $\sum_{j = 1}^k A_j\dagger A_j = I$ that compute the OR
    operation in the sense that, for all $a, b\in\{0,1\}$, when
    $\rho = \kera{a,b}$,
    $\sum_{j = 1}^k A_j\rho A_j\dagger = \kera{a\vee b}$.
    
    \part The operation from part (a) maps all basis states to pure
    states. Does it map all pure input states to pure output states?
    Either prove it, or provide a counterexample.
  \end{parts}
  
  % \question[15] (Simulating POVM) We've mentioned in class that any
  % admissible quantum operation may be simulated by unitary
  % operations and ordinary projective measurement under standard
  % basis. In this problem, you will prove a special case, the POVM
  % measurement. Recall in a POVM, we are only interested in the
  % probabilities of observing each outcome, and don’t care about how
  % the state collapses afterward. Let's consider a POVM measurement
  % specified by positive semi-definite operators $E_1, \ldots, E_m$
  % on $n$ qubits satisfying $\sum_jE_j = I$. Let $m = 2^k$ for some
  % integer $k$. Show how to simulate it by applying a unitary
  % operation on a lager quantum system and then measuring under
  % standard basis. (Hint: consider $m 2^n \times 2^n}$ matrix $V$
  % formed by vertically stacking $E_j$. Show the columns are
  % orthogonal, and extend $V$ to a unitary of dimension $m2^n$.)
  
  \question (Quantum error-correcting) 
  \begin{parts}
    \part[15] Let $E$ be an arbitrary 1-qubit unitary, and $I,X,Y,Z$
    are the four $2\times 2$ Pauli matrices.
    \begin{enumerate}[label=\roman*)]
    \item Show that it can be written as
      $E = \alpha_0 I + \alpha_1X + \alpha_2 Y + \alpha_3 Z$, for some
      complex coefficients $\alpha_i$ with
      $\sum_{i=0}^3 |\alpha_i|^2 = 1$. (Hint: compute the trace
      $Tr(E^\dagger E)$ in two ways, and use the fact that
      $Tr(AB) = 0$ if $A$ and $B$ are distinct Pauli matricies, and
      $Tr(AB) = Tr(I) = 2$ if $A$ and $B$ are the same Pauli.)
    \item Write the 1-qubit Hadamard transform H as a linear combination
      of the four Pauli matrices.
    \item Suppose an $H$-error happens on the first qubit of
      $\alpha\ket{\bar 0} + \beta\ket{1}$ using the 9-qubit code.
      Give the various steps in the error-correction procedure that
      corrects this error.
    \end{enumerate}
    \part[10] Show that there cannot be a quantum code that encodes
    one logical qubit by $2k$ physical qubits while being able to
    correct errors on up to $k$ of the qubits. (Hint: No-cloning
    theorem)
  \end{parts}
  % \question (Algorithms)
  % \begin{parts}
  %   \part search / decision factoring
  % \end{parts}

  \question (Learning parities) Let $s\in \bool^n$ be a secret $n$-bit
  string. Suppose $f: \bool^n \to \bool$ computes the dot product
  $f(x) = s \cdot x = \sum_{i = 1}^{n} s_i x_i \pmod 2$ (i.e., the
  parity of the bits in $s$ chosen by the non-zero positions of
  $x$). In this problem, we will (mainly) consider the query
  complexity of learning $s$.
  %by algorithms with zero-error, meaning that they always output the
  %correct answer.
  \begin{parts}
    \part[8] How many queries are needed to classically learn $s$ with
    zero-error (i.e., always outputting the correct answer)? Give an
    algorithm for this problem, and show that it is optimal.
    \part[7] Explain why even if we allow the classical algorithm to
    fail with some fixed probability (e.g., probability $1/3$), it
    requires the same asymptotic query complexity as the zero-error
    case.
    \part[10] How many queries are needed by a \emph{quantum} algorithm to
    learn $s$ with zero-error? Give an algorithm for this problem, and
    show that it is optimal. As usual, we assume a quantum oracle
    $O_f$: $\ket{x}\ket{y}\mapsto \ket{x}\ket{x\cdot s \pmod 2}$ is
    given. (Hint: Deutsch-Josza)
    \part (Bonus 10pts) Now consider a \emph{noisy} version $\tilde f$
    of $f$: $\tilde f(x) = x\cdot s + e_x \pmod 2$ where
    $e_x \in \bool$ is a random bit independently drawn for each $x$,
    and $b_x = 1$ with probability $\eta$. Given oracle access to
    $\tilde f$, how many queries are needed by a quantum algorithm for
    finding $s$ with probability at least $\Omega((1-2\eta)^2)$?
    \part (Bonus 15pts) Suppose that we no longer have oracle access
    to $f$. Instead we are given a sequence of classical samples
    $(x_i, y_i), i=1,\ldots, m$, where $x_i\gets \bool^n$ chosen
    uniformly at random and $y_i = s \cdot x_i + e_{x_i} \pmod 2$ with
    independent $e_{x_i}\gets \text{COIN}_\eta$. Let $m$ be a
    polynomial in $n$. Give a (quantum or classical) algorithm that
    runs in time polynomial in $n$ for finding $s$ for constant $\eta$
    (e.g., $\eta = 1/4$).
  \end{parts}
  \question (Testing entanglement) Suppose that Alice and Bob share a
  two-qubit state, and they want to test if it is the EPR pair
  $\ket{\phi^+} := \frac{1}{\sqrt 2}(\ket{00}+\ket{11})$ with local
  measurements and classical communication. Consider the following
  procedure: they randomly select a measurement basis: with
  probability $1/2$, they both measure in the standard basis
  $\{\ket{0},\ket{1}\}$; and, with probability $1/2$, they both
  measure in the Hadamard (diagonal) basis $\{\ket{+},\ket{-}\}$. Then
  they perform the measurement and they accept if and only if their
  outcomes are the same.
  \begin{parts}
    \part[5] Show that the state $\phi^+$ is always accepted by this
    test with zero-error.
    \part[8] Show that, for an arbitrary 2-qubit state $\ket{\mu}$, the
    probability that it passes the test is at most
    \begin{equation*}
      \frac{1 + |\bret{\mu}{\phi^+}|^2}{2}   \, .
    \end{equation*}
    (Hint: decomposing $\ket{\mu}$ under the four Bell states.)
    \part[7] Now consider another (malicious) party Eve, who may have
    intervened with the state that Alice shares with Bob. Let
    $\rho_{ABE}$ be their joint state. Now assume that Alice and Bob
    are certain that they two perfectly share $\ket{\phi^+}$, show
    that Alice and Eve's state cannot be in $\ket{\phi^+}$ as well.

    Note: we can actually show that $\rho_{ABE}$ must be of form
    $\kera{\phi^+}_{AB}\otimes \rho_E$ (i.e., Eve's state is
    uncorrelated with that of Alice and Bob). This is an example of
    \emph{monogamy of entanglement}: the more system $A$ is entangled
    with $B$, the less $A$ is entangled with another system $C$.
  \end{parts}
  \question (Quantum rewinding) Let $Q$ be a unitary quantum circuit
  that takes an $m$-qubit input state $\ket{\psi}$ and ancilla 
  $\ket{0^k}$, and let the output state be 
  \begin{equation*}
    Q \ket{\psi}\ket{0^k} = \sqrt{p(\psi)} \ket{0}\ket{\phi_0(\psi)}
    + \sqrt{1-p(\psi)}\ket{\phi_1(\psi)} \, .
  \end{equation*}
  Namely if we measure the first of the $m+k$ output qubits, we see
  $0$ (1 respectively) with probability $p(\psi)$ ($1-p(\psi)$ resp.)
  and the state collapses to $\ket{0}\ket{\phi_0(\psi)}$
  ($\ket{1}\ket{\phi_1(\psi)}$ resp.) Suppose that we would like to
  produce $\ket{\phi_0(\psi)}$ from input state $\ket{\psi}$. If we
  have multiple copies of $\ket{\psi}$, we may repeat running $Q$ and
  hope to measure $0$ in one of the instances. This problem explores
  when we can do so with just a single copy of $\ket{\psi}$.
  \begin{parts}
    \part[5] Give an example of $Q$ and $\psi$ such that if we run $Q$
    and unluckily get measurement outcome $1$, it becomes impossible
    to produce $\ket{\phi_0(\psi)}$.
    \part[5] Consider a special case of $Q$ where $p(\psi) = p$ for
    all $\ket{\psi}$ (i.e., $p(\psi)$ is independent of
    $\ket{\psi}$). Give such an example of $Q$. 
    \part (Do not need to turn it in) Assume that the condition of
    part (b) holds and for  is constant over all
    choices of the $\ket{\psi}$ with $p\in (0,1)$). Show that for
    every $\veps > 0$ there is a general quantum circuit $R$, with
    $\text{size}(R) = O(\log(1/\veps)\cdot \text{size}(Q))$ such that
    the output $\rho(\psi)$ of $R$ satisfies
    \begin{equation*}     
      \langle \phi_0(\psi)|\rho(\psi)|\phi_0(\psi)\rangle \geq 1 -
      \veps \, .
    \end{equation*}
    (Hint: techniques in Grover's algorithm.)  Note: we may weaken the
    condition slightly to \emph{almost constant} probabilities of
    measuring $0$ and $1$ and show a similar ``quantum rewinding''
    procedure. Read more on
    \url{https://cs.uwaterloo.ca/~watrous/Papers/ZeroKnowledgeAgainstQuantum.pdf}.
    \part (Bonus 10pts) Under the same condition above, show how to
    recover $\ket{\psi}$ exactly from $\ket{\phi_1(\psi)}$. (Therefore
    you may repeat running $Q$ from start again and again utill you
    get $\ket{\phi_0(\psi)}$)
    \part (Bonus 15pts) Can you identify other conditions where
    quantum rewinding is possible?
  \end{parts}
\end{questions}


\end{document}
