\documentclass[11pt]{article}
%============Macros==================%
% you don't need to change anything. start editing from main body.
\usepackage{amsmath,amsfonts,amssymb,amsthm}
\usepackage{qcircuit}
\usepackage[margin=1in]{geometry}
%--------------Cosmetic----------------%
\usepackage{mathtools}
\usepackage{hyperref}
\usepackage{fullpage}
\usepackage{microtype}
\usepackage{xspace}
\usepackage[svgnames]{xcolor}
\usepackage[sc]{mathpazo}
\usepackage{enumitem}
\usepackage{braket}
\setlist[enumerate]{itemsep=1pt,topsep=2pt}
\setlist[itemize]{itemsep=1pt,topsep=2pt}
%--------------Header------------------%
\def\course{CS 410/510 Introduction to Quantum Computing}
\def\term{Portland State U, Spring 2017}
\def\prof{Lecturer: Fang Song}
\newcommand{\handout}[5]{
   \renewcommand{\thepage}{\arabic{page}}
   \begin{center}
   \framebox{
      \vbox{
    \hbox to 5.78in { \hfill \large{\course} \hfill }
       \vspace{2mm}
       \hbox to 6in { {\Large \hfill #5  \hfill} }
       \vspace{2mm}
       \hbox to 6in { \term \hfill \emph{#2}}
       \hbox to 6in { {#3 \hfill \emph{#4}}}
      }
   }
   \end{center}
   \vspace*{4mm}
}
\newcommand{\lecture}[4]{\handout{#1}{#2}{#3}{#4}{{Lecture #1}}}

\newcommand\fang[1]{{\color{red}[Fang: #1]}}
%=============Main Doc=================%
\begin{document}
%-----Specs: change accordingly--------%
\def\lecdate{April 20, 2017} % put lecture date here
\def\scribe{Scribe: John Donahue} % put your name here
\def\lecnum{6} % change the lecture number

\lecture{\lecnum}{\lecdate}{\prof}{\scribe}%

\begin{center}
{\textsc{Version: \today}}  
\end{center}

%==============================%
\section{Phase Estimation}
%==============================%
Remember, for this problem, our input is a quantum circuit for a unitary U, as well as the eigenvector $\ket{\psi}$. $U\ket{\psi}=e^{2\pi i \theta}\ket{\psi}$. Our goal is to approximate $\theta$.

\[
\Qcircuit @C=1em @R=1em {
& \lstick{\ket{0}}    & \qw & \gate{H} & \qw & \ctrl{1} & \qw & \multigate{3}{\ \mathcal{F}\ } & \qw & \meter \\
& \lstick{\ket{0}}    & \qw & \gate{H} & \qw & \ctrl{2} & \qw & \ghost{\ \mathcal{F}\ } & \qw & \meter \\
& ...\\
& \lstick{\ket{0}}    & \qw & \gate{H} & \qw & \ctrl{1} & \qw & \ghost{\ \mathcal{F}\ } & \qw & \meter \\
& \lstick{\ket{\psi}} & \qw & \qw      & \qw & \gate{U} & \qw 
}
\]

%==============================%
\subsection{Special Case}
%==============================%

First, lets consider the special case $\theta =
\frac{j}{{2^m}}$. Remember, the state going into $F$ is
\[  \frac{1}{\sqrt[]{2^m}}\sum_{x \in \{0,1\}^m} e^{2 \pi
    i \theta x} \ket{x} \]

Let $\omega := e^{\frac{2\pi i}{2^m}}$, rewrite it as
$ \frac{1}{\sqrt[]{2^m}}\sum_{x \in \{0,1\}^m} \omega^{j x} \ket{x} =
: \ket{\phi_j}$. Therefore, computing $j$ amounts to distinguishing
each $\ket{\phi_j}$. 

Now, consider $\{ \ket{\phi_j} : j \in \{ 0,1 \}^m
\}$. $\braket{\phi_j | \phi_{j'}'}$ will equal 1 if $j = j'$, and 0 if
otherwise. This forma an orthonormal basis for $m$-qubit states in
$\mathbb{C}$. There is also a standard basis
$\{ \ket{j} : j \in \{ 0,1 \}^m \}$. We know that there exists a
unitary F such that $\ket{\phi_j} \mapsto^F \ket{j}$. So, if we apply
$F$ to $\ket{\phi_j}$, we will see $j$.

But what is $F$? If $j = 0$, that means $F$ is the following
$2^m \times 2^m$ matrix.

\begin{align*}
F = 
\begin{bmatrix}
1 & 1 & \dots & 1 & \dots & 1 \\
1 & \omega & \dots & \omega^j & \dots & \dots \\
\dots & \dots & \dots & \dots & \dots & \dots & \\
1 & \omega^{2^m - 1} & \dots & \omega^{j(2^m - 1)} & \dots & \omega^{(2^m - 1)(2^m - 1)}
\end{bmatrix}
\end{align*}

This is the discrete Fourier Transform.

%==============================%
\subsection{General Case}
%==============================%

Now, lets look at the general case, where $\theta \in [0,1)$. We will see the following.

\begin{align*}
F\ket{x} = \frac{1}{\sqrt[]{2^m}}\sum_{y \in \{0,1\}^m} \omega^{-y x} \ket{y}
\end{align*}
\begin{align*}
F\ket{\phi} = \frac{1}{\sqrt[]{2^m}}\sum_{x \in \{0,1\}^m} \omega^{j x} F\ket{x}
\end{align*}
\begin{align*}
F\ket{\phi} = \frac{1}{2^m}\sum_{x \in \{0,1\}^m} \omega^{j x} \sum_{y \in \{0,1\}^m} \omega^{-y x} \ket{y}
\end{align*}
\begin{align*}
F\ket{\phi} = \frac{1}{2^m}\sum_{y \in \{0,1\}^m} (\sum_{x \in \{0,1\}^m} e^{2 \pi i x (\theta - \frac{y}{2^m})} )\ket{y}
\end{align*}

For measuring, we will see this.

\begin{align*}
Pr[y] = \left|\frac{1}{2^m}\sum_{x \in \{0,1\}^m} e^{2 \pi i x (\theta - \frac{y}{2^m})}\right|^2
\end{align*}

Note that this looks exactly like a geometric series. That means we can rewrite it without the $\sum$.

\begin{align*}
Pr[y] = \frac{1}{2^{2m}}|\frac{e^{2 \pi i (M\theta - y)} - 1}{e^{2 \pi i (\theta - \frac{y}{M})} - 1}|^2
\end{align*}

\fang{add the details of analysis}

This means $\theta = \frac{j}{M} + \epsilon$, $|\epsilon| \leq \frac{1}{2^{m+1}}$. Essentially, this ends up showing $Pr[j] \geq \frac{4}{\pi^2} \geq 0.4$. The larger Pr[j] becomes, the more precise our approximation of $\theta$ will be. In short, the chance that we will have m bits corrext at least 40\% of the time.

%==============================%
\section{Quantum Fourier Transform}
%==============================%

\fang{More to add here}

The following shows a quantum circuit for computing Fourier Transform.

\begin{align*}
R_k = 
\begin{bmatrix}
1 & 0 \\
0 & e^{\frac{2 \pi i}{2^k}}
\end{bmatrix}
\end{align*}

\[
\Qcircuit @C=1em @R=1em {
& \lstick{\ket{x_{m-1}}} & \qw & \gate{H} & \gate{R_2} & ... & \qw & \gate{R_{m-1}} & \gate{R_{m}} & \qw & \qw & \qw & \qw \\
& \lstick{\ket{x_{m-2}}} & \qw & \qw & \ctrl{-1} & ... \\
& ... \\
& \lstick{\ket{x_1}} & \qw & \qw & \qw & \qw & \qw & \ctrl{-3} & \qw & \gate{H} & \gate{R_2} & \qw & \qw \\
& \lstick{\ket{x_0}} & \qw & \qw & \qw & \qw & \qw & \qw & \ctrl{-4} & \qw & \ctrl{-1} & \gate{H} & \qw \\
}
\]

Lastly, the outputs are all reversed such that $\ket{x_0}$ is on the top.

\end{document}