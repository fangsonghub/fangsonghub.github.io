% Created by Fang Song on April 4 2017. Instructions: you don't need
% to change anything in the macros, but feel free to define new
% commands as you wish. Starting from the main body, change the specs
% (e.g., your name). use \begin{solution} \end{solution} environment
% to write your solutions. Don't forget to list your collaborators.

\documentclass[12pt,answers]{exam}
%============Macros==================%
\usepackage{amsmath,amsfonts,amssymb,amsthm}
\usepackage{qcircuit}
\usepackage[margin=1in]{geometry}
%--------------Cosmetic----------------%
\usepackage{mathtools}
\usepackage{hyperref}
\usepackage{fullpage}
\usepackage{microtype}
\usepackage{xspace}
\usepackage[svgnames]{xcolor}
\usepackage[sc]{mathpazo}
\usepackage{soul} %strike
\usepackage{enumitem}
\setlist[enumerate]{itemsep=1pt,topsep=2pt}
\setlist[itemize]{itemsep=1pt,topsep=2pt}
%----------Header--------------------%
\def\course{CS 410/510 Introduction to Quantum Computing}
\def\term{Portland State U, Spring 2017}
\def\prof{Lecturer: Fang Song}
\newcommand{\handout}[5]{
   \renewcommand{\thepage}{\arabic{page}}
   \begin{center}
   \framebox{
      \vbox{
    \hbox to 5.78in { \hfill \large{\course} \hfill }
       \vspace{2mm}
       \hbox to 5.78in { {\Large \hfill #5  \hfill} }
       \vspace{2mm}
       \hbox to 5.78in { \term \hfill \emph{#2}}
       \hbox to 5.78in { {#3 \hfill \emph{#4}}}
      }
   }
   \end{center}
   \vspace*{4mm}
}
\newcommand{\hw}[4]{\handout{#1}{#2}{#3}{#4}{Homework #1}}

%-----defs and commands-----%
\def\mG{[\textbf{G}]\xspace}
\def\veps{\varepsilon}
\def\complex{\mathbb{C}}
\newcommand{\bra}[1]{\langle #1 \rvert}
\newcommand{\ket}[1]{\lvert #1 \rangle}
\newcommand{\kera}[1]{\ket{#1}\bra{#1}}
\newcommand{\bret}[2]{\langle{#1}|{#2}\rangle}
\newcommand{\corr}[1]{{\color{blue}{#1}}}
\newcommand{\bool}{\ensuremath{\{0,1\}}}
%=======Main document==============%
\begin{document}

%----Specs: change accordingly-----%
\newif\ifstudent % comment out false
%\studenttrue 
\studentfalse

\def\hwnum{3} %number
\def\issuedate{May 2, 2017, Update: May 07}
\def\duedate{May 16, 2017} % 
\def\yourname{your name} % put your name here
%------------------------------%
\ifstudent
\hw{\hwnum}{\issuedate}{Student: \yourname}{Due: \duedate}%
\else
\hw{\hwnum}{\issuedate}{\prof}{Due: \duedate}%
\fi
\noindent \textbf{Instructions.}
%Solutions must be typeset in \LaTeX\ (a template for this homework is available on the course web page).
Your solutions will be graded on \emph{correctness} and
\emph{clarity}. You should only submit work that you believe to be
correct; if you cannot solve a problem completely, you will get
significantly more partial credit if you clearly identify the gap(s)
in your solution. It is good practice to start any long solution with
an informal (but accurate) ``proof summary'' that describes the main
idea. For this problem set, a random subset of problems will be
graded. Problems marked with ``\mG'' are required for graduate
students. Undergraduate students will get bonus points for solving them.

\medskip
\noindent You may collaborate with others on this problem set% and consult external sources
.  However, you must \textbf{\emph{write up your own solutions}} and
\textbf{\emph{list your collaborators}} for each problem.

\begin{questions}
  \question (Shor's algorithm) In this problem, we study further
  properties of quantum Fourier transform, and analyze Shor's
  algorithm for order finding that we've seen briefly in class. Recall
  the Quantum Fourier Transform $QFT_{\mathbb{Z}_M}$ for some integer
  $M$:
    \begin{equation*}
      \ket{\alpha} = \sum_{x = 0}^{M-1} \alpha_x \ket{x}
      \quad\stackrel{QFT_{\mathbb{Z}_N}}{\longmapsto} \quad \ket{\hat
        \alpha}=\sum_{y=0}^{M-1} \hat \alpha_y \ket{y} 
    \end{equation*}
    where $\hat \alpha_y = \frac{1}{\sqrt M}\sum_{x} \omega_M^{xy}\corr{\cdot\alpha_x}$,
    \st{${\omega_N}$} $\corr{\omega_M} = e^{2\pi i /M}$, for each $y \in [M]$.

\begin{parts}
  \part[5] Index shift. Let $j \in \mathbb{Z}_M$, and define
  $\ket{\alpha_{+j}}:= \sum_x \alpha_x \ket{x + j \bmod M}$. Compute
  $\ket{\hat \alpha_{+j}}:=QFT_{\mathbb{Z}_M} \ket{\alpha_{+j}} =?
  $. Justify from your result that measuring $\ket{\hat \alpha}$ and
  $\ket{\hat \alpha_{+j}}$ produce the same probability distribution.
  \part[8] Periodic superposition. Consider an integer $r$ such that
  $M= \ell \cdot r$ for some $\ell \in \mathbb{Z}$. Define
  $\ket{P_r}:= \frac{1}{\sqrt{\ell}}\sum_{k = 0}^{\ell - 1} \ket{kr} =
  \frac{1}{\sqrt{\ell}}(\ket{0} + \ket{r} + \ldots + \ket{(\ell - 1)
    r})$. Show that applying $QFT_{\mathbb{Z}_M}$ on $\ket{P_r}$ gives
  $\frac{1}{\sqrt{r}}\sum_{k=0}^{r-1}\ket{k\ell} = \frac{1}{\sqrt
    r}(\ket{0} + \ket{\ell} + \ldots + \ket{(r-1)\ell})$.
  \part[12] We can use the properties above to analyze Shor's order
  finding algorithm as illustrated in the circuit below. Our goal is
  to find the order of $a$ mod a positive integer $N$, namely the
  smallest $r$ such that $a^r=1\bmod N$.
  $E_a: \ket{x}\ket{y}\mapsto\ket{x}\ket{y+a^x\bmod N}$ is the unitary
  circuit implementing the modular exponentiation function
  $x\in\mathbb{Z}_{2^m} \mapsto a^x \bmod N$. 
  \begin{figure}[ht]
    \centerline{ \Qcircuit @C=2em @R=1.5em { \lstick{\ket{0^m}} &
        \gate{H^{\otimes m}} & \multigate{1}{E_a} &
        \gate{QFT_{\mathbb{Z}_{2^m}}} &  \meter & \qw \quad {s}  \\
        \lstick{\ket{0^n}}& \qw &\ghost{E_a} & \qw & \qw & \qw } }
    \caption{Shor's algorithm}
  \end{figure}

  \begin{enumerate}[label=\roman*)]
  \item What is the quantum state right after applying $E_a$? Since
    the bottom register will never be used, we may assume that it gets
    measured immediately. What do you see as the measurement outcome?
    Suppose that the outcome is $z$, what is the state on the top
    register then?
  \item Then QFT is applied on top followed by a measurement, we call
    the outcome a sample $s$. Suppose that in this simple case, we are
    able to pick $m$ such that $r | 2^m$. Show how to use multiple
    samples to recover the order of $a$ with high probability.
  \end{enumerate}
  \part (Exercise. You do not need to turn it in.) In general, we do
  not know an $m$ such that $r|2^m$ to run Shor's algorithm. Describe
  how to compute the order $\text{ORD}_N(a)$ then. Note: the core
  technique in Shor's algorithm is usually referred to as
  \emph{Fourier sampling} as it produces random samples from the
  Fourier transform of the function. Conceptually, you may view order
  finding as a hidden subgroup problem on $\mathbb{Z}$ given the
  function $x\in \mathbb{Z} \mapsto a^x \bmod N$, and Shor's algorithm
  uses $QFT_{2^m}$ in order to \emph{approximately} Fourier sample
  over $\mathbb{Z}$.
\end{parts}
\question (Grover's algorithm) This problem explores extensions of
Grover's search algorithm. Consider as usual a function
$f:\{0,1\}^n \to \{0,1\}$. Define
$A = f^{-1}(1):= \{ x\in \bool^n: f(x) = 1\}$ and
$B = f^{-1}(0) := \{x\in \bool^n: f(x) =0\}$. Let $a = |A|$ be the
number of ``marked'' items and $b = N-a$ where $N=2^n$. We further
define $\ket{A}: = \frac{1}{\sqrt a} \sum_{x\in A} \ket{x}$ and
$\ket{B}:= \frac{1}{\sqrt b} \sum_{x\in B} \ket{x}$. 
\begin{parts}
  \part[5] Run Grover's algorithm with $f$ on input
  $\ket{0^n}$. Namely we apply unitary $G := -HZ_0HZ_f$ repetitively
  on $H\ket{0^n}$ where $Z_0: \ket{x}\mapsto - \ket{x}$ iff. $x = 0^n$
  and $Z_f: \ket{x}\mapsto (-1)^{f(x)}\ket{x}$. Denote
  $\ket{\psi^0} = H\ket{0^n}$ and $\ket{\psi^t}:= G^t\ket{\psi^0}$ for
  $t\geq 1$.  Show that every $\ket{\psi^t}$ can be written as
  \st{$\cos$} $\corr{\sin}(\theta_t)\ket{A} + $ \st{$\sin$}
  $\corr{\cos}(\theta_t)\ket{B}$. How does $\theta_t$ change to
  $\theta_{t+1}$? (Hint: geometric interpretation of Grover's
  algorithm)

  \part[5] Show that if we pick $t$ to be the nearest integer to
  $\frac 1 2 (\frac{\pi}{2\theta_0} - 1)$,
  $|\bret{\psi^t}{A}|^2 \geq 1/2$. This implies that measuring
  $\ket{\psi^t}$ will output an element $x$ of $A$ with probability at
  least $1/2$. What is the distribution of $x$, conditioned on being
  in $A$?

  \part[5] Assuming $a$ is known to us. How to find an element of $A$
  with high probability using $O(\sqrt{N/a})$ queries to the oracle?
  \part[7] What if $a$ is unknown? Show that if one picks
  $t\in \{1,\ldots, \sqrt N + 1\}$ at random and run Grover's
  algorithm, it succeeds in finding a marked element with probability
  at least $1/4$. 
  \part[10] Show how to find a marked item with high probability
  within $O(\sqrt{N/a})$ queries. (Hint: pick random
  $t \in \{1,\ldots T\}$ with a small $T$ in the beginning and try. If
  fail increment $T$ slowly but exponentially.)
  \part[10] \mG One natural application of Grover's algorithm is
  finding the minimum element in a list of $N$ numbers. Give a
  $O(\sqrt N)$ quantum algorithm. (Hint: can you pick a \emph{random}
  pivot element every time that is smaller than the current pivot
  element? Part b) may be helpful.)
 
\end{parts}
\question (Mixed states and density matrix) 

\begin{parts}
  \part[5] A density matrix ρ corresponds to a pure state if and only if
  $\rho = \kera{\psi}$. Show that $\rho$ corresponds to a pure state
  if and only if $Tr(\rho^2) = 1$.
  \part[5] Show that every $2\times 2$ density matrix $\rho$ can be
  expressed as an equally weighted mixture of pure states. That is
  $\rho = \frac 1 2 \kera{\psi_1} + \frac 1 2 \kera{\psi_2}$ (Note:
  the two states do not need to be orthogonal).
  % \part \mG Show that, for any matrix $\rho$ that is Hermitian,
  % positive definite (i.e., has no negative eigenvalues) with trace 1,
  % there is a probabilistic mixture of pure states whose denisty matrix
  % is $\rho$.
  \part[5] Imagine two parties Alice and Bob. Alice flips a biased coin
  which is HEADS with probability $\cos^2(\pi/8)$. Alice prepares
  $\ket{0}$ when she sees coin $0$ and $\ket{1}$ otherwise. From
  Alice’s perspective (who knows the coin value), the density matrix
  of the state she created will be either $\kera{0}$ or
  $\kera{1}$. She then sends the qubit to Bob. What is the density
  matrix of the state from Bob’s perspective (who does not know the
  coin value)? Write down the matrix.
\end{parts}

\question[8] (Teleporting part of an entangled state.) Recall that, in the
teleportation protocol, Alice and Bob initially have a joint state of
the form
$(\alpha\ket{0}_A + \beta\ket{1}_A) \frac{1}{\sqrt 2}
(\ket{00}_{AB}+\ket{11}_{AB})$. (Subscripts indicate who possesses
each qubit). At the end of the protocol, there remains only Bob’s
qubit, and it is in state $\alpha\ket{0}_B + \beta\ket{1}_B$. Now we
introduce a third party, Carol. Suppose that Alice possesses a qubit
that she want to teleport to Bob, and this qubit is entangled with
Carol’s qubit, in state
$\frac{1}{\sqrt 2} \ket{00}_{CA} + \ket{11}_{CA}$. Set the initial
state to

\begin{equation*}
  \frac{1}{\sqrt 2} (\ket{00}_{CA} + \ket{11}_{CA}) \otimes
  \frac{1}{\sqrt 2}( \ket{00}_{AB} + \ket{11}_{AB}) \, .
\end{equation*}

Perform the teleportation protocol on Alice’s and Bob’s qubits. At the
end of the protocol, there will remain two qubits: Carol’s and
Bob’s. What is the state of Carol and Bob’s qubits? Justify your
answer by a clear proof.
\end{questions}


\end{document}
