% Created by Fang Song on April 4 2017. Instructions: you don't need
% to change anything in the macros, but feel free to define new
% commands as you wish. Starting from the main body, change the specs
% (e.g., your name). use \begin{solution} \end{solution} environment
% to write your solutions. Don't forget to list your collaborators.

\documentclass[12pt,answers]{exam}
%============Macros==================%
\usepackage{amsmath,amsfonts,amssymb,amsthm}
\usepackage{qcircuit}
\usepackage[margin=1in]{geometry}
%--------------Cosmetic----------------%
\usepackage{mathtools}
\usepackage{hyperref}
\usepackage{fullpage}
\usepackage{microtype}
\usepackage{xspace}
\usepackage[svgnames]{xcolor}
\usepackage[sc]{mathpazo}
\usepackage{enumitem}
\setlist[enumerate]{itemsep=1pt,topsep=2pt}
\setlist[itemize]{itemsep=1pt,topsep=2pt}
%----------Header--------------------%
\def\course{CS 410/510 Introduction to Quantum Computing}
\def\term{Portland State U, Spring 2017}
\def\prof{Lecturer: Fang Song}
\newcommand{\handout}[5]{
   \renewcommand{\thepage}{\arabic{page}}
   \begin{center}
   \framebox{
      \vbox{
    \hbox to 5.78in { \hfill \large{\course} \hfill }
       \vspace{2mm}
       \hbox to 5.78in { {\Large \hfill #5  \hfill} }
       \vspace{2mm}
       \hbox to 5.78in { \term \hfill \emph{#2}}
       \hbox to 5.78in { {#3 \hfill \emph{#4}}}
      }
   }
   \end{center}
   \vspace*{4mm}
}
\newcommand{\hw}[4]{\handout{#1}{#2}{#3}{#4}{Homework #1}}

%-----defs and commands-----%
\def\mG{[\textbf{G}]\xspace}
\def\veps{\varepsilon}
\newcommand{\bra}[1]{\langle #1 \rvert}
\newcommand{\ket}[1]{\lvert #1 \rangle}
\newcommand{\kera}[1]{\ket{#1}\bra{#1}}

%=======Main document==============%
\begin{document}

%----Specs: change accordingly-----%
\newif\ifstudent % comment out false
\studenttrue 
%\studentfalse

\def\issuedate{April 3, 2017}
\def\duedate{April 18, 2017} % 
\def\yourname{your name} % put your name here
%------------------------------%
\ifstudent
\hw{1}{\issuedate}{Student: \yourname}{Due: \duedate}%
\else
\hw{1}{Out: \issuedate}{\prof}{Due: \duedate}%
\fi
\noindent \textbf{Instructions.}
%Solutions must be typeset in \LaTeX\ (a template for this homework is available on the course web page).
Your solutions will be graded on \emph{correctness} and
\emph{clarity}. You should only submit work that you believe to be
correct; if you cannot solve a problem completely, you will get
significantly more partial credit if you clearly identify the gap(s)
in your solution. It is good practice to start any long solution with
an informal (but accurate) ``proof summary'' that describes the main
idea. For this problem set, a random subset of problems will be
graded. Problems marked with ``\mG'' are required for graduate
students. Undergraduate students will get bonus points for solving them.

\medskip
\noindent You may collaborate with others on this problem set% and consult external sources
.  However, you must \textbf{\emph{write up your own solutions}} and
\textbf{\emph{list your collaborators}} for each problem.

\def\coin{\mathrm{COIN}}
\def\fail{\mathrm{FAIL}}
\begin{questions}
  \question (Tensor product) Recall the \emph{tensor product} of two
  matrices $A$ and $B$ is $A\otimes B : = (a_{ij}B)$.
  %\begin{solution}
   % uncomment the environment and write your solution here
  %\end{solution}
  \begin{parts}
    \part[5] Show that $(A\otimes B)^\dagger = A^\dagger \otimes
    B^\dagger$.
    \part[5] Show that if $U$ and $V$ are unitary matrices, then so is $U
    \otimes V$. 
  \end{parts}

\question (Quantum states and gates)   
\begin{parts}
  \part[12] In each case, describe the resulting state.
  $H = \frac{1}{\sqrt 2} \left(
    \begin{array}{lr}
      1 & 1\\
      1 & -1
    \end{array} \right)$
  \begin{enumerate}[label=\roman*)]
  \item Apply $H$ to the qubit $\frac{1}{\sqrt 2} (\ket{0} +
    \ket{1})$.
    \item Apply $H$ to the first qubit of state $\frac{1}{\sqrt 2}
      (\ket{00} + \ket{11})$.
    \item Apply $H$ to both qubits of $\frac{1}{\sqrt 2}
      (\ket{00} + \ket{11})$.
    \item Apply $\frac{1}{\sqrt 2} \left(
    \begin{array}{lr}
      1 & i\\
      i & 1
    \end{array} \right)$ to both qubits of state $\frac{1}{\sqrt 2}
      (\ket{00} + \ket{11})$.
    \end{enumerate}
    \part[10] (1-qubit gates) Let $X = \frac{1}{\sqrt 2} \left(
    \begin{array}{lr}
      0 & 1\\
      1 & 0
    \end{array} \right)$ and $Z = \frac{1}{\sqrt 2} \left(
    \begin{array}{lr}
      1 & 0\\
      0 & -1
    \end{array} \right)$.
  \begin{enumerate}[label=\roman*)]
    \item Suppose we have a qubit and we first apply $X$ and then
      $Z$. Is it equivalent to first applying $Z$ and then $X$?
    \item Suppose we have two qubits. We apply $X$ to both and then
      $Z$ to both. Is it equivalent to applying $Z$ to both and then
      applying $X$ to both? Determine your answer by explicitly
      computing $X\otimes X$, $Z\otimes Z$, and their products both
      ways. 
    \end{enumerate}
    \part[8] (SWAP gate) A SWAP gate takes two inputs $a$ and $b$ and
    outputs $b$ and $a$; i.e., it swaps the values of two input
    registers. Show how to build a SWAP gate using only CNOT gates.
    (Hint: you’ll need 3 of them.)
  \end{parts}

  
  \question (Product states versus entangled states) In each of the
  following, either express the 2-qubit state as a tensor product of
  1-qubit states or prove that it cannot be expressed this way. 
  \begin{parts}
    \part[5] $\frac 1 2 \ket{00} - \frac 1 2 \ket{01} - \frac 1 2
    \ket{10} + \frac 1 2 \ket{11}$ 
    \part[5] $\frac 1 2 \ket{00} + \frac 1 2 \ket{01} +\frac1 2
    \ket{10} - \frac{1}{2} \ket{11}$
    \part[5] \mG $\frac{3}{4}\ket{00} + \frac{\sqrt
      3}{4}\ket{01}+\frac{\sqrt 3}{4}\ket{10}+\frac{1}{4}\ket{11}$
  \end{parts}

  \question (Distinguishing states by local measurements) In this
  question, we suppose Alice and Bob are physically separated from
  each other, and are each given one of the qubits of some 2-qubit
  state. Working as a team, they are required to distinguish between
  State I and State II with only local measurements. Namely they can
  each perform a local (one-qubit) unitary operation and then a
  measurement (in the computational basis) of their own qubit. After
  their measurements, they can send only classical bits to each
  other. (This is usually referred to as LOCC: local operation and
  classical communication.)  In each case below, either give a perfect
  distinguishing procedure (that never errs) or explain why there is
  no perfect distinguishing procedure (i.e., that for any procedure
  the success probability must be less than 1).

  \begin{parts}
    \part[5] State I: $\frac{1}{\sqrt 2}(\ket{00} + \ket{11})$; State II: $\frac{1}{\sqrt 2}(\ket{01} + \ket{10})$ 
    \part[5] State I: $\frac{1}{\sqrt 2} (\ket{00} + \ket{11})$; State
    II: $\frac{1}{\sqrt 2} (\ket{00}-\ket{11})$
    \part[5] \mG State I: $\frac{1}{\sqrt 2}(\ket{00} + i \ket{11})$;
    State II: $\frac{1}{\sqrt 2}(\ket{00}-i\ket{11})$
  \end{parts}

  
  \question (Simulating a biased coin) For $0 \leq p \leq 1$, let
  $\coin_p$ denote a gate that has no input and one output, the output
  being a random bit which is 1 with probability $p$ and 0 with
  probability $1- p$. This question compairs the power of general
  $\coin_p$ for an arbitrary rational $p$ and the special case of
  fair coin gate $\coin_{1/2}$.
  \begin{parts}
    \part[10] In one sense, general $\coin_p$ gates are more powerful than
    $\coin_{1/2}$ gates. Show that if we only allow $\coin_{1/2}$
    gates (as well as AND, OR, NOT, etc.), it is impossible to
    construct a circuit that exactly simulates $\coin_{1/3}$.

    \part[10] However, in another sense, $\coin_p$ gates are not
    fundamentally more powerful than $\coin_{1/2}$. Show that for any
    $\veps > 0$, there is a circuit of $\coin_{1/2}$ (and AND, OR, NOT
    etc.) of size $O(\log(1/\veps)$ that almost exactly simulates a
    $\coin_{1/3}$ gate. Precisely, your circuit should have two output
    bits, called $r$ and $\fail$. The output bit $\fail$ should be 1
    with probability at most $\veps$ And the output bit $r$ should
    have the property that $\Pr[r = 1 | \fail \neq 1] = 1/3$ exactly.

    (Note: once you've figured how to do it for $1/3$, I believe
    you'll be able to do so for any rational value $p$.)

  \end{parts}

\end{questions}


\end{document}
