% Created by Fang Song on April 4 2017. Instructions: you don't need
% to change anything in the macros, but feel free to define new
% commands as you wish. Starting from the main body, change the specs
% (e.g., your name). use \begin{solution} \end{solution} environment
% to write your solutions. Don't forget to list your collaborators.

\documentclass[12pt,answers]{exam}
%============Macros==================%
\usepackage{amsmath,amsfonts,amssymb,amsthm}
\usepackage{qcircuit}
\usepackage[margin=1in]{geometry}
%--------------Cosmetic----------------%
\usepackage{mathtools}
\usepackage{hyperref}
\usepackage{fullpage}
\usepackage{microtype}
\usepackage{xspace}
\usepackage[svgnames]{xcolor}
\usepackage[sc]{mathpazo}
\usepackage{enumitem}
\setlist[enumerate]{itemsep=1pt,topsep=2pt}
\setlist[itemize]{itemsep=1pt,topsep=2pt}
%----------Header--------------------%
\def\course{CS 410/510 Introduction to Quantum Computing}
\def\term{Portland State U, Spring 2017}
\def\prof{Lecturer: Fang Song}
\newcommand{\handout}[5]{
   \renewcommand{\thepage}{\arabic{page}}
   \begin{center}
   \framebox{
      \vbox{
    \hbox to 5.78in { \hfill \large{\course} \hfill }
       \vspace{2mm}
       \hbox to 5.78in { {\Large \hfill #5  \hfill} }
       \vspace{2mm}
       \hbox to 5.78in { \term \hfill \emph{#2}}
       \hbox to 5.78in { {#3 \hfill \emph{#4}}}
      }
   }
   \end{center}
   \vspace*{4mm}
}
\newcommand{\hw}[4]{\handout{#1}{#2}{#3}{#4}{Homework #1}}

%-----defs and commands-----%
\def\mG{[\textbf{G}]\xspace}
\def\veps{\varepsilon}
\def\complex{\mathbb{C}}
\newcommand{\bra}[1]{\langle #1 \rvert}
\newcommand{\ket}[1]{\lvert #1 \rangle}
\newcommand{\kera}[1]{\ket{#1}\bra{#1}}

%=======Main document==============%
\begin{document}

%----Specs: change accordingly-----%
\newif\ifstudent % comment out false
\studenttrue 
%\studentfalse

\def\issuedate{April 18, 2017}
\def\duedate{May 02, 2017} % 
\def\yourname{your name} % put your name here
%------------------------------%
\ifstudent
\hw{2}{\issuedate}{Student: \yourname}{Due: \duedate}%
\else
\hw{2}{\issuedate}{\prof}{Due: \duedate}%
\fi
\noindent \textbf{Instructions.}
%Solutions must be typeset in \LaTeX\ (a template for this homework is available on the course web page).
Your solutions will be graded on \emph{correctness} and
\emph{clarity}. You should only submit work that you believe to be
correct; if you cannot solve a problem completely, you will get
significantly more partial credit if you clearly identify the gap(s)
in your solution. It is good practice to start any long solution with
an informal (but accurate) ``proof summary'' that describes the main
idea. For this problem set, a random subset of problems will be
graded. Problems marked with ``\mG'' are required for graduate
students. Undergraduate students will get bonus points for solving them.

\medskip
\noindent You may collaborate with others on this problem set% and consult external sources
.  However, you must \textbf{\emph{write up your own solutions}} and
\textbf{\emph{list your collaborators}} for each problem.

\begin{questions}
  \question (Norms) For a vector
  $v = (v_0, \ldots, v_{N-1})\in \complex^N$, let
  $\|v\|:=\sqrt{\sum_{i=0}^{N-1} |v_i|^2}$, which is the usual
  Euclidean length of $v$. For any $N \times N$ matrix
  $M\in \complex^{n\times n}$, define its \emph{spectral norm}
  $\|M \|$ as $\|M\| = \max_{\ket{\psi}} \| M \ket{\psi}\|$, where the
  maximum is taken over quantum states (i.e., vectors $\ket{\psi}$
  such that $\| \ket{\psi}\| = 1$).  Define the distance between two
  $N \times N$ unitary matrices $U_1$ and $U_2$ as $\|U_1 - U_2\|$.

  \begin{parts}
    \part[5] Show that $\|A - B\| \leq \|A - C\| + \| C - B \|$, for any
    three $N \times N$ matrices A, B, and C. (Thus, this distance
    measure satisfies the \emph{triangle inequality}.)
    \part[5] Show that for any $A$ and identity matrix $I$,
    $\|A \otimes I\| = \|A\|$.
    \part[5] Show that, for any two $N\times N$ unitary matrices $U_1$
    and $U_2$, and any matrix $A$, $\|U_1AU_2\| = \|A\|$.
  \end{parts}
  Note: more generally we can define $p$-norms for $v\in \complex^N$
  as $\|v\|_p:= \left(\sum_i |v_i|^p\right)^{1/p}$ for
  $1\leq p < \infty$ and $\|v\|_\infty:= \max_i \{|v_i|\}$. The
  Euclidean distance is then the special case $\|\cdot\|_2$. These
  vector norms give rise to \emph{induced norms} on matrices
  $M \in \complex^{N\times N}$ by
  $\|M\|_{p}:=\sup\{\|Mv\|_{p}: v\in \complex^N,
  \|v\|_p=1\}$. Therefore the spectral norm is the induced Euclidean
  ($p=2$) norm.
  
  \question (Quantum Fourier Transform)
  \begin{parts}
    

  \part[12] Let $F_N$ denote the $N$-dimensional Fourier transform
  \begin{equation*}
    F_N := \frac{1}{\sqrt N}\left(
    \begin{array}{ccccc}
      1 & 1 & 1 & \cdots & 1\\
      1 &\omega_N & \omega_N^2&\cdots& \omega_N^{N-1} \\
      1 &\omega^2_N & \omega_N^4&\cdots& \omega_N^{2(N-1)}\\
      \vdots &\vdots&\vdots&\vdots&\vdots\\
      1 &\omega_N^{N-1} & \omega_N^{2(N-1)}&\cdots& \omega_N^{(N-1)^2}\\
    \end{array} \right),  \text{ where } \omega_N := e^{2\pi i / N} (i =\sqrt {−1})
  \end{equation*}

  (an $N\times N$ matrix, with entry
  $\frac{1}{\sqrt{N}} e^{(2\pi i /N)jk}$ position $j,k$ for
  $j, k \in \{0, 1, \ldots, N-1\}$.
  \begin{enumerate}[label=\roman*)]
  \item Show that all rows in $F_N$ are vectors of length 1, and any
    two rows are orthogonal.
  \item What is $F_N^2$? (Hint: The matrix has a very simple form.)
  \item What is the minimum $j$ such that $F_N^j = I$ is the identity?
    
  \end{enumerate}

  \part[5] In class, we computed the QFT modulo $N=2^n$ by a quantum
  circuit of size $O(n^2)$.  Recall that it uses gates of the form
    \begin{equation*}
      R_k = \left(
    \begin{array}{lllc}
      1 & 0 &0 &0\\
      0 & 1 &0 &0 \\
      0 & 0 & 1& 0 \\
      0 & 0 & 0 & e^{2\pi i /{2^k}} 
    \end{array} \right)   
    \end{equation*}
    for $k\in \{2,\ldots,n\}$. Show that $\|R_k - I\| \leq 2\pi/{2^k}$
    , where I is the $4 \times 4$ identity matrix. (Thus, $R_k$ gets
    very close to $I$ when $k$ increases.)
    
    \part[8] Here we compute an \emph{approximation} of this QFT
    within $\veps$ by a quantum circuit of size $O(n
    \log(n/\veps)$. The idea to start with the $O(n^2)$ circuit and
    then remove some of its $R_k$ gates (it is equivalent to changing
    the $R_k$ gate to identity gate). Removing an $R_k$ gate makes the
    circuit smaller but also changes the unitary transformation, but
    if $k$ is large then from above we can deduce that removing a
    $P_k$ gate changes the unitary transformation by only a small
    amount. Show how to use this approach to obtain a quantum circuit
    of size $O(n\log (n/\veps)$ that computes a unitary transformation
    $\tilde F_N$ such that $\| \tilde F_N − F_N \| \leq \veps$.
    (Hint: Try removing all $R_k$ gates where $k\geq t$, for some
    carefully chosen threshold $t$. The properties of our distance
    measure from the previous question should be useful for your
    analysis here.)
\end{parts}

\question (Square root of a quantum operation) Let $U$ be a unitary
quantum circuit on $n$ qubits. In this problem, we want to construct
another circuit that computes a square root of $U$ (i.e., a unitary
$V$ such that $V^2 = U$).

\begin{parts}
  


\part[5] Let $X = \left(
    \begin{array}{lr}
      0 & 1\\
      1 & 0
    \end{array} \right)$ be the matrix for a 1-qubit NOT gate. Find a $2\times 2$ matrix $V$ (i.e.,
  1-qubit gate) such that $V^2 = X$.

  \part[5] Suppose we construct $V$ by simply taking the square root
  of each gate $U_i$ in circuit $U$, does this work, i.e. is
  $V^2 = U$? Justify your answer.

  \part[20] We explore a strategy of implementing $V$ using the
  \emph{phase estimation} algorithm. Suppose $U$ is constituted by $s$
  two-qubit gates. We study a simple case here. Let
  $\{\ket{\psi_x}: x ∈ \{0,\ldots, 2^n - 1\}\}$ be a set of
  orthonormal eigenvectors of $U$ with eigenvalues in
  $\{\pm 1, \pm i\}$. Namely
  $U \ket{\psi_x} = i^{\phi_x} \ket{\psi_x}$ with
  $\phi_x \in \{0,1,2,3\}$. We outline a construction of $V$ as
  follows such that $V\ket{\psi_x} = \omega^{\phi_x} \ket{\psi_x}$
  where $\omega= e^{2\pi i /8}$:
  \begin{itemize}
  \item Construct a generalized-control-$U$, with two control-qubits,
    i.e.,
    $\ket{a b}\otimes \ket{c} \mapsto \ket{ab}\otimes U^{ab}{\ket{c}}$
    (applying $U$ iff. both $a=1$ and $b=1$).
  \item Then apply the phase-estimation algorithm to this
    controlled-$U$ gate, which results in a circuit that computes
    $ab = \phi_x$, in two ancillary qubits for any input
    $\ket{\psi_x}$.
  \item Apply gates to those two ancilliary qubits to induce the
    mapping $\ket{ab} \mapsto \omega^{ab}\ket{ab}$.
  \item Then apply the inverse of the phase-estimation circuit.
\end{itemize}
 Answer the following questions:
 \begin{enumerate}[label=\roman*)]
 \item Explain how to construct a circuit computing the two-qubit
   controlled-$U$ operation using $3s$ 3-qubit gates.
 \item Explain how to construct a circuit computing $ab = \phi_x$ on
   input $\ket{\phi_x}$ using $3s$ 3-qubit gates, one 2-qubit gate,
   and four 1-qubit gates.
 \item Give a quantum circuit consisting of two 1-qubit gates that
   maps each basis state $\ket{ab}$ to $\omega^{ab}\ket{ab}$.
 \item Verify that the construction $V$ is correct, i.e.,
   $V\ket{\psi_x} = \omega^{\phi_x} \ket{\psi_x}$. Explain why this
   implies $V^2 = U$, namely $V$ computes the squre root of $U$ on any
   input state $\ket{\psi}$.

 \end{enumerate}
 Note: the total gate cost is $6s$ 3-qubit gates plus two 2-qubit
 gates plus eight 1-qubit gates. This can be converted into a circuit
 consisting of $O(s)$ 2-qubit gates, not much more than the original
 circuit.
\end{parts}  
\end{questions}


\end{document}
