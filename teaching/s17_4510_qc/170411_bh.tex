\documentclass[11pt]{article}
%============Macros==================%
% you don't need to change anything. start editing from main body.
\usepackage{amsmath,amsfonts,amssymb,amsthm}
\usepackage{qcircuit}
\usepackage[margin=1in]{geometry}
%--------------Cosmetic----------------%
\usepackage{mathtools}
\usepackage{hyperref}
\usepackage{fullpage}
\usepackage{microtype}
\usepackage{xspace}
\usepackage[svgnames]{xcolor}
\usepackage[sc]{mathpazo}
\usepackage{enumitem}
\setlist[enumerate]{itemsep=1pt,topsep=2pt}
\setlist[itemize]{itemsep=1pt,topsep=2pt}
%--------------Header------------------%
\def\course{CS 410/510 Introduction to Quantum Computing}
\def\term{Portland State U, Spring 2017}
\def\prof{Lecturer: Fang Song}
\newcommand{\handout}[5]{
   \renewcommand{\thepage}{\arabic{page}}
   \begin{center}
   \framebox{
      \vbox{
    \hbox to 5.78in { \hfill \large{\course} \hfill }
       \vspace{2mm}
       \hbox to 6in { {\Large \hfill #5  \hfill} }
       \vspace{2mm}
       \hbox to 6in { \term \hfill \emph{#2}}
       \hbox to 6in { {#3 \hfill \emph{#4}}}
      }
   }
   \end{center}
   \vspace*{4mm}
}
\newcommand{\lecture}[4]{\handout{#1}{#2}{#3}{#4}{{Lecture #1}}}

\newcommand{\vecket}[2]{\begin{pmatrix}#1\\#2\end{pmatrix}}
\newcommand{\rttwoinv}{\frac{1}{\sqrt{2}}}
\newcommand{\ket}[1]{|#1\rangle}
\newcommand{\poly}{\text{poly}}
\newcommand{\makeroom}{\vspace{5mm}\noindent}
\newcommand{\overwire}[1]{\text{\raisebox{5mm}{$^{#1}$}}}

\newtheorem{theorem}{Theorem}

%=============Main Doc=================%
\begin{document}
%-----Specs: change accordingly--------%
\def\lecdate{April 11, 2017} % put lecture date here
\def\scribe{Scribe: Ben Hamlin} % put your name here
\def\lecnum{3} % change the lecture number

\lecture{\lecnum}{\lecdate}{\prof}{\scribe}%

\begin{center}
{\textsc{Version: \today}}  
\end{center}

%==============================%
\section{Measurement}
%==============================%

What is it doing?

\makeroom
Measurement in the standard, or ``computational'' basis:
\begin{align*}
\ket{\Phi}
        = \alpha\ket{0} + \beta\ket{1}
        = \vecket{\alpha}{\beta}
        \in \mathbb{C}^2 && \text{where } |\alpha|^2 + |\beta|^2 = 1
\end{align*}
\begin{itemize}
\item $\alpha$ and $\beta$ are ``amplitudes''.
\item Projects $\vecket{\alpha}{\beta}$ onto one of $\vecket{0}{1}$ or
      $\vecket{1}{0}$.
\item Probability is magnitude: $|\alpha|^2$ or $|\beta|^2$.
\end{itemize}

\makeroom
We can also measure in other orthonormal bases, e.g., the diagonal basis:
\[\{\ket{+},\ket{-}\} = \left\{\rttwoinv\vecket{1}{1},
  \rttwoinv\vecket{1}{-1} \right\}\]

%==============================%
\section{A General Quantum Circuit}
%==============================%

\[\Qcircuit @C=1em @R=1em {
\ket{\Phi}      &&& {/}\qw & \multigate{1}{\mathcal{U}_f} & {/}\qw & \meter \\
\ket{0 \dots 0} &&& {/}\qw &        \ghost{\mathcal{U}_f} & {/}\qw & {|}\qw
}\]
\begin{itemize}
\item $\ket{\Phi}$ is an $n$-qubit register.
\item The lower register are $\poly(n)$ scrap---or ``ancillary''---qubits.
\item We measure $m$ qubits at the end and discard the rest.
\end{itemize}

\makeroom
Note that we only measure at the end. Is this too restrictive? No

\makeroom
\textbf{Principle of Deferred Measurement:}
\begin{theorem}
\textit{Informally:} A quantum circuit with intermediate measurement can be
simulated by a quantum circuit thet only measures at the end with linear
overhead.
\end{theorem}

How? For any intermediate measurement on register $A$, replace it by
introducing an ancillary register $B$ and apply CNOT gate with $A$
being the control and $B$ as the target. $A$ goes through whatever
operation that comes next and $B$ is left untouched till the endo of
computation at which point it gets measured (i.e. discarded). The
actual output registers will have the same distribution as the
original circuit. Clearly the \emph{overhead} of the transformation is
only linear in the size of the original circuit. 

%==============================%
\section{Reversible Computation}
%==============================%

Is a quantum circuit at least as powerful as a classical circuit? Yes.
\begin{itemize}
\item Since quantum gates are unitary matrices, they are reversible (bijective).
\item Classical gates like \framebox{$\land$} are \textit{not} reversible.
\item We can simulate classical gates reversibly with extra bits and Toffoli
      gate:
\end{itemize}

\begin{align*}
\text{\framebox{$\land$}} \-\ &\equiv \-\ \-\ \Qcircuit @C=1em @R=1em {
a && \ctrl{2} & \qw & \qw & a \\
b && \ctrl{1} & \qw & \qw & b \\
0 && \targ    & \qw &     & a \land b
} \\ \\
\text{\framebox{$\lor$}} \-\ &\equiv \-\ \-\ \Qcircuit @C=1em @R=1em {
a && \gate{\lnot} & \ctrl{2} & \qw          & \qw & \qw & a \\
b && \gate{\lnot} & \ctrl{1} & \qw          & \qw & \qw & b \\
0 && \qw          & \targ    & \gate{\lnot} & \qw &     & a \lor b
} \\ \\
\text{\framebox{FANOUT}} \-\ &\equiv \-\ \-\ \Qcircuit @C=1em @R=1em {
a && \qw          & \ctrl{2} & \qw    & \qw & a \\
0 && \gate{\lnot} & \ctrl{1} & {|}\qw &     &   \\
0 && \qw          & \targ    & \qw    & \qw & a
}
\end{align*}

\begin{theorem}\label{thm:rev}
\textit{Informally:} A classical circuit $C_f$ implementing an arbitrary
function $f : \{0,1\}^n \to \{0,1\}^m$ using \framebox{$\land$},
\framebox{$\lor$}, and $\framebox{$\lnot$}$ can be simulated by a reversible
circuit $RC_f$ using $\poly(n)$ \framebox{$\lnot$} and Toffoli gates. Such a
reversible circuit will have an additional $\ell = \poly(n)$ junk input bits and
an additional $n+\ell-m$ output junk bits.
\end{theorem}

\makeroom
We can clean up the junk bits from Theorem \ref{thm:rev}. Given $RC_f$, we
construct $RC_f^{-1}$ by flipping the order of application. Then we construct
$U_f : (x,0^m) \mapsto (x,f(x))$ by composing the two:

\[\Qcircuit @C=1em @R=1em {
x \in \{0,1\}^n &&&& {/}\qw & \multigate{1}{RC_f} & {/}\qw & \ctrl{2} & \qw
        & f(x) & & {/}\qw & \multigate{1}{RC_f^{-1}} & {/}\qw & \qw & x \\
0^{\ell} && \qw & \qw & {/}\qw & \ghost{RC_f} & {/}\qw & \qw & \qw
        & \qw & \qw & {/}\qw & \ghost{RC_f^{-1}} & {/}\qw & \qw & 0^{\ell} \\
0^{m} && \qw & \qw & {/}\qw & \qw & \qw & \targ & \qw
        & \qw & \qw & \qw & \qw & {/}\qw & \qw & f(x)
}\]

%==============================%
\section{The Power of Quantum Circuits}
%==============================%

$U_f$ can be implemented as a quantum circuit, since it is unitary. And since
the above construction is polynomial in time and space complexity,
\[P \subseteq BQP\]

\makeroom
And since a quantum circuit has randomness (via applying $H$ and measuring),
\[BPP \subseteq BQP\]

\makeroom
Can quantum algorithms do better than their classical counterparts? Consider the
following toy example:

\begin{align*}
&\Qcircuit @C=1em @R=1em {
\ket{0} && \qw & \gate{H} & {\overwire{\ket{+}}}\qw & \gate{H} & \meter & \qw & 0
        &&& (1) \\
\ket{0} && \qw & \gate{H} & \meter                  & \gate{H} & \meter & \qw & \phi
        &&& (2)
} \\
&\phi = \begin{cases}
                0 & \text{ w.p. } \frac{1}{2} \\
                1 & \text{ w.p. } \frac{1}{2}
        \end{cases}
\end{align*}

\makeroom
In (1), defering measurement allow amplitudes to interfere, eliminating the
possibility of evaluating to 1, whereas in (2), measuring after each gate limits
us to classical probabilities.

\makeroom
Quantum speedup uses interference to
\begin{itemize}
\item reinforce the amplitudes of outcomes we want, and
\item cancel out the amplitudes of ``undesired'' outcomes.
\end{itemize}

%==============================%
\section{Query Model}
%==============================%

\textit{Given:}
\begin{itemize}
\item An oracle $O_f : (\ket{x}\otimes \ket{y}) \mapsto (\ket{x}\otimes \ket{f(x) \oplus y})$
      for the function $f(\cdot)$.
\item $O_f$ is a quantum circuit that can be queried in superposition.
\end{itemize}

\makeroom
\textit{Goal:}
\begin{itemize}
\item Compute some information about $f(\cdot)$ by querying $O_f$.
\item Complexity calculated in number of queries.
\item Ideally pre-/post-processing is also time-efficient (polynomial), but
      that's not the emphasis.
\end{itemize}

\makeroom
\textit{Why do we study this model?}
\begin{itemize}
\item Simple and easy to analyze.
\item Captures the essence of QC and provides insight.
\item Despite its generality, captures concrete problems, such as factoring.
\end{itemize}

%==============================%
\section{Deutsch's Problem and Algorithm}
%==============================%

\textit{Given:}
\begin{itemize}
\item A function $f : \{0,1\} \to \{0,1\}$.
\item Classically, 2 queries are necessary and sufficient.
\end{itemize}

\makeroom
\textit{Goal:} Decide whether $f(\cdot)$ is constant ($f(0) = f(1)$) or balanced
($f(0) \not= f(1)$).

\makeroom \emph{Classical algorithms}: no matter deterministic or
randomized, it is easy to verify that 2 queries are both sufficient
and necessary to solve this problems. However, there is a
\emph{quantum} algorithm that needs only $\mathbf{1}$ query.

\makeroom \textit{Quantum Algorithm:}
\[\Qcircuit @C=1em @R=1em {
\ket{0} && \gate{H} & \multigate{1}{O_f} & \gate{H} & \meter & \qw &&& z \in \{0,1\} \\
\ket{1} && \gate{H} &        \ghost{O_f} & {|}\qw   &        &     &&&
}\]
\begin{itemize}
\item $f(\cdot)$ is balanced iff $z = 1$.
\item 1 query is sufficient.
\end{itemize}

\end{document}
