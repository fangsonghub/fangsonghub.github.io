% Created by Fang Song on April 5 2020. Instructions: you don't need
% to change anything in the macros, but feel free to define new
% commands as you wish. Starting from the main body, change the specs
% (e.g., your name). use \begin{solution} \end{solution} environment
% to write your solutions. Don't forget to list your collaborators.

\documentclass[12pt,answers]{exam}
%============Macros==================%
\usepackage{amsmath,amsfonts,amssymb,amsthm}
\usepackage[qm]{qcircuit}
\usepackage[margin=1in]{geometry}

%--------------Cosmetic----------------%
\usepackage{mathtools}
\usepackage{hyperref}
\usepackage{fullpage}
\usepackage{microtype}
\usepackage{xspace}
\usepackage[svgnames]{xcolor}
\usepackage[sc]{mathpazo}
\usepackage{enumitem}
\setlist[enumerate]{itemsep=1pt,topsep=2pt}
\setlist[itemize]{itemsep=1pt,topsep=2pt}
\usepackage{mdframed} %
\mdfdefinestyle{figstyle}{ %
  linecolor=black!7, %
  backgroundcolor=black!7, %
  innertopmargin=10pt, %
  innerleftmargin=25pt, %
  innerrightmargin=25pt, %
  innerbottommargin=10pt %
}

%----------Header--------------------%
\def\course{CS 410/510 Introduction to Quantum Computing}
\def\term{Portland State U, Spring 2020}
\def\prof{Lecturer: Fang Song}
\newcommand{\handout}[5]{
   \renewcommand{\thepage}{\arabic{page}}
   \begin{center}
   \framebox{
      \vbox{
    \hbox to 5.78in { \hfill \large{\course} \hfill }
       \vspace{2mm}
       \hbox to 5.78in { {\Large \hfill #5  \hfill} }
       \vspace{2mm}
       \hbox to 5.78in { \term \hfill \emph{#2}}
       \hbox to 5.78in { {#3 \hfill \emph{#4}}}
      }
   }
   \end{center}
   \vspace*{4mm}
 }
 
\newcommand{\hw}[4]{\handout{#1}{#2}{#3}{#4}{Homework #1}}

%-----defs and commands-----%
\def\mG{[\textbf{G}]\xspace}
\def\veps{\varepsilon}
\def\tr{\mathrm{tr}}
\newcommand{\bit}{\{0,1\}}
\newcommand{\bra}[1]{\langle #1 \rvert}
\newcommand{\ket}[1]{\lvert #1 \rangle}
\newcommand{\kera}[1]{\ket{#1}\bra{#1}}

%=======Main document==============%
\begin{document}

%----Specs: change accordingly-----%
\newif\ifstudent % comment out false
\studenttrue 
%\studentfalse

\def\hwnum{2} %
\def\issuedate{04/12/2020} %
\def\duedate{11:59pm PDT, 04/19/2020} % 
\def\yourname{your name} % put your name here
%------------------------------%
\ifstudent
\hw{\hwnum}{\issuedate}{Student: \yourname}{Due: \duedate}%
\else
\hw{\hwnum}{\issuedate}{\prof}{Due: \duedate}%
\fi

\noindent \textbf{Instructions.}  This problem set contains \numpages\
pages (including this cover page) and \numquestions\
questions. Problems marked with ``\mG'' are required for 510
students. Students enrolled in 410 will get bonus points for solving
them. A random subset of problems will be graded.

Your solutions will be graded on \emph{correctness} and
\emph{clarity}. You should only submit work that you believe to be
correct, and you will get significantly more partial credit if you
clearly identify the gap(s) in your solution. It is good practice to
start any long solution with an informal (but accurate) summary that
describes the main idea.

You need to submit a PDF file via Gradescope before the
deadline. Either a clear scan of you handwriting or a typeset document
is accepted. You will get 5 bonus points for typing in LaTeX (Download
and use the accompany TeX file).

\medskip
\noindent You may collaborate with others on this problem set% and consult external sources
.  However, you must \textbf{\emph{write up your own solutions}} and
\textbf{\emph{list your collaborators}} for each problem.

\medskip

\begin{mdframed}[style=figstyle,innerleftmargin=10pt,innerrightmargin=10pt]

Across this entire problem set, let

\[X = \begin{pmatrix}
      0 & 1 \\
      1 & 0 
    \end{pmatrix}
    ,\quad Y =
    \begin{pmatrix}
      0 & i \\
      -i & 0
    \end{pmatrix}, \quad Z=
    \begin{pmatrix}
      1 & 0 \\
      0 & -1
    \end{pmatrix} \, .\]
\end{mdframed}

\newpage 
\begin{questions}
\question (Tensor product) Recall the \emph{tensor product} of two
  matrices $A$ and $B$ is $A\otimes B : = (a_{ij}B)$.
  \begin{parts}
  \part[6] Write out the $4 \times 4$ matrix representing
    $ X \otimes Y$. Does it equal $Y \otimes X$?
% \begin{solution}
% your solution here  
% \end{solution}

\newpage 
\part[6] Show that if $U$ and $V$ are unitary matrices, then
  $U \otimes V$ is also unitary.
  
\newpage
\part[6] Let $A,B,C$ and $D$ be matrices such that the matrix products
  $AC$ and $BD$ are well defined. Show that
  $(A\otimes B) (C\otimes D) = (AC)\otimes (BD)$.

  
\newpage 
\end{parts}

\question (Quantum states and gates) 
  \begin{parts}
  \part[12] For each of the processes below, describe the resulting
    quantum states.
  \begin{enumerate}[label=\roman*)]
    \item Apply $H$ to the first qubit of state $\frac{1}{\sqrt 2}
      (\ket{00} + \ket{11})$.
    \item Apply $H$ to both qubits of state
      $\frac{1}{\sqrt 2} (\ket{00} + \ket{11})$.
    \item Apply $\frac{1}{\sqrt 2} \left(
    \begin{array}{lr}
      1 & i\\
      i & 1
    \end{array} \right)$ to both qubits of state $\frac{1}{\sqrt 2}
      (\ket{00} + \ket{11})$.
    \end{enumerate}

    \newpage
    
  \part[6] Suppose we have two qubits. We apply $X$ to both and then
    $Z$ to both. Is it equivalent to applying $Z$ to both and then
    applying $X$ to both? Determine your answer by explicitly
    computing $X\otimes X$, $Z\otimes Z$, and their products both
    ways.

    \newpage 
  \part[6] Analyze the following quantum circuit. Describe its effect
    and write down its matrix representation.

    \begin{mdframed}[style=figstyle]
     \Qcircuit @C=1em @R=.7em {
        & \ctrl{1} & \targ & \ctrl{1} & \qw \\
        & \targ & \ctrl{-1} & \targ & \qw } 
    \end{mdframed}
    
  \newpage 
  \end{parts}

  \question (Product states versus entangled states) In each of the
  following, either express the 2-qubit state as a tensor product of
  1-qubit states or prove that it cannot be expressed this way. 
  \begin{parts}
    \part[5] $\frac 1 2 \ket{00} + \frac 1 2 \ket{01} +\frac 1 2
      \ket{10} - \frac{1}{2} \ket{11}$
      
      \newpage
    \part[5] \mG $\frac{3}{4}\ket{00} + \frac{\sqrt
        3}{4}\ket{01}+\frac{\sqrt 3}{4}\ket{10}+\frac{1}{4}\ket{11}$
      \newpage 
  \end{parts}

\end{questions}


\end{document}
