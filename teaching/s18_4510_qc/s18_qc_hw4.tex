% Created by Fang Song on April 4 2018. Instructions: you don't need
% to change anything in the macros, but feel free to define new
% commands as you wish. Starting from the main body, change the specs
% (e.g., your name). use \begin{solution} \end{solution} environment
% to write your solutions. Don't forget to list your collaborators.

\documentclass[12pt,answers]{exam}
%============Macros==================%
\usepackage{amsmath,amsfonts,amssymb,amsthm}
\usepackage{qcircuit}
\usepackage[margin=1in]{geometry}
%--------------Cosmetic----------------%
\usepackage{mathtools}
\usepackage{hyperref}
\usepackage{fullpage}
\usepackage{microtype}
\usepackage{xspace}
\usepackage[svgnames]{xcolor}
\usepackage[sc]{mathpazo}
\usepackage{enumitem}
\setlist[enumerate]{itemsep=1pt,topsep=2pt}
\setlist[itemize]{itemsep=1pt,topsep=2pt}
%----------Header--------------------%
\def\course{CS 410/510 Introduction to Quantum Computing}
\def\term{Portland State U, Spring 2018}
\def\prof{Lecturer: Fang Song}
\newcommand{\handout}[5]{
   \renewcommand{\thepage}{\arabic{page}}
   \begin{center}
   \framebox{
      \vbox{
    \hbox to 5.78in { \hfill \large{\course} \hfill }
       \vspace{2mm}
       \hbox to 5.78in { {\Large \hfill #5  \hfill} }
       \vspace{2mm}
       \hbox to 5.78in { \term \hfill \emph{#2}}
       \hbox to 5.78in { {#3 \hfill \emph{#4}}}
      }
   }
   \end{center}
   \vspace*{4mm}
}
\newcommand{\hw}[4]{\handout{#1}{#2}{#3}{#4}{Homework #1}}

%-----defs and commands-----%
\def\mG{[\textbf{G}]\xspace}
\def\veps{\varepsilon}
\def\tr{\mathrm{tr}}
\newcommand{\bit}{\{0,1\}}
\newcommand{\bZ}{\mathbb{Z}}
\newcommand{\complex}{\mathbb{C}}
\newcommand{\bra}[1]{\langle #1 \rvert}
\newcommand{\ket}[1]{\lvert #1 \rangle}
\newcommand{\kera}[1]{\ket{#1}\bra{#1}}
\newcommand{\corr}[1]{{\color{blue}{#1}}}
%=======Main document==============%
\begin{document}

%----Specs: change accordingly-----%
\newif\ifstudent % comment out false
\studenttrue 
%\studentfalse

\def\hwnum{4}
\def\issuedate{May 16, 2018}
\def\duedate{May 30, 2018} % 
\def\yourname{your name} % put your name here
%------------------------------%
\ifstudent
\hw{\hwnum}{\issuedate}{Student: \yourname}{Due: \duedate}%
\else
\hw{\hwnum}{\issuedate}{\prof}{Due: \duedate}%
\fi
\noindent \textbf{Instructions.}
Your solutions will be graded on \emph{correctness} and
\emph{clarity}. You should only submit work that you believe to be
correct; if you cannot solve a problem completely, you will get
significantly more partial credit if you clearly identify the gap(s)
in your solution. It is good practice to start any long solution with
an informal (but accurate) summary that describes the main
idea. For this problem set, a random subset of problems will be
graded. Problems marked with ``\mG'' are required for graduate
students. Undergraduate students will get bonus points for solving
them. Download the TeX file if you want to typeset your solutions
using LaTeX. 

\medskip
\noindent You may collaborate with others on this problem
set. However, you must \textbf{\emph{write up your own solutions}} and
\textbf{\emph{list your collaborators}} for each problem.

\begin{questions}

  \question (OR gate as a quantum operation) Recall the binary OR
  operation, denoted as $\vee$, defined as $a \vee b = 0$ if
  $a = b = 0$ and $a\vee b = 1$ otherwise. Here we consider operations
  that map the two-qubit state $\ket{a, b}$ to the one-qubit state
  $\ket{a\vee b}$, for all $a,b \in \{0,1\}$. Of course, no unitary
  operation can perform this mapping, since the input and output
  dimension do not match; however, general quantum operations can
  compute this mapping.

  \begin{parts}
    \part[8] Give a sequence of $2\times 4$ matrices $A_1,\ldots, A_k$
    with $\sum_{j = 1}^k A_j^\dagger A_j = I$ that compute the OR
    operation in the sense that, for all $a, b\in\{0,1\}$, when
    $\rho = \kera{a,b}$,
    $\sum_{j = 1}^k A_j\rho A_j\dagger = \kera{a\vee b}$.
    
    \part[7] The operation from part (a) maps all basis states to pure
    states. Does it map all pure input states to pure output states?
    Prove it, or give a counterexample.
    
  \end{parts}

  \question (Partial Trace) Let $Tr_Y(\cdot)$ be the operation of
  partial trace of a subsystem $Y$.
  \begin{parts}
    \part[5] Let $\rho_{AB} = \rho_A\otimes \rho_B$ with $A$ and $B$
    both $n$-qubit systems. Show that $Tr_Y(\rho_{AB}) = \rho_A$.
    \part[5] Let
    $\ket{\phi^+}_{AB} = \frac{1}{\sqrt 2} \ket{00} + \ket{11}$ be an
    EPR state. Compute $Tr_B(\ket{\phi^+}\bra{\phi^+})$.
    \part (Bonus 5pts) Show that for any density operator $\rho$ on a
    system $A$, there exists a pure state $\ket{\psi}$ on some larger
    system $A\otimes B$ such that $Tr_B(\ket{\psi}\bra{\psi}) =
    \rho$. 
  \end{parts}
  
  \question (Quantum measurement)
  \begin{parts}
    \part[5] Let
    $\ket{\phi_0} = \frac{1}{\sqrt 2} \ket{00} + \ket{11}$,
    $\ket{\phi_1} = \frac{1}{\sqrt 2} \ket{00} - \ket{11}$,
    $\ket{\phi_2} = \frac{1}{\sqrt 2} \ket{01} + \ket{10}$, and
    $\ket{\phi_3} = \frac{1}{\sqrt 2} \ket{01} - \ket{10}$ be the four
    Bell states. We know that they form a basis for 2-qubit
    systems. Find the measurement operators $M = \{M_0,M_1,M_2,M_3\}$,
    under the general measurement formalism, for the measurement under
    the Bell basis.

    \part[5] Continuing from above. Let $\rho$ be a state on two
    qubits, and measure it under the Bell basis. When
    $\rho = \ket{\psi}\bra{\psi}$ is a pure state, the measurement
    postulate states that the probability of observing outcome
    $i\in\{0,1,2,3\}$ is $|\langle{\psi} \ket{\phi_i}|^2$. Show that for
    general density $\rho$, one observes $i$ with probability
    $\bra{\phi_i} \rho \ket{\phi_i}$. Verify that this probability
    also equals $Tr(M_i\rho M_i^+)$ for the $M_i$ you identified in
    part a). 
    
    \part (Bonus 10pts) (Simulating POVM) We've mentioned in class
    that any admissible quantum operation may be simulated by unitary
    operations and ordinary projective measurement under standard
    basis. In this problem, you will prove a special case, the POVM
    measurement. Recall in a POVM, we are only interested in the
    probabilities of observing each outcome, and don’t care about how
    the state collapses afterward. Let's consider a POVM measurement
    specified by positive semi-definite operators $E_1, \ldots, E_m$
    on $n$ qubits satisfying $\sum_jE_j = I$. Let $m = 2^k$ for some
    integer $k$. Show how to simulate it by applying a unitary
    operation on a lager quantum system and then measuring under
    standard basis. (Hint: consider $m 2^n \times 2^n$ matrix $V$
    formed by vertically stacking $E_j$. Show the columns are
    orthogonal, and extend $V$ to a unitary of dimension $m2^n$.)
  \end{parts}


  \question (Entropy) Let $H(\cdot)$ denote the Shannon entropy and
$S(\cdot)$ be the von Neumann entropy. $S(A:B)$ denotes quantum mutual
information.
  \begin{parts}
    \part (Exercise) Let $X$ be a random variable taking values in
    $\{0,\ldots, 2^{m^2}\}$ with probability distribution $p_x =
    \left\{
      \begin{array}{c l}
        1 - 1/m & \text{ if } x = 0\\
        \frac{1}{m 2^{m^2}} & \text{ otherwise } \, .
      \end{array} \right.$
    Calculate $H(X)$? Conclude that $H(X) \to \infty$ as $m\to
    \infty$, but one sample of $X$ is almost {certainly} 0. 

    \part[5] Let $\rho = p \kera{0} + (1-p)\kera{+}$. Compute
    $S(\rho)$. How does it compare to the entropy of a biased coin $X$
    which is HEADS with probability $p$?

    \part[8] Consider $\rho_{AB} = \rho_A\otimes \rho_B$ where $A$ and
    $B$ are both $n$-qubit systems. Compute the von Neumann entropy
    $S(\rho_{AB})$ and the quantum mutual information $S(A:B)$.

  \end{parts}
  
  \question (Quantum error-correcting) 
  \begin{parts}
    \part[10] Let $E$ be an arbitrary 1-qubit unitary, and $I,X,Y,Z$
    are the four $2\times 2$ Pauli matrices.
    \begin{enumerate}[label=\roman*)]
    \item Show that it can be written as
      $E = \alpha_0 I + \alpha_1X + \alpha_2 Y + \alpha_3 Z$, for some
      complex coefficients $\alpha_i$ with
      $\sum_{i=0}^3 |\alpha_i|^2 = 1$. (Hint: compute the trace
      $Tr(E^\dagger E)$ in two ways, and use the fact that
      $Tr(AB) = 0$ if $A$ and $B$ are distinct Pauli matricies, and
      $Tr(AB) = Tr(I) = 2$ if $A$ and $B$ are the same Pauli.)
    \item Write the 1-qubit Hadamard transform H as a linear combination
      of the four Pauli matrices.
    \end{enumerate}
    
      \part[10] Show that there cannot be a quantum code that encodes
      one logical qubit by $2k$ physical qubits while being able to
      correct errors on up to $k$ of the qubits. (Hint: No-cloning
      theorem)
    \end{parts}

  \question (Testing entanglement) Suppose that Alice and Bob share a
  two-qubit state, and they want to test if it is the EPR pair
  $\ket{\phi^+} := \frac{1}{\sqrt 2}(\ket{00}+\ket{11})$ with local
  measurements and classical communication. Consider the following
  procedure: they randomly select a measurement basis: with
  probability $1/2$, they both measure in the standard basis
  $\{\ket{0},\ket{1}\}$; and, with probability $1/2$, they both
  measure in the Hadamard (diagonal) basis $\{\ket{+},\ket{-}\}$. Then
  they perform the measurement and they accept if and only if their
  outcomes are the same.
  \begin{parts}
    \part[5] Show that the state $\phi^+$ is always accepted by this
    test with zero-error.
    \part[8] Show that, for an arbitrary 2-qubit state $\ket{\mu}$, the
    probability that it passes the test is at most
    \begin{equation*}
      \frac{1 + |\langle {\mu}|{\phi^+}\rangle|^2}{2}   \, .
    \end{equation*}
    (Hint: decomposing $\ket{\mu}$ under the four Bell states.)
    \part[7] Now consider another (malicious) party Eve, who may have
    intervened with the state that Alice shares with Bob. Let
    $\rho_{ABE}$ be their joint state. Now assume that Alice and Bob
    are certain that they two perfectly share $\ket{\phi^+}$, show
    that Alice and Eve's state cannot be in $\ket{\phi^+}$ as well.

    Note: we can actually show that $\rho_{ABE}$ must be of form
    $\kera{\phi^+}_{AB}\otimes \rho_E$ (i.e., Eve's state is
    uncorrelated with that of Alice and Bob). This is an example of
    \emph{monogamy of entanglement}: the more system $A$ is entangled
    with $B$, the less $A$ is entangled with another system $C$.
  \end{parts}
\end{questions}


\end{document}
