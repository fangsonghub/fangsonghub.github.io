% Created by Fang Song on April 4 2018. Instructions: you don't need
% to change anything in the macros, but feel free to define new
% commands as you wish. Starting from the main body, change the specs
% (e.g., your name). use \begin{solution} \end{solution} environment
% to write your solutions. Don't forget to list your collaborators.

\documentclass[12pt,answers]{exam}
%============Macros==================%
\usepackage{amsmath,amsfonts,amssymb,amsthm}
\usepackage{qcircuit}
\usepackage[margin=1in]{geometry}
%--------------Cosmetic----------------%
\usepackage{mathtools}
\usepackage{hyperref}
\usepackage{fullpage}
\usepackage{microtype}
\usepackage{xspace}
\usepackage[svgnames]{xcolor}
\usepackage[sc]{mathpazo}
\usepackage{enumitem}
\setlist[enumerate]{itemsep=1pt,topsep=2pt}
\setlist[itemize]{itemsep=1pt,topsep=2pt}
%----------Header--------------------%
\def\course{CS 410/510 Introduction to Quantum Computing}
\def\term{Portland State U, Spring 2018}
\def\prof{Lecturer: Fang Song}
\newcommand{\handout}[5]{
   \renewcommand{\thepage}{\arabic{page}}
   \begin{center}
   \framebox{
      \vbox{
    \hbox to 5.78in { \hfill \large{\course} \hfill }
       \vspace{2mm}
       \hbox to 5.78in { {\Large \hfill #5  \hfill} }
       \vspace{2mm}
       \hbox to 5.78in { \term \hfill \emph{#2}}
       \hbox to 5.78in { {#3 \hfill \emph{#4}}}
      }
   }
   \end{center}
   \vspace*{4mm}
}
\newcommand{\hw}[4]{\handout{#1}{#2}{#3}{#4}{Homework #1}}

%-----defs and commands-----%
\def\mG{[\textbf{G}]\xspace}
\def\veps{\varepsilon}
\def\tr{\mathrm{tr}}
\newcommand{\bit}{\{0,1\}}
\newcommand{\bZ}{\mathbb{Z}}
\newcommand{\complex}{\mathbb{C}}
\newcommand{\bra}[1]{\langle #1 \rvert}
\newcommand{\ket}[1]{\lvert #1 \rangle}
\newcommand{\kera}[1]{\ket{#1}\bra{#1}}
\newcommand{\corr}[1]{{\color{blue}{#1}}}
%=======Main document==============%
\begin{document}

%----Specs: change accordingly-----%
\newif\ifstudent % comment out false
\studenttrue 
%\studentfalse

\def\hwnum{3}
\def\issuedate{May 2, 2018, Update: May 16}
\def\duedate{May 18, 2018} % 
\def\yourname{your name} % put your name here
%------------------------------%
\ifstudent
\hw{\hwnum}{\issuedate}{Student: \yourname}{Due: \duedate}%
\else
\hw{\hwnum}{\issuedate}{\prof}{Due: \duedate}%
\fi
\noindent \textbf{Instructions.}
Your solutions will be graded on \emph{correctness} and
\emph{clarity}. You should only submit work that you believe to be
correct; if you cannot solve a problem completely, you will get
significantly more partial credit if you clearly identify the gap(s)
in your solution. It is good practice to start any long solution with
an informal (but accurate) summary that describes the main
idea. For this problem set, a random subset of problems will be
graded. Problems marked with ``\mG'' are required for graduate
students. Undergraduate students will get bonus points for solving
them. Download the TeX file if you want to typeset your solutions
using LaTeX. 

\medskip
\noindent You may collaborate with others on this problem
set. However, you must \textbf{\emph{write up your own solutions}} and
\textbf{\emph{list your collaborators}} for each problem.

\begin{questions}
  \question (QFT and periodic states) Let $m$ be some integer and
  $M=2^m$. Denote $\omega_M = e^{2\pi i/M}$ and
  $[M]= \{0,1,\ldots, M-1 \}$. Recall the Quantum Fourier Transform
  $F_{M}$:
  \begin{equation*}
    \forall x\in [M], \quad \ket{x}\, \overset{F_M}{\mapsto} \, \frac{1}{\sqrt
      M}\sum_{y=0}^{M-1} \omega_M^{x y} \ket{y}  \,  . 
  \end{equation*}
  \begin{parts}
    \part[3] Let $\ket{\alpha} := \sum_{x = 0}^{M-1} \alpha_x \ket{x}$ be
    an arbitrary $m$-qubit state (note:
    $\sum_{x=0}^{M-1} |\alpha_x|^2 = 1$). Show that
    $$F_M \ket{\alpha} =  \ket{\hat \alpha}: = \sum_{y=0}^{M-1} \hat
    \alpha_y \ket{y} \, ,$$ where
    $\hat \alpha_y = \frac{1}{\sqrt M}\sum_{x} \omega_M^{xy}\alpha_x$, 
    for all $y \in [M]$.
    
    \part[5] (Index shift) Let $j \in [M]$, and define
    $\ket{\alpha_{+j}}:= \sum_x \alpha_x \ket{x + j \bmod M}$. Compute
    $\ket{\hat \alpha_{+j}}:=F_M \ket{\alpha_{+j}} =?  $. According to
    your result, explain that measuring $\ket{\hat \alpha}$ and
    $\ket{\hat \alpha_{+j}}$ produce the same probability
    distribution.

    \part[7] (Periodic state) Consider an integer $r$ such that
    $M= \ell \cdot r$ for some $\ell \in \mathbb{Z}$. Let
    $b \in \{0, 1, \ldots, r-1\}$. Define
    $$\ket{P_{r,b}}:= \frac{1}{\sqrt{\ell}}\sum_{k = 0}^{\ell - 1} \ket{kr+b}
    = \frac{1}{\sqrt{\ell}}(\ket{0+b} + \ket{r+b} + \ldots +
    \ket{(\ell - 1) r + b})\, .$$ Show that
    $$F_M \ket{P_{r,b}} = \frac{1}{\sqrt{r}} \sum_{k=0}^{r-1}
    \omega_r^{b k} \ket{k\ell} = \frac{1}{\sqrt r}(\ket{0} + \omega_r\corr{^b}
    \ket{\ell} + \ldots + \omega_r^{b(r-1)}\ket{(r-1)\ell})\, . $$
    \part[5] Given multiple copies of $\ket{P_{r,b}}$, how to find $r$?
  \end{parts}

  
  \question (Shor's algorithm) In this problem, we analyze Shor's
  algorithm for order finding that we've seen briefly in class. Our
  goal is to find the order of $a$ mod a positive integer $N$, i.e.,
  the smallest $r$ such that $a^r=1\bmod N$.
  $E_a: \ket{x}\ket{y}\mapsto\ket{x}\ket{y+a^x\bmod N}$ is the unitary
  circuit implementing the modular exponentiation function
  $x\in\mathbb{Z}_{2^m} \mapsto a^x \bmod N$.
  \begin{figure}[ht]
    \centerline{ \Qcircuit @C=2em @R=1.5em { \lstick{\ket{0^m}} &
        \gate{H^{\otimes m}} & \multigate{1}{E_a} &
        \gate{F_M} &  \meter & \qw \quad {s}  \\
        \lstick{\ket{0^n}}& \qw &\ghost{E_a} & \qw & \qw & \qw } }
    \caption{Shor's algorithm}
  \end{figure}
  \begin{parts}
    \part[4] Consider the point right after applying $E_a$. Since the
    bottom register will no longer be used, we may assume that it gets
    measured immediately. Suppose that the outcome is
    $s = a^b \bmod N$ for some $b\in [r]$, what is the state on the
    top register then?
    \part[4] Assuming it happens that $r | M = 2^m$. How to find $r$?
    \part[4] In general, we may not be able to pick an $m$ such that
    $r | M$ (since we do not know $r$). In this case ($r \nmid M$),
    what is the state after applying $F_M$?
    \part (Bonus 6 points) Continuing part (c), how to find $r$ in the
    general case?
  \end{parts}

  \question (Grover's algorithm) Consider as usual a function
  $f:\{0,1\}^n \to \{0,1\}$. Define
  $A = f^{-1}(1):= \{ x\in \bit^n: f(x) = 1\}$ and
  $B = f^{-1}(0) := \{x\in \bit^n: f(x) =0\}$. Let $a = |A|$ be the
  number of ``marked'' items and $b = N-a$ where $N=2^n$. We further
  define $\ket{A}: = \frac{1}{\sqrt a} \sum_{x\in A} \ket{x}$ and
  $\ket{B}:= \frac{1}{\sqrt b} \sum_{x\in B} \ket{x}$.
\begin{parts}
  \part[4] Run Grover's algorithm with $f$ on input
  $\ket{0^n}$. Namely we apply unitary $G := -HZ_0HZ_f$ repetitively
  on $\ket{\psi_0}:=H\ket{0^n}$ where $Z_0: \ket{x}\mapsto - \ket{x}$
  iff. $x = 0^n$ and $Z_f: \ket{x}\mapsto (-1)^{f(x)}\ket{x}$. Denote
  $\ket{\psi^0} = H\ket{0^n}$ and $\ket{\psi^t}:= G^t\ket{\psi^0}$ for
  $t\geq 1$.  Show that every $\ket{\psi^t}$ can be written as
  ${\sin}(\theta_t)\ket{A} + \cos(\theta_t)\ket{B}$. How does
  $\theta_t$ change to $\theta_{t+1}$? (Hint: geometric interpretation
  of Grover's algorithm)

  \part[6] Show that $|\langle{\psi^t}|{A}\rangle|^2 \geq 1/2$ if we
  pick $t$ to be the nearest integer to
  $\frac 1 2 (\frac{\pi}{2\theta_0} - 1)$. As a result, measuring
  $\ket{\psi^t}$ will give an $x$ of $A$ with probability at least
  $1/2$. Assuming $a$ is known to us. How to find an element of $A$
  with high probability using $O(\sqrt{N/a})$ queries to $f$?

  \part (Bonus 8 points) What if $a$ is unknown? Show how to find a
  marked item with probability $\geq 0.99$ within $O(\sqrt{N/a})$
  queries.  (Hint: what if one picks $t\in \{1,\ldots, \sqrt N + 1\}$
  at random and run Grover's algorithm, how likely will one find a
  marked element? Then how about picking random $t \in \{1,\ldots T\}$
  with a small $T$ in the beginning and try? If fail increment $T$
  slowly but exponentially.)
\end{parts}

\question (Amplitude estimation) Consider a unitary $U$ on $n$ qubits
such that
  $$U\ket{0^n} = \sin(\theta) \ket{\psi_A} + \cos(\theta)\ket{\psi_B}
  \, ,$$ with $0 < \theta < \pi/2$, where
  $\ket{\psi_A} = \sum_{x\in A} \alpha_x \ket{x}$ and
  $\ket{\psi_B} = \sum_{x\in B} \beta_x \ket{x}$ with
  $A,B\subseteq \bit^n$ and $A \cap B = \emptyset$. Let $Z_f$ be an
  efficient operator such that
  \begin{equation*}
    Z_f\ket{\psi_A} = - \ket{\psi_A}, \quad     Z_f\ket{\psi_B} = 
    \ket{\psi_B} \, .
  \end{equation*}
  The goal is to compute $\sin(\theta)$. 

  \begin{parts}
    
    \part[5] Let $Q$ be the rotation of $2\theta$ on the plane spanned by
    $\{\ket{\psi_A}, \ket{\psi_B}\}$. In particular 
    $$ Q \ket{\psi_A} = \cos(2\theta) \ket{\psi_A} - \sin(2\theta)
    \ket{\psi_B}, \quad Q \ket{\psi_B} = \sin(2\theta) \ket{\psi_A} +
    \sin(2\theta) \ket{\psi_B} \, .$$ Show how to implement $Q$
    efficiently.

    \part[5] Let
    $\ket{\psi_+}: = \frac{1}{\sqrt 2} (\ket{\psi_B} - i
    \ket{\psi_A})$ and
    $\ket{\psi_-}: = \frac{1}{\sqrt 2} (\ket{\psi_B} +
    i\ket{\psi_A})$. Show that $\ket{\psi_+}$ and $\ket{\psi_-}$ are
    eigenvectors of $Q$ with eigenvalues $e^{i 2\theta}$ and
    $e^{-i2\theta}$ respectively.

    \part[5] Design an efficient quantum algorithm to estimate
    $\theta$. (Hint: how does
    $H^{\otimes n} \ket{0^n} = \frac{1}{\sqrt {2^n}}\sum \ket{x}$
    relate to $\ket{\psi_+}$ and $\ket{\psi_-}$?)

    \part (Bonus 5 points) Back to Grover's search problem. Can you
    design a quantum algorithm that (approximately) counts the number
    of marked elements?
  
  \end{parts}

\question (Mixed states and density matrix)

\begin{parts}
  \part[5] A density matrix $\rho$ corresponds to a pure state if and
  only if $\rho = \kera{\psi}$. Show that $\rho$ corresponds to a pure
  state if and only if $Tr(\rho^2) = 1$.
  \part[5] Show that every $2\times 2$ density matrix $\rho$ can be
  expressed as an equally weighted mixture of pure states. That is
  $\rho = \frac 1 2 \kera{\psi_1} + \frac 1 2 \kera{\psi_2}$ (Note:
  the two states need not be orthogonal).
  \part[5] Imagine two parties Alice and Bob. Alice flips a biased coin
  which is HEADS with probability $\cos^2(\pi/8)$. Alice prepares
  $\ket{0}$ when she sees coin $0$ and $\ket{1}$ otherwise. From
  Alice’s perspective (who knows the coin value), the density matrix
  of the state she created will be either $\kera{0}$ or
  $\kera{1}$. She then sends the qubit to Bob. What is the density
  matrix of the state from Bob’s perspective (who does not know the
  coin value)? Write down the matrix.
\end{parts}


\end{questions}


\end{document}
