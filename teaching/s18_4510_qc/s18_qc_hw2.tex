% Created by Fang Song on April 4 2018. Instructions: you don't need
% to change anything in the macros, but feel free to define new
% commands as you wish. Starting from the main body, change the specs
% (e.g., your name). use \begin{solution} \end{solution} environment
% to write your solutions. Don't forget to list your collaborators.

\documentclass[12pt,answers]{exam}
%============Macros==================%
\usepackage{amsmath,amsfonts,amssymb,amsthm}
\usepackage{qcircuit}
\usepackage[margin=1in]{geometry}
%--------------Cosmetic----------------%
\usepackage{mathtools}
\usepackage{hyperref}
\usepackage{fullpage}
\usepackage{microtype}
\usepackage{xspace}
\usepackage[svgnames]{xcolor}
\usepackage[sc]{mathpazo}
\usepackage{enumitem}
\setlist[enumerate]{itemsep=1pt,topsep=2pt}
\setlist[itemize]{itemsep=1pt,topsep=2pt}
%----------Header--------------------%
\def\course{CS 410/510 Introduction to Quantum Computing}
\def\term{Portland State U, Spring 2018}
\def\prof{Lecturer: Fang Song}
\newcommand{\handout}[5]{
   \renewcommand{\thepage}{\arabic{page}}
   \begin{center}
   \framebox{
      \vbox{
    \hbox to 5.78in { \hfill \large{\course} \hfill }
       \vspace{2mm}
       \hbox to 5.78in { {\Large \hfill #5  \hfill} }
       \vspace{2mm}
       \hbox to 5.78in { \term \hfill \emph{#2}}
       \hbox to 5.78in { {#3 \hfill \emph{#4}}}
      }
   }
   \end{center}
   \vspace*{4mm}
}
\newcommand{\hw}[4]{\handout{#1}{#2}{#3}{#4}{Homework #1}}

%-----defs and commands-----%
\def\mG{[\textbf{G}]\xspace}
\def\veps{\varepsilon}
\def\tr{\mathrm{tr}}
\newcommand{\bit}{\{0,1\}}
\newcommand{\bZ}{\mathbb{Z}}
\newcommand{\complex}{\mathbb{C}}
\newcommand{\bra}[1]{\langle #1 \rvert}
\newcommand{\ket}[1]{\lvert #1 \rangle}
\newcommand{\kera}[1]{\ket{#1}\bra{#1}}
\newcommand{\corr}[1]{{\color{blue}{#1}}}
%=======Main document==============%
\begin{document}

%----Specs: change accordingly-----%
\newif\ifstudent % comment out false
\studenttrue 
%\studentfalse

\def\hwnum{2}
\def\issuedate{\corr{Update}: April 23, 2018}
\def\duedate{May 02, 2018} % 
\def\yourname{your name} % put your name here
%------------------------------%
\ifstudent
\hw{\hwnum}{\issuedate}{Student: \yourname}{Due: \duedate}%
\else
\hw{\hwnum}{\issuedate}{\prof}{Due: \duedate}%
\fi
\noindent \textbf{Instructions.}
Your solutions will be graded on \emph{correctness} and
\emph{clarity}. You should only submit work that you believe to be
correct; if you cannot solve a problem completely, you will get
significantly more partial credit if you clearly identify the gap(s)
in your solution. It is good practice to start any long solution with
an informal (but accurate) summary that describes the main
idea. For this problem set, a random subset of problems will be
graded. Problems marked with ``\mG'' are required for graduate
students. Undergraduate students will get bonus points for solving
them. Download the TeX file if you want to typeset your solutions
using LaTeX. 

\medskip
\noindent You may collaborate with others on this problem
set. However, you must \textbf{\emph{write up your own solutions}} and
\textbf{\emph{list your collaborators}} for each problem.

\begin{questions}
  \question (Linear algebra)
  \begin{parts}

    \part[6] A linear operator on a vector space $V$ (think of
    $\complex^k$) is a linear transformation $T: V \to V$ of the
    vector space to itself. A vector $\ket{\psi}$ is called an
    eigenvector of an operator $T$ if
    $T \ket{\psi} = \lambda \ket{\psi}$ for some constant
    $\lambda \in \complex$. $\lambda$ is called the eigenvalue
    corresponding to the eigenvector $\ket{\psi}$. Find the eigenvalues
    and eigenvectors of Pauli operator $X$. Show that the eigenvectors
    form an orthonormal basis of $\complex^2$. Do the same for
    $X\otimes X$.
    % \begin{solution}
      
    % \end{solution}

    \part[6] An operator $T$ is called \emph{Hermitian}, if
    $T^\dagger = T$. Prove that the eigenvalues of a Hermitian
    operator are all real numbers. Show that the eigenvalues of
    unitary operators are of the form $e^{i\theta}$ for some
    $\theta \in [0,2\pi)$.
    
    \part[4] Show that for any $x\in \bit^n$,
    $H^{\otimes n} \ket{x} = \corr{\frac{1}{\sqrt{2^n}}}\sum_{y\in \bit^n} (-1)^{x\cdot
      y}\ket{y}$. $x\cdot y : = \sum_{i = 1}^n x_iy_i$ is the dot
    product over $\bZ_2^n$.

    \part[4] Let $x,y\in \bit^n$ and let $s = x\oplus y$. Show that
    \begin{equation*}
      H^{\otimes n} \frac{1}{\sqrt 2}(\ket{x} + \ket{y}) =
      \frac{1}{\sqrt {2^{n-1}}}\sum_{z: z\cdot s = 0} (-1)^{x\cdot z}
      \ket{z} \, .
    \end{equation*}

    \part[5] Suppose that
    $\ket{v_1}, \ket{v_2}, \ldots \ket{v_k} \in \complex^k$ form an
    orthonormal basis. Show that $\sum_{i=1}^k \ket{v_i}\bra{v_i}$ is
    the identity matrix.

    \part[5] $\mG$ Show that every unitary one-qubit gate with real entries can
    be written as a rotation matrix, possibly preceded and followed by
    Z-gates. In other words, show that for every $2 \times 2$ real
    unitary U, there exist signs $s_1, s_2, s_3 \in \{1, -1\}$ and
    angle $\theta \in [0, 2\pi)$ such that

    \begin{equation*}
      U =  s_1   \left(\begin{array}{lr}
      1 & 0\\
      0 & s_2
              \end{array} \right) \left( \begin{array}{lr}
      \cos \theta & - \sin\theta\\
      \sin\theta & \cos\theta
                                         \end{array} \right) \left(    \begin{array}{lr}
                                                                1 & 0\\
                                                                0 & s_3
                                                                       \end{array} \right) \, . 
    \end{equation*}

    \part[8] For a vector $v = (v_0, \ldots, v_{k-1})\in \complex^k$,
    let $\|v\|:=\sqrt{\sum_{i=0}^{k-1} |v_i|^2}$, which is the usual
    Euclidean length of $v$. For any $k \times k$ matrix
    $M\in \complex^{k\times k}$, define its \emph{spectral norm}
    $\|M \|$ as $\|M\| = \max_{\ket{\psi}} \| M \ket{\psi}\|$, where
    the maximum is taken over quantum states (i.e., vectors
    $\ket{\psi}$ such that $\| \ket{\psi}\| = 1$).  Define the
    distance between two $k \times k$ unitary matrices $M_1$ and $M_2$
    as $\|M_1 - M_2\|$. Show that
    \begin{enumerate}[label=\roman*)]
    \item $\|A - B\| \leq \|A - C\| + \| C - B \|$, for any three
      $k \times k$ matrices A, B, and C. (Thus, this distance measure
      satisfies the \emph{triangle inequality}.
    \item Show that, for any two $k \times k$ unitary matrices $U_1$
      and $U_2$, and any matrix $A$, $\|U_1AU_2\| = \|A\|$.
    \end{enumerate}
  \end{parts}
  
  \question (Simple search algorithms) In the context of this
  question, we are interested in exact solutions (with failure
  probability zero).

  \begin{parts}
    \part[6] (1-out-of-4 search) Consider a black-box function
    $f:\bit^2 \to \bit$ with the property that there is a unique
    $x\in \bit^{\corr{2}}$ such that $f(x) = 1$ and the goal is to
    determine $x$. How many classical queries are necessary to solve
    this problem? Design a quantum algorithm that finds $x$ using 1
    quantum query.
    \part[6] (2-out-of-4 search) Given a black-box for a function
    $f:\bit^2 \to \bit$ with exactly two $x\in \bit^2$ such that
    $f(x) = 1$ and the goal is to determine both $x$'s. Prove that 3
    classical queries are necessary to solve this problem and that 2
    quantum queries are sufficient to solve this problem.

  \end{parts}

  \question (Quantum Fourier Transform)
  \begin{parts}
    \part[12] Let $F_N$ denote the $N$-dimensional Fourier transform
  \begin{equation*}
    F_N := \frac{1}{\sqrt N}\left(
    \begin{array}{ccccc}
      1 & 1 & 1 & \cdots & 1\\
      1 &\omega_N & \omega_N^2&\cdots& \omega_N^{N-1} \\
      1 &\omega^2_N & \omega_N^4&\cdots& \omega_N^{2(N-1)}\\
      \vdots &\vdots&\vdots&\vdots&\vdots\\
      1 &\omega_N^{N-1} & \omega_N^{2(N-1)}&\cdots& \omega_N^{(N-1)^2}\\
    \end{array} \right),  \text{ where } \omega_N := e^{2\pi i / N} (i =\sqrt {-1})
  \end{equation*}

  (an $N\times N$ matrix, with entry
  $\frac{1}{\sqrt{N}} e^{(2\pi i /N)jk}$ position $j,k$ for
  $j, k \in \{0, 1, \ldots, N-1\}$.

  \begin{enumerate}[label=\roman*)]
  \item Let $N=3$. Calculate $F_3(0,1,0)^T$ and
    $F_3(1, \omega, \omega^2)^T$. $(\cdot)^T$ denotes the transpose.
  \item Show that all rows in $F_N$ are vectors of length 1, and any
    two rows are orthogonal.
  \item What is $F_N^2$? (Hint: The matrix has a very simple form.)
  \item What is the minimum $j$ such that $F_N^j = I$ is the identity?
    
  \end{enumerate}

  \part[5] In class, we computed the QFT modulo $N=2^n$ by a quantum
  circuit of size $O(n^2)$.  Recall that it uses gates of the form
    \begin{equation*}
      R_k = \left(
    \begin{array}{lllc}
      1 & 0 &0 &0\\
      0 & 1 &0 &0 \\
      0 & 0 & 1& 0 \\
      0 & 0 & 0 & e^{2\pi i /{2^k}} 
    \end{array} \right)   
    \end{equation*}
    for $k\in \{2,\ldots,n\}$. Show that $\|R_k - I\| \leq 2\pi/{2^k}$
    , where I is the $4 \times 4$ identity matrix. (Thus, $R_k$ gets
    very close to $I$ when $k$ increases.)
    
    \part[8] Here we compute an \emph{approximation} of this QFT
    within $\veps$ by a quantum circuit of size $O(n
    \log(n/\veps)$. The idea to start with the $O(n^2)$ circuit and
    then remove some of its $R_k$ gates (it is equivalent to changing
    the $R_k$ gate to identity gate). If $k$ is large then removing a
    ${R}_k$ gate changes the unitary transformation by only a small
    amount. Show how to use this approach to obtain a quantum circuit
    of size $O(n\log (n/\veps)$ that computes a unitary transformation
    $\tilde F_N$ such that $\| \tilde F_N - F_N \| \leq \veps$.
    (Hint: Try removing all $R_k$ gates where $k\geq t$, for some
    carefully chosen threshold $t$. The properties of our distance
    measure from the previous question should be useful for your
    analysis here.) For your reference the quantum circuit for QFT is
    given below.

    \begin{figure}[ht]
\centerline{
\Qcircuit @C=1em @R=0.75em {
   \lstick{\ket{x_{n-1}}}   &   \gate{H}  &   \gate{R_2}   &   \gate{R_3}   &   \cdots      &   \gate{R_{n}}   &   \qw        &   \qw           &   \qw      &   \qw           &   \qw        &   \qw     &   \qw           &   \qw        &   \rstick{\ket{y_0}} \qw       \\
   \lstick{\ket{x_{n-2}}}   &   \qw       &   \ctrl{-1}     &   \qw           &   \qw      &   \qw           &   \gate{H}   &   \gate{R_2}   &   \cdots      &   \gate{R_{n-1}}   &   \qw        &   \qw     &   \qw           &   \qw        &   \rstick{\ket{y_1}} \qw       \\
   \lstick{\vdots }         &             &                 &
   &   \ddots   &                 &              &                 &
   \ddots   &                 &              &   \ddots  &
   &              &   \rstick{\vdots }             \\
\lstick{\ket{\corr{x_1}}}   &   \qw       &   \qw           &   \ctrl{-3}     &   \qw      &  \qw            &   \qw        &   \ctrl{-2}     &   \qw      &   \qw           &   \gate{H}   &   \qw     &   \gate{R_2}   &   \qw        &   \rstick{\ket{\corr{y_{n-2}}}} \qw       \\
   \lstick{\ket{x_{0}}}     &   \qw       &   \qw           &   \qw           &   \qw      &   \ctrl{-4}     &    \qw       &   \qw           &   \qw      &   \ctrl{-3}     &   \qw        &   \qw     &   \ctrl{-1}     &   \gate{H}   &   \rstick{\ket{y_{n-1}}} \qw
}
}
\caption{QFT circuit in $\mathbb{Z}_{2^n}$.}
\end{figure}
   
\end{parts}

\end{questions}


\end{document}
