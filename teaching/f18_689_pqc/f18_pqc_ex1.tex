% Instructions: you don't need to change anything in the macros, but
% feel free to define new commands as you wish. Starting from the main
% body, change the specs (e.g., your
% name). use \begin{solution} \end{solution} environment to write your
% solutions. Don't forget to list your collaborators.

\documentclass[12pt,answers]{exam}
%============Macros==================%
\usepackage{amsmath,amsfonts,amssymb,amsthm}
\usepackage{qcircuit}
\usepackage[margin=1in]{geometry}
%--------------Cosmetic----------------%
\usepackage{mathtools}
\usepackage{hyperref}
\usepackage{fullpage}
\usepackage{microtype}
\usepackage{xspace}
\usepackage[svgnames]{xcolor}
\usepackage[sc]{mathpazo}
\usepackage{enumitem}
\setlist[enumerate]{itemsep=1pt,topsep=2pt}
\setlist[itemize]{itemsep=1pt,topsep=2pt}
%----------Header--------------------%
\def\course{CSCE689: FDNS of Post-Quantum Crypto}
\def\term{Texas A\&M U, Fall 2018}
\def\prof{Lecturer: Fang Song}
\newcommand{\handout}[5]{
   \renewcommand{\thepage}{\arabic{page}}
   \begin{center}
   \framebox{
      \vbox{
    \hbox to 5.78in { \hfill \large{\course} \hfill }
       \vspace{2mm}
       \hbox to 5.78in { {\Large \hfill #5  \hfill} }
       \vspace{2mm}
       \hbox to 5.78in { \term \hfill \emph{#2}}
       \hbox to 5.78in { {#3 \hfill \emph{#4}}}
      }
   }
   \end{center}
   \vspace*{4mm}
}
\newcommand{\hw}[4]{\handout{#1}{#2}{#3}{#4}{Homework #1}}
\newcommand{\ex}[4]{\handout{#1}{#2}{#3}{#4}{Exercise #1}}

%-----defs and commands-----%
\def\mG{[\textbf{G}]\xspace}
\def\veps{\varepsilon}
\def\tr{\mathrm{tr}}
\newcommand{\bit}{\{0,1\}}
\newcommand{\bra}[1]{\langle #1 \rvert}
\newcommand{\ket}[1]{\lvert #1 \rangle}
\newcommand{\kera}[1]{\ket{#1}\bra{#1}}

%=======Main document==============%
\begin{document}

%----Specs: change accordingly-----%
\newif\ifstudent % comment out false
%\studenttrue 
\studentfalse

\def\issuedate{September 7, 2018}
\def\duedate{September 18, 2018} % 
\def\yourname{your name} % put your name here
%------------------------------%
\ifstudent
\ex{1}{\issuedate}{Student: \yourname}{Due: \duedate}%
\else
\ex{1}{\issuedate}{\prof}{Due: \duedate}%
\fi
\noindent \textbf{Instructions.}
% Your solutions will be graded on \emph{correctness} and
% \emph{clarity}. You should only submit work that you believe to be
% correct; if you cannot solve a problem completely, you will get
% significantly more partial credit if you clearly identify the gap(s)
% in your solution. It is good practice to start any long solution with
% an informal (but accurate) summary that describes the main
% idea. For this problem set, a random subset of problems will be
% graded. Problems marked with ``\mG'' are required for graduate
% students. Undergraduate students will get bonus points for solving
% them.
This is for your own practice only. For each of the four problems, you
are required to turn in your work for selected subproblems of you own
choice. It will not be graded, and it counts towards your
participation grade. Download the TeX file if you want to typeset your
solutions using LaTeX.

% \medskip
% \noindent You may collaborate with others on this problem set% and consult external sources
% .  However, you must \textbf{\emph{write up your own solutions}} and
% \textbf{\emph{list your collaborators}} for each problem.

\begin{questions}
  \question (Basic Algebra)  Let $X,Y,Z$ be the Pauli operators. 
  \begin{parts}
    \part (complex number) For complex number $c = a + bi$, recall
    that the real and imaginary parts of $c$ are denoted $Re(c) = a$
    and $Im(c) = b$.
  \begin{itemize}
  \item Prove that $c + c^* = 2 \cdot Re(c)$.
  \item Prove that $|c|^2: = cc^* = a^2 + b^2$.
  \item What is the polar form of
    $c = \frac{1}{\sqrt 2} + \frac{1}{\sqrt 2} i$? Use the fact that
    $e^{i\theta} = \cos\theta + i \sin \theta$?
  \end{itemize}
  %\begin{solution}
  %  uncomment the environment and write your solution here
  %\end{solution}

  \part (Trace) Recall the trace of a matrix
  $M = (m_{ij})_{n\times n}, m_{ij}\in \mathbb{C}$ is defined by
  $\tr(M): = \sum_{i = 1}^n m_{ii}$. % Let $A = \left(
    % \begin{array}{lr}
    %   a & b\\
    %   c & d
            %       \end{array} \right)$.
  \begin{itemize}
  \item What is $\tr(X \ket{0}\bra{1})$? 
  \item Show that $\tr(YZ) = \tr(ZY)$. Prove that this holds for
    general matrices: any $n\times n$ matrices $M$ and $N$, $\tr(MN) =
    \tr(NM)$.
    \end{itemize}

    \part (Inner product)
    \begin{itemize}
    \item Let
      $\ket{\phi} = \frac{1}{\sqrt 2} \ket{0} + \frac{i}{\sqrt 2}
      \ket{1}$.
      $\ket{\psi} = \sqrt {\frac{1}{3}}\ket{0}-
      \sqrt{\frac{2}{3}}\ket{1}$. Calculate
      $\bra{\phi}{\psi}\rangle$. 
      
    \item Show that $X = \ket{0}\bra{1} + \ket{1}\bra{0}$. Express
      $Y,Z$ in this outer product form too. Calculate
      $\bra{1} X \ket{0}$ (using linearity).
    \end{itemize}
  \part (Tensor product) Recall the \emph{tensor product} of two
  matrices $A$ and $B$ is $A\otimes B : = (a_{ij}B)$.
  \begin{itemize}
  \item Write out the $4 \times 4$ matrix representing $ X \otimes
    Y$. Does it equal $Y \otimes X$?
  \item Show that $(A\otimes B)^\dagger = A^\dagger \otimes
    B^\dagger$.
    \item Show that $(A\otimes B)(C\otimes D) = AC \otimes BD$. 
  \item Show that if $U$ and $V$ are unitary matrices, then so is
    $U \otimes V$.
  \end{itemize}

  \end{parts}

\question (Quantum states and gates)   
\begin{parts}
  \part In each case, describe the resulting state.
  $H = \frac{1}{\sqrt 2} \left(
    \begin{array}{lr}
      1 & 1\\
      1 & -1
    \end{array} \right)$. 
  \begin{enumerate}[label=\roman*)]
  \item Apply $H$ to the qubit
    $\frac{1}{\sqrt 2} (\ket{0} + i \ket{1})$.
    \item Apply $H$ to the first qubit of state $\frac{1}{\sqrt 2}
      (\ket{00} + \ket{11})$.
    \item Apply $H$ to both qubits of $\frac{1}{\sqrt 2}
      (\ket{00} + \ket{11})$.
    \item Apply $\frac{1}{\sqrt 2} \left(
    \begin{array}{lr}
      1 & i\\
      i & 1
    \end{array} \right)$ to both qubits of state $\frac{1}{\sqrt 2}
      (\ket{00} + \ket{11})$.
    \end{enumerate}
    \part Let $X = \left(
    \begin{array}{lr}
      0 & 1\\
      1 & 0
    \end{array} \right)$ and $Z = \left(
    \begin{array}{lr}
      1 & 0\\
      0 & -1
    \end{array} \right)$.
  \begin{enumerate}[label=\roman*)]
    \item Suppose we have a qubit and we first apply $X$ and then
      $Z$. Is it equivalent to first applying $Z$ and then $X$?
    \item Suppose we have two qubits. We apply $X$ to both and then
      $Z$ to both. Is it equivalent to applying $Z$ to both and then
      applying $X$ to both? Determine your answer by explicitly
      computing $X\otimes X$, $Z\otimes Z$, and their products both
      ways. 
    \end{enumerate}
    \part (SWAP gate) A SWAP gate takes two inputs $a$ and $b$ and
    outputs $b$ and $a$; i.e., it swaps the values of two input
    registers. Show how to build a SWAP gate using only CNOT gates.
    (Hint: you’ll need 3 of them.)
  \end{parts}
  
  \question (Product states versus entangled states) In each of the
  following, either express the 2-qubit state as a tensor product of
  1-qubit states or prove that it cannot be expressed this way. 
  \begin{parts}
    \part $\frac 1 2 \ket{00} + \frac 1 2 \ket{01} +\frac 1 2
    \ket{10} - \frac{1}{2} \ket{11}$
    \part $\frac{3}{4}\ket{00} + \frac{\sqrt
      3}{4}\ket{01}+\frac{\sqrt 3}{4}\ket{10}+\frac{1}{4}\ket{11}$
  \end{parts}

  \question (Distinguishing states by local measurements) Suppose
  Alice and Bob are physically separated from each other, and are each
  given one of the qubits of some 2-qubit state. They are required to
  distinguish between State I and State II with only local
  measurements. Namely they can each perform a local (one-qubit)
  unitary operation and then a measurement (in the computational
  basis) of their own qubit. After their measurements, they can send
  only classical bits to each other. (This is usually referred to as
  LOCC: local operation and classical communication.)  In each case
  below, either give a perfect distinguishing procedure (that never
  errs) or explain why there is no perfect distinguishing procedure
  (i.e., that for any procedure the success probability must be less
  than 1).

  \begin{parts}
    \part State I: $\frac{1}{\sqrt 2}(\ket{00} + \ket{11})$; State II: $\frac{1}{\sqrt 2}(\ket{01} + \ket{10})$ 
    \part State I: $\frac{1}{\sqrt 2} (\ket{00} + \ket{11})$; State
    II: $\frac{1}{\sqrt 2} (\ket{00}-\ket{11})$
    \part State I: $\frac{1}{\sqrt 2}(\ket{00} + i \ket{11})$;
    State II: $\frac{1}{\sqrt 2}(\ket{00}-i\ket{11})$
  \end{parts}

\end{questions}


\end{document}
