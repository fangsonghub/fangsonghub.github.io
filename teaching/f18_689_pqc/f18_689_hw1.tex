% Instructions: you don't need to change anything in the macros, but
% feel free to define new commands as you wish. Starting from the main
% body, change the specs (e.g., your
% name). use \begin{solution} \end{solution} environment to write your
% solutions. Don't forget to list your collaborators.

\documentclass[12pt,answers]{exam}
%============Macros==================%
\usepackage{amsmath,amsfonts,amssymb,amsthm}
\usepackage{qcircuit}
\usepackage[margin=1in]{geometry}
%--------------Cosmetic----------------%
\usepackage{mathtools}
\usepackage{hyperref}
\usepackage{fullpage}
\usepackage{microtype}
\usepackage{xspace}
\usepackage[svgnames]{xcolor}
\usepackage[sc]{mathpazo}
\usepackage{enumitem}
\setlist[enumerate]{itemsep=1pt,topsep=2pt}
\setlist[itemize]{itemsep=1pt,topsep=2pt}
%----------Header--------------------%
\def\course{CSCE689: FDNS of Post-Quantum Crypto}
\def\term{Texas A\&M U, Fall 2018}
\def\prof{Lecturer: Fang Song}
\newcommand{\handout}[5]{
   \renewcommand{\thepage}{\arabic{page}}
   \begin{center}
   \framebox{
      \vbox{
    \hbox to 5.78in { \hfill \large{\course} \hfill }
       \vspace{2mm}
       \hbox to 5.78in { {\Large \hfill #5  \hfill} }
       \vspace{2mm}
       \hbox to 5.78in { \term \hfill \emph{#2}}
       \hbox to 5.78in { {#3 \hfill \emph{#4}}}
      }
   }
   \end{center}
   \vspace*{4mm}
}
\newcommand{\hw}[4]{\handout{#1}{#2}{#3}{#4}{Homework #1}}
\newcommand{\ex}[4]{\handout{#1}{#2}{#3}{#4}{Exercise #1}}

%-----defs and commands-----%
\def\mG{[\textbf{G}]\xspace}
\def\veps{\varepsilon}
\def\tr{\mathrm{tr}}
\newcommand{\bit}{\{0,1\}}
\newcommand{\bra}[1]{\langle #1 \rvert}
\newcommand{\ket}[1]{\lvert #1 \rangle}
\newcommand{\kera}[1]{\ket{#1}\bra{#1}}
\newcommand{\negl}{\text{negl}}
\newcommand{\srd}[2]{\textsf{SR}_{#1}^{#2}(X)}
\newcommand{\corr}[1]{{\color{blue}{#1}}}
%=======Main document==============%
\begin{document}

%----Specs: change accordingly-----%
\newif\ifstudent % comment out false
\studenttrue 
% \studentfalse

\def\issuedate{October 11, 2018; \corr{Update: Oct. 21}}
\def\duedate{October 30, 2018} % 
\def\yourname{your name} % put your name here

%------------------------------%
\ifstudent
\hw{1}{\issuedate}{Student: \yourname}{Due: \duedate}%
\else
\hw{1}{\issuedate}{\prof}{Due: \duedate}%
\fi
\noindent \textbf{Instructions.} Your solutions will be graded on
\emph{correctness} and \emph{clarity}. You should only submit work
that you believe to be correct; if you cannot solve a problem
completely, you will get significantly more partial credit if you
clearly identify the gap(s) in your solution. For this problem set, a
random subset of problems will be graded. You may keep working on
bonus problems till November 13. 

 % Problems marked with ``\mG'' are required for graduate
 % students. Undergraduate students will get bonus points for solving
 % them.
\medskip

\noindent You may collaborate with others on this problem set
% and consult external sources.
However, you must \textbf{\emph{write up your own solutions}} and
\textbf{\emph{list your collaborators}} for each problem.

\begin{questions}
\question[10] (Birthday bound) Fix a positive integer $N$, and
  $q\leq \sqrt{2N}$. Choose elements $y_1, \ldots, y_q$ uniformly and
  independently at random from a set of size $N$. Show that the
  probability that there exist distinct $i, j$ with $y_i = y_j$ is
  $\Theta(q^2/N)$. (Note: you need to prove both lower and upper bounds.)

\question (Density matrix) Given a state $\ket{\psi}$, we
  introduce a new representation, \emph{density matrix}, defined by
  $\rho := \ket{\psi}\bra{\psi}$. In general, given an ensemble of
  states $\{p_i, \ket{\psi_i}\}_{i=1}^k$ on a quantum register, where
  the register is in $\ket{\psi_i}$ with probability $p_i$, we define
  its density matrix as
  $\rho = \sum_{i =1}^k p_i\ket{\psi_i}\bra{\psi_i}$. We call a state
  pure if its density matrix $\rho$ can be writen as
  $\ket{\psi}\bra{\psi}$ for some $\ket{\psi}$. Otherwise wesay it is
  a \emph{mixed} state.
  
  \begin{parts}
  \part (Exercise) Compute the density matrices of
    $\ket{0}, \ket{+}, \frac{1}{\sqrt 3}\ket{0} + \frac{e^{\corr{i} \pi/8}\sqrt
      2}{\sqrt{3}} \ket{1}$.
  \part[5] A density matrix $\rho$ corresponds to a pure state if and
  only if $\rho = \kera{\psi}$. Show that $\rho$ corresponds to a pure
  state if and only if $Tr(\rho^2) = 1$.
  \part[5] Show that every $2\times 2$ density matrix $\rho$ can be
  expressed as an equally weighted mixture of pure states. That is
  $\rho = \frac 1 2 \kera{\psi_1} + \frac 1 2 \kera{\psi_2}$ (Note:
  the two states need not be orthogonal).
\end{parts}
  
\question (Grover's Search)
  
  \begin{parts}
  \part[10] (Claim in lower bound Proof) Let
    $O_r: \ket{x,y}\mapsto \ket{x,y\oplus f_r(x)}$, where
    $\forall x\in \bit^n$, $f_r(x) =1$ iff. $x=r$.
    $O_\emptyset: \ket{x,y}\mapsto \ket{x,y\oplus f_\emptyset(x)}$,
    where $\forall x\in \bit^n$, $f_\emptyset(x) =0$. Let
    $A = A_kOA_{k-1}\ldots A_1OA_0$ be a $k$-query quantum
    algorithm. For $j = 0,\ldots, k$, define
      \begin{align*}
        \ket{\psi_r^{(j)}} = O_rA_{j-1}\ldots O_rA_0\ket{0^n}, & \quad
        \ket{\phi^{(j)}} = O_\emptyset A_{j-1}\ldots
        O_\emptyset A_0\ket{0^n}        \, ; \\
        D_r^{(j)}: = \| \ket{\psi_r^{(j)}} - \ket{\phi^{
        (j)}}\|, & \quad E_r^{(j)}:= \|O_r\ket{\phi^{(j)}} - \ket{\phi^{(j)}}\| \, .
      \end{align*}
      Show the following:
      \begin{itemize}
      \item $ D_r^{(j)} \leq D_r^{(j-1)} + E_r^{(j-1)}, \forall j = 1,\ldots,k$.
      \item Suppose
        $\ket{\phi^{(j)}} = \sum_{x} \alpha_x^{(j)} \ket{x} $. Show
        that
        $E_r^{(j)}\leq 2|\alpha_r^{(j)}|, \forall j = 0,\ldots,k$.
      \end{itemize}
    \part[10] (Multiple marked items) Given $f: \bit^n \to \bit$. Let
      $A = f^{-1}(1) = \{x\in\bit^n: f(x) =1\}$ and
      $B = f^{-1}(0) = \{x\in\bit^n: f(x)=0\}$. Suppose $|A| \geq 1$
      is known. Given $O_f: \ket{x,y}\mapsto\ket{x,f(x)\oplus
        y}$. Design an algorithm that finds some $x\in A$ with
      $O(\sqrt{N/a})$ queries to $O_f$.
    \part (Bonus 10pts. Unknown size of marked items) What if $a$ is unknown in
      Part (b)? Design an algorithm that counts $a$ (approximately). 
    \part (Bonus 10pts. Fine performance of quantum search) Let
      $f:X\to \bit$ be a function such that $|f^{-1}(1)|=a$. Describe
      a $q$-query quantum algorithm that finds a $x$ s.t. $f(x)=1$
      with probability $\Omega(q^2a/N)$. (NB. this is also optimal.)
  \end{parts}

\question (Reduction and Hybrid argument) In this problem, we practice
  security proofs in cryptography. Let $G:\bit^{n} \to \bit^{n+1}$ be
  a pseudorandom generator that expands the seed by 1 bit.
  \begin{parts}
  \part[10] (Multi-sample security i.e., Parallel composition) Consider an
    adversary $A$ given $(r_1,\ldots,r_k)$ generated in one of two
    ways below:
    \begin{enumerate}[label=\roman*)]
    \item Pick $s_1,\ldots, s_k \gets \bit^n$ independently and
      uniformly at random, and output $r_i = G(s_i)$ for
      $i =1,\ldots, k$.
    \item Pick $r_1,\ldots, r_k \gets \bit^{n+1}$ independently and
      uniformly at random, and output them.  
    \end{enumerate}
    Show that no efficient $A$ can distinguish the two cases. Namely
    the parallel composition of $G$, $G'= G\|\ldots \|G$, is also a
    PRG.

  \part[10] (Sequential composition) Let $z[i]$ denotes the $i$th bit of a
    string $z$. Consider the construction $G^k$ below for increasing
    the expansion of $G$ (due to Blum and Micali): on random seed
    $s\gets \bit^n$, let $r_0 = s$.
    \begin{enumerate}[label=\roman*)]
    \item For $i =1, \ldots, k$, compute $y_i := G(r_{i-1})$, and let
      $r_i = y_i[2,\ldots, n]$. 
    \item Output $r = y_1[1]\|y_2[1]\|\ldots \|y_{k-1}[1]\|y_k[1] \in \bit^{k}$. 
    \end{enumerate}
    Namely, at each iteration, we save one bit of the output of $G$
    and use the rest as a seed for the next invocation. We prove that
    $G^k$ is a PRG for any polynomially bounded $k$ (in particular we
    get a lenth-doubling PRG by setting $k=2n$) by a \emph{hybrid
      argument}. For $j = 1, \ldots, k$, define $H^j$ as a revised
    generator:

    \begin{enumerate}[label=\roman*)]
    \item Pick random $z_j\gets \bit^j$ and $r_j \gets \bit^{n}$.
    \item For $i =j+1, \ldots, k$, compute
      $y_i := G(r_{i-1})$, and let $r_i = y_i[2,\ldots, n]$.
    \item Output $r = z_j\|y_{j+1}[1]\|\ldots \|y_{k-1}[1]\|y_k[1] \in \bit^{k}$. 
    \end{enumerate}
    Let $H_0 = G^k$ be the original construction. Note that the output
    in $H_k$ is a truly random $r\gets \bit^{k}$.

    Show that for all $j = 0,\ldots, k-1$,
    \begin{equation*}
      |\Pr_{r\gets H_j}[A(r) = 1] - \Pr_{r\gets H_{j+1}}[A(r) = 1]| \leq \negl(n) \, 
    \end{equation*}
    holds for any efficient $A$. Conclude from this that $G^k$ is a
    PRG. (Exercise: what is the advantage of $G^k$ over the parallel
    $G'$ from part a?)
    
  \part[10] Let $f: X\to Y$ be a function. We say $f$ is \emph{one-way} if
    $f(x)$ can be computed efficiently (i.e., $poly(|x|)$); but is
    hard to invert \emph{on-average}, i.e.,
    \begin{equation*}
      \Pr_{x\gets X}[f(x') = f(x) : x'\gets A(f(x))] \leq \negl(|x|) \, ,
    \end{equation*}
    holds for any poly-time algorithm $A$ (Note: it's crucial that $x$
    is generated randomly.)  Construct a one-way function from the PRG
    $G$, and give a proof that it is one-way. Let
    $F = \{F_k: \bit^n \to \bit^n\}_{k\in\bit^n}$ be a pseudorandom
    function family. Construct a one-way function from $F$.
    
  \part (Bonus 10pts. WeakOWF to StrongOWF) The definition of one-way
    function above is strong in the sense that no efficient algorithm
    can invert it with non-negligible probability. We consider a
    weaker notion of one-way function, where we consider it secure as
    long as no efficient algorithm can invert with probability close
    to 1: for any poly-time $A$
    \begin{equation*}
      \Pr_{x\gets X}[f(x') = f(x) : x'\gets A(f(x))] \leq 1 - 1/p(|x|) \, .
    \end{equation*}
    for some polynomoal $p(\cdot)$. Show that if there is a weak
    one-way function, then there is a strong one-way function. (Hint:
    given a weak one-way function $f$, consider
    $f': (x_1,\ldots, x_m) \mapsto (f(x_1),\ldots, f(x_m))$.)
  \end{parts}

\question (Small-range distribution) Let $D$ be a distribution on set
  $Y$. For an arbitrary set $X$ of size $N$, consider $D^X$ and
  $\srd{r}{D}$, which are both distributions on $\{f:X\to Y\}$. We
  claimed in class (without proof) that the output distributions of
  any $q$-query quantum algorithm to either $\srd{r}{D}$ or $D^X$ are
  $O(q^3/r)$-close.
  
    \begin{parts}
    \part[15] (Collision finding) We develop a quantum algorithm for
      finding collision in a function. Given a function $f:X\to Y$, we
      call a pair of inputs $(x\in X,x'\in X)$ a collision, if
      $x\neq x'$ and $f(x) = f(x')$. Let $f$ be a function with $k$
      collisions. Consider the following algorithm that uses $q$
      queries:

      \begin{itemize}
      \item Pick a random subset $S \subseteq X$ of size
        $\corr{0<}q_1 \leq q$. Query all inputs in $S$ \emph{classically},
        and find a collision in $S$.
      \item if there were none, then apply a quantum search algorithm
        to find collision between $S$ and $X\backslash S$. 
      \end{itemize}
      Describe how to implement the second step (i.e., building the
      Grover oracle). What is the success probability that the
      algorithm finds a collision? Optimize the choice of $q_1$ and
      $q_2$. (Hint: Problem 2.d may be helpful.)
    \part[15] (Quantum distinguisher) Describe a quantum algorithm
      that distinguishes $\srd{r}{D}$ from $D^X$ with probability
      $\Omega(q^3/r)$.
    \part[10] (Classical attack) Give a classical algorithm to distinguish
      $f\gets \srd{r}{D}$ from a truly random function with constant
      probability (e.g., $>1/4$) with as few queries as possible. (You
      do not need to prove its optimality, 5 bonus pts if you actually
      do.)
    \part (Bonus 10pts. Statistical Oracle Indistinguishability) We
      have proved in class that if an \emph{efficient} quantum
      algorithm distinguishes $D_0^X$ and $D_1^X$ with advantage
      $\veps$, then one can distinguish $D_1$ and $D_2$ with advantage
      $\Omega(\veps^2/q^3)$. Show that this holds
      \emph{statistically}.  Namely, if an \emph{unbounded} quantum
      algorithm distinguishes $D_0^X$ and $D_1^X$ with advantage
      $\veps$, then $D_1$ and $D_2$ must be $\Omega(\veps^2/q^3)$ far.
      
  \end{parts}

%  \question Simon variants
  
\end{questions}


\end{document}
